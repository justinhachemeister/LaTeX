\documentclass[12pt]{article}
\usepackage{amsmath}
\usepackage{times}
\usepackage{anyfontsize}
\usepackage{amsfonts}
\usepackage{mathtools}
\usepackage[paperheight=5in,paperwidth=5in,margin=0.6in]{geometry}
\usepackage[x11names]{xcolor}
\usepackage{tikz}
\usepackage{tcolorbox}
\setlength{\parindent}{0cm}
\usepackage{graphicx}
\graphicspath{{images/}}


\begin{document}
\thispagestyle{empty}
\begin{tikzpicture}[remember picture,overlay]
\path [left color=LightSkyBlue1, right color=white] (current page.north east)rectangle (current page.south west);
\end{tikzpicture}

%\vspace*{2cm}

\section{Statement:-}

\textbf{Let }$\boldsymbol{\left\{a_{1}, a_{2}, \cdots, a_{n} \right\},\ \left\{b_{1}, b_{2}, \cdots, b_{n}\right\} ,\ \cdots\cdots ,}$\linebreak $\boldsymbol{ \left\{l_{1}, l_{2}, \cdots, l_{n}\right\}}$ \textbf{be }$l $ \textbf{sets of positive real numbers and }$\boldsymbol{\alpha, \beta, \cdots,\lambda}$\textbf{ be positive rational numbers such that }$\boldsymbol{\alpha+\beta+\cdots+\lambda=1 .}$ \textbf{Then}
\begin{multline*}
\boldsymbol{a_{1}^{\alpha} b_{1}^{\beta} \cdots l_{1}^{\lambda}+a_{2}^{\alpha} b_{2}^{\beta} \cdots l_{2}^{\lambda}+\cdots + a_{n}^{\alpha} b_{n}^{\beta} \cdots l_{n}^{\lambda}}\\ 
\boldsymbol{ \leq \left( a_{1}+\cdots+a_{n} \right)^{\alpha} \left( b_{1}+\cdots +b_{n}\right)^{\beta} } \\ 
\boldsymbol{ \cdots \left(l_{1}+l_{2}+\cdots+l_{n}\right)^{\lambda} }
\end{multline*}
\textbf{The equality occurs when }$ \boldsymbol{a_i,b_i, \cdots,l_i} $ \textbf{are proportional.}




















\end{document}
