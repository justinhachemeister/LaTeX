\documentclass[12pt]{article}
\usepackage{amsmath}
\usepackage{times}
\usepackage{anyfontsize}
\usepackage{amsfonts}
\usepackage{mathtools}
\usepackage[paperheight=5in,paperwidth=5in,margin=0.6in]{geometry}
\usepackage[x11names]{xcolor}
\usepackage{tikz}
\usepackage{tcolorbox}
\setlength{\parindent}{0cm}
\usepackage{graphicx}
\usepackage{background}
\usepackage{blindtext}
\backgroundsetup{ scale=1, angle=0, opacity=1, contents={\begin{tikzpicture}[remember picture,overlay] \path [left color=LightSkyBlue1, right color=white] (current page.south west)rectangle (current page.north east);  
\end{tikzpicture}} }
\graphicspath{{images/}}
\setcounter{section}{4}
\pagestyle{empty}
\begin{document}
\color{black}
\pagebreak

\vspace*{0.35cm}
\section{Integral Version of H\"{o}lder  Inequality}
\subsection{Statement:-}
\textbf{Let }$\boldsymbol{ f_1,f_2,\cdots, f_n}$\textbf{ are n functions who are positive in the region }$\boldsymbol{[a,b] }$\textbf{. And }$\boldsymbol{p_1, p_2, \cdots,p_n}$\textbf{ be positive rational numbers such that   }$\boldsymbol{p_1+p_2+\cdots+p_n=1 .}$\textbf{ Then}$$\boldsymbol{\prod_{k=1}^n \left[\left(\int_a^b f_k (x)dx \right)^{p_k}\right]\geq \int_a^b\left[ \prod_{k=1}^n\left(f_k (x)\right)^{p_k}\right]dx}$$


\end{document}