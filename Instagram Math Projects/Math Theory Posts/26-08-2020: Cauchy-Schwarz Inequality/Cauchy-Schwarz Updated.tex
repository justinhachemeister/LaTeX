\documentclass[12pt]{article}
\usepackage[utf8]{inputenc}
\usepackage{xcolor}
\usepackage[T1]{fontenc}
\usepackage{pagecolor}
\usepackage{amssymb}
\usepackage{lmodern}

\usepackage{mathtools, nccmath}
\usepackage{courier}
\usepackage[overload]{empheq}
\usepackage[inline, shortlabels]{enumitem}
\usepackage{amsmath}
\usepackage{tcolorbox}
\usepackage{mathtools}
\definecolor{myyellow}{RGB}{225,225,20}
\definecolor{myred}{RGB}{255,50,40}
\definecolor{mygreen}{RGB}{100,255,100}
\definecolor{myblue}{RGB}{74,230,230}
\definecolor{mypurple}{RGB}{200,140,255}
\definecolor{myorange}{RGB}{255,150,50}
\color{white}
\pagecolor{black}
\title{}
\author{}
\date{}
\pagestyle{empty}
\begin{document}
\maketitle
\huge

\textcolor{myred}{$\qquad \!\!\!$ \underline{Cauchy-Schwarz Inequality}}

\small
\medskip
$$\text{@all.about.mathematics} \times
\text{@creative.math\_solving}$$

\large
$$\left(\textcolor{myblue}{\sum_{k=1}^{n}}\, \textcolor{myyellow}{a_{k}}^{2}\right)\left(\textcolor{myblue}{\sum_{k=1}^{n}}\, \textcolor{mygreen}{b_{k}}^{2}\right) \geq\left(\textcolor{myblue}{\sum_{k=1}^{n}}\textcolor{myyellow}{a_{k}} \,\textcolor{mygreen}{b_{k}}\right)^{2}$$
\LARGE

\vspace{3cm}
$$\quad \text{Swipe left} \implies \text{for explanation!} \:\:$$
\normalsize
$$\text{(including Titu's lemma and examples!)}$$

\newpage
\large
\section{Statement}
Let $\mathbf{u},\mathbf{v}$ belong to an inner product space $\mathbb{F}$. Then
\begin{equation} 
|\langle\mathbf{u}, \mathbf{v}\rangle|^{2} \leq\langle\mathbf{u}, \mathbf{u}\rangle \cdot\langle\mathbf{v}, \mathbf{v}\rangle
\end{equation}
Where $\langle \cdot , \cdot\rangle$ denotes the inner product.

\medskip
\noindent
If $\mathbf{F}=\mathbb{R}^n$, the inner product is the dot product.

\medskip
\noindent
In this case, the inequality can be restated as:

\medskip
\noindent
If $\{a_{1}, \cdots ,a_{n}\}$ and $\{b_{1}, \cdots ,b_{n}\}$ are two sets of real numbers,
\begin{equation}
\left({\sum_{k=1}^{n}}\, {a_{k}}^{2}\right)\left({\sum_{k=1}^{n}}\, {b_{k}}^{2}\right) \geq\left({\sum_{k=1}^{n}}{a_{k}} \,{b_{k}}\right)^{2}
\end{equation}
Equality in $(1)$ holds if and only if $\mathbf{u},\mathbf{v}$ are linearly dependent.

\medskip
\noindent
Therefore, equality in $(2)$ holds if and only if
$$a_k=c\,(b_k) \quad \forall k\in\{1,\cdots,n\}$$
For some constant $c$

\newpage
\section{Proof of special case}
Let's consider the case where $n=2$.

\medskip
\noindent
Let the 2 sets of real numbers be $\{a,b\}$ and $\{c,d\}$. 

\medskip
\noindent
Then the theorem states that
\begin{equation} 
\left(a^{2}+b^{2}\right)\left(c^{2}+d^{2}\right) \geq(a c+b d)^{2}
\end{equation}
To prove $(3)$, recall the Fibonacci-Brahmagupta identity:
$$\left(a^{2}+b^{2}\right)\left(c^{2}+d^{2}\right)=(a c+b d)^{2}+(a d-b c)^{2}$$
Then $(3)$ follows since
$$(a d-b c)^{2}\geq 0$$
Note that equality in $(3)$ holds if and only if 
$$ad=bc \Longleftrightarrow a=\lambda c \:,\: b=\lambda d$$
For some constant $\lambda$.

\medskip
\noindent
This special case can also be proven by the AM-GM inequality.

\newpage
\section{Proof of general case}
Let 
$$A=\sum_{k=1}^{n} {a_{k}}^{2}\:,\:\: B=\sum_{k=1}^{n} {b_{k}}^{2}\:,\:\: C=\sum_{k=1}^{n} a_{k} \,b_{k}$$
Then the theorem is equivalent to
\begin{equation}
C^2\leq A B
\end{equation}If $B=0,$ $b_{i}=0$ for $i=1,\cdots ., n$. So $C=0$ and (4) is true. 

\medskip
\noindent
Now, we only need to consider the case $ B >0 $. We have
\begin{align*}
0 \leq \sum_{i=1}^{n}\left(B a_{i}-C b_{i}\right)^{2} &=\sum_{i=1}^{n}\left(B^{2} a_{i}^{2}-2 B C a_{i} b_{i}+C^{2} b_{i}^{2}\right) \\
&=B^{2} \sum_{i=1}^{n} a_{i}^{2}-2 B C \sum_{i=1}^{n} a_{i} b_{i}+C^{2} \sum_{i=1}^{n} b_{i}^{2} \\
&=B\left(A B-C^{2}\right)
\end{align*}
Since $B>0, A B-C^{2} \geq 0 $ and we obtain $(4)$. 

\medskip 
\noindent 
Moreover, equality in $(4)$ holds if and only if
$$\sum_{i=1}^{n}\left(B a_{k}-C b_{k}\right)^{2}=0 \Longleftrightarrow {a_{k}}=\frac{C}{B}\, {b_{k}} \quad \forall k\in\{1, \cdots, n\}$$

\newpage
\section{Example: IrMO 2004 P5}
Let $a,b\ge 0$. Prove that
$$\sqrt{2}\left(\sqrt{a(a+b)^3}+b\sqrt{a^2+b^2}\right)\le 3(a^2+b^2)$$    
Rewriting the terms in the L.H.S,
$$\sqrt{2a(a+b)^3}=\sqrt{2a(a+b)}\times \sqrt{(a+b)^2}$$and$$b\sqrt{2(a^2+b^2)}=\sqrt{2b^2}\times \sqrt{a^2+b^2}$$
Applying the Cauchy-Schwarz Inequality we get:
\begin{align*}
\sqrt{2a(a+b)^3}+\sqrt{2b^2(a^2+b^2)}&\leq \sqrt{\big(2a(a+b)+2b^2\big)\big((a+b)^2+a^2+b^2\big)}\\
&= \sqrt{\big(2a^2+2b^2+2ab\big)\big(2a^2+2b^2+2ab\big)}\\
&=2a^2+2b^2+2ab\leq 3(a^2+b^2)
\end{align*}
As $a^2+b^2\geq 2ab$
\newpage
\normalsize
\section{Example: Japan TST 2004}
Let $a,b,c$ be positive reals with $a+b+c=1$. Prove that $$\frac{1+a}{1-a}+\frac{1+b}{1-b}+\frac{1+c}{1-c}\leq \frac{2a}{b}+\frac{2b}{c}+\frac{2c}{a}$$
Note that
$$\sum\limits_{cyc}\frac{1+a}{1-a}=3+\sum\limits_{cyc}\frac{2a}{1-a}\leq  \sum\limits_{cyc}\frac{2a}{b}\implies \frac{3}{2}\leq\sum\limits_{cyc}\bigg(\frac{a}{b}-\frac{a}{b+c}\bigg)=\sum\limits_{cyc}\frac{ca}{b(b+c)}$$
By the Cauchy-Schwarz inequality,
$$\bigg(\sum\limits_{cyc}\frac{c^2a^2}{abc(b+c)}\bigg)\bigg(\sum\limits_{cyc} abc(b+c) \bigg) \geq (ab+bc+ca)^2$$ $$\implies \sum\limits_{cyc}\frac{c^2a^2}{abc(b+c)} \geq \frac{(ab+bc+ca)^2}{2abc (a+b+c)}$$Now$$(ab+bc+ca)^2=a^2b^2+b^2c^2+c^2a^2+2abc(a+b+c)\geq 3abc(a+b+c)$$
Hence
$$\sum\limits_{cyc}\frac{ca}{b(b+c)}\geq \frac{3}{2}$$
\newpage 
\section{Titu's lemma}
This lemma is named after Titu Andreescu and is a direct consequence of the Cauchy-Schwarz inequality.

\medskip
\noindent
To obtain the lemma, substitute the below into the inequality
$$a_{k}=\frac{x_{k}}{\sqrt{y_{k}}} \quad b_{k}=\sqrt{y_{k}}$$ 
We have
$$\left(\frac{{x_{1}}^{2}}{y_{1}}+\frac{{x_{2}}^{2}}{y_{2}}+\cdots+\frac{{x_{n}}^{2}}{y_{n}}\right)\left(y_{1}+y_{2}+\cdots+y_{n}\right) \geq\left(x_{1}+x_{2}+\cdots+x_{n}\right)^{2}$$
It is often written in the form
$$\frac{{x_{1}}^{2}}{y_{1}}+\frac{{x_{2}}^{2}}{y_{2}}+\cdots+\frac{{x_{n}}^{2}}{y_{n}} \geq \frac{\left(x_{1}+x_{2}+\cdots+x_{n}\right)^{2}}{\left(y_{1}+y_{2}+\cdots+y_{n}\right)}$$
This inequality is very useful in mathematical Olympiads, as we'll see in the next 2 slides.

\newpage
\section{Example: Nesbitt's Inequality}
Nesbitt's inequality states that for positive real numbers $a,b,c,$
\begin{equation}
\frac{a}{b+c}+\frac{b}{a+c}+\frac{c}{b+a} \geq \frac{3}{2}
\end{equation}
To apply Titu's lemma, we rewrite $(5)$:
$$\frac{a^{2}}{a b+a c}+\frac{b^{2}}{a b+b c}+\frac{c^{2}}{b c+a c}\geq \frac{3}{2}$$
By Titu's lemma, we have
$$\frac{a^{2}}{a b+a c}+\frac{b^{2}}{a b+b c}+\frac{c^{2}}{b c+a c} \geq \frac{(a+b+c)^{2}}{2(a b+b c+c a)}$$
To prove $(5)$, note that
$$\frac{(a+b+c)^{2}}{2(a b+b c+c a)}\geq\frac{3}{2} \Longleftrightarrow (a+b+c)^{2} \geq 3(a b+b c+c a)$$
$$\Longleftrightarrow a^{2}+b^{2}+c^{2} \geq a b+b c+c a $$
\begin{equation} 
\Longleftrightarrow (a-b)^{2}+(b-c)^{2}+(c-a)^{2} \geq 0
\end{equation}
Since $(6)$ is true, $(5)$ follows.

\newpage
\section{Example: IMO 1995 P2}
If $a,b,c$ are positive real numbers and $abc=1$, prove that
\begin{equation} 
\frac{1}{a^{3}(b+c)}+\frac{1}{b^{3}(a+c)}+\frac{1}{c^{3}(a+b)} \geq \frac{3}{2}
\end{equation}
We will solve this question using Titu's lemma. Rewriting (7), 
$$\frac{\frac{1}{a^{2}}}{a b+a c}+\frac{\frac{1}{b^{2}}}{b a+b c}+\frac{\frac{1}{c^{2}}}{c a+c b} \geq \frac32$$
By Titu's lemma,
\begin{equation} 
\frac{\frac{1}{a^{2}}}{a b+a c}+\frac{\frac{1}{b^{2}}}{b a+b c}+\frac{\frac{1}{c^{2}}}{c a+c b} \geq
\frac{\left(\frac{1}{a}+\frac{1}{b}+\frac{1}{c}\right)^{2}}{2(a b+b c+c a)}
\end{equation}
Using the condition $abc=1$, we have
$$\left(\frac{1}{a}+\frac{1}{b}+\frac{1}{c}\right)^{2}=(a b+b c+c a)^{2}$$
By the AM-GM inequality, the L.H.S of $(8)$
is
$$\geq \frac{(a b+b c+c a)}{2} \geq  \frac{\sqrt[\leftroot{2}\uproot{2}3]{(a b c)^{2}}}{2}=\frac{3}{2}$$
Therefore the problem is solved.

\newpage
\Large
\section{Exercise to the readers :D}
Congratulations if you've read till the end!

\bigskip
\noindent
Here's an easy problem that can be solved by applying the Cauchy-Schwarz inequality or Titu's lemma!
 
\bigskip
\begin{tcolorbox}
Given that $a,b,c>0$ and $a+b+c=1$.

\medskip
\noindent
Find the minimum value of
$$a^2+2b^2+c^2$$
\end{tcolorbox}
\bigskip
\noindent
\LARGE
$$\text{Comment your answers!}$$
\end{document} 


