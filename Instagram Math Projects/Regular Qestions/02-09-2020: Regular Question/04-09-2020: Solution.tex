\documentclass[12pt]{article}
\usepackage[paperheight=7in,paperwidth=7in,margin=0.55in]{geometry}
%\usepackage{fullpage}

\usepackage{mathtools}
\usepackage{amsfonts}
\usepackage{amsmath}
\usepackage{MnSymbol}
\usepackage[x11names]{xcolor}
\usepackage{tikz}
\usepackage{titlesec}
\usepackage{background}
\usepackage{blindtext}
\backgroundsetup{ scale=1, angle=0, opacity=1, contents={\begin{tikzpicture}[remember picture,overlay] \path[inner color = DarkOliveGreen1,outer color = SpringGreen1] (current page.south west)rectangle (current page.north east);  \end{tikzpicture}} }

\titleformat*{\section}{\Huge\bfseries}

\pagestyle{empty}
\begin{document}
\pagebreak 
%\vspace*{4cm}
	
\Huge{\section{Part 1 $\boldsymbol{\coloneq}$}
Let $p $ be a prime. Therefore according to \textbf{Fermat's Little Theorem} $$p\mid a^p-a$$Given that $p\mid (a^p -1)$. Hence $$p\mid (a^p-1)-(a^p-a)\implies p\mid a-1$$Now,\begin{align*}
a^p-1\ & =(a-1)(a^{p-1}+a^{p-2}+\cdots+1)\\ 
& = (a-1)(a^{p-1}-1+a^{p-2}-1\\ 
& \qquad \qquad \qquad\quad +\cdots+a-1+p)\\ 
\end{align*}


\begin{flushright}
\textbf{Swipe }$\boldsymbol{\ggg}$
\end{flushright}

As $a\equiv 1\ (\text{ mod } p\ )\implies a^k\equiv 1\ (\text{ mod } p\ )$ $\forall\ k\in\mathbb{Z^+}$. Hence $p\mid a^{p-1}-1,$ $p\mid a^{p-2}-1$, $\cdots$, $p\mid a^{2}-1$. Therefore  $$p\mid a^{p-1}-1+a^{p-2}-1+\cdots+a-1+p$$Hence $$p^2\mid a^p-1$$Hence all the primes show the propert $P $. [Proved]
\vfill

\begin{center}
\textbf{Swipe }$\boldsymbol{\implies}$\textbf{ For Part 2}
\end{center}

\pagebreak 

\section{Part 2 $\boldsymbol{\coloneq}$}
Let $p $, $q $ are primes such that $pq\mid a^{pq}-1$.

As we proved before $p^2\mid a^p-1$ and\linebreak $q^2\mid a^q-1$. 

Now, \begin{align*}
a^{pq}-1\ & =(a^p-1)\big((a^{p})^{q-1}+(a^{p})^{q-2}\\ 
& \qquad \qquad \qquad\quad+ \cdots+a^p+1\big)
\end{align*}Hence $p^2\mid a^{pq}-1$. Again\begin{align*}
a^{pq}-1\ & =(a^q-1)\Big((a^{q})^{p-1}+(a^{q})^{p-2}\\ 
& \qquad \qquad \qquad\quad+ \cdots+a^q+1\Big)
\end{align*}

\begin{flushright}
\textbf{Swipe }$\boldsymbol{\ggg}$
\end{flushright}
Hence $q^2\mid a^{pq}-1$. Therefore $$p^2q^2\mid a^{pq}-1$$As there are infinitely many primes. Hence there are infinitely many composite numbers which follows the property $P$

\vfill

\begin{flushright}
\textbf{Swipe }$\boldsymbol{\ggg}$
\end{flushright}
}

\end{document}
