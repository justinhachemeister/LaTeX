\documentclass[12pt]{article}
\usepackage[paperheight=13in,paperwidth=13in,margin=1in]{geometry}
%\usepackage{fullpage}
\usepackage{mathtools}
\usepackage{amsfonts}
\usepackage{amsmath}

\begin{document}
	
\Huge{Let $ P(x,y)$ be the assertion $$P(x,y)\implies x^2y^2(f(x + y)-f(x) - f(y)) = 3(x + y)f(x)f(y)$$\par
Hence$$P(x,0)\implies xf(x)f(0) = 0\ \forall\ x$$If $ f(0)\neq 0$, then $$ f(x) = 0\ \forall\ x\ne 0$$
Then putting $x=1,y=-1$ we get $$ P(1, - 1)\implies f(0)-f(1)-f(-1)=0\implies f(0) = 0$$ and we got $$ f(x) = 0\ \forall\ x$$ which indeed is a solution.\par
So we'll from now consider $ f(0) = 0$ and $ f(x)$ not all zero.\par
Now,$$ P(x, - x)\implies x^4\left( f(0)-f(x)-f(-x)\right) =0\implies f(-x)=-f(x)\ \forall\ x\ne 0$$\par
Let $ f(1) = a$. Now,$$ P(1,1) \implies f(2)-2f(1)=6\left(f(1)\right)^2\implies f(2) = 6a^2 + 2a$$
Again
\begin{align*}
P(2, - 1) & \implies 4(f(1)-f(2)-f(-1))=3f(2)f(-1)\\
&\implies 4(2a - f(2)) = - 3af(2)\\
&\implies f(2)(3a - 4) = - 8a
\end{align*}\par
So $$ (6a^2 + 2a)(3a - 4) = - 8a \iff a^2(a - 1) = 0$$\par
If $ a = 0$, then:
\begin{align*}
P(x,1) &\implies x^2(f(x + 1) - f(x)-f(1)) =3(x+1)f(x)f(1)\\
&\implies x^2(f(x+1)-f(x))=0
\end{align*}
and so $$ f(x + 1) = f(x)\ \forall\ x$$\par
Now,$$ P(x + 1,y) \implies (x + 1)^2y^2(f(x + y) - f(x) - f(y)) = 3(x + y + 1)f(x)f(y)$$
Subtracting it from the given equation we get$$(2x + 1)y^2(f(x + y) - f(x) - f(y)) = 3f(x)f(y)$$\par
Now,\begin{align*}
& x^2y^2(f(x + y)-f(x) - f(y)) = 3(x + y)f(x)f(y)\\
\implies & x^2y^2(2x+1)(f(x + y)-f(x) - f(y))=3(2x+1)(x + y)f(x)f(y)\\
\implies & 3x^2f(x)f(y)=3(2x+1)(x+y)f(x)f(y)\\
\implies & x^2f(x)f(y)=(2x+1)(x+y)f(x)f(y)\ \forall\ x,y
\end{align*}
Putting $x=y$ we get:-$$2x(2x+1)(f(x))^2=x^2(f(x))^2\implies (f(x))^2(3x^2+2x)=0$$Hence we can deduce $$ f(x) = 0\ \forall\ x$$So $ f(1) = 1$ and therefore $ f(2) = 8$
\begin{align*}
P(x,1) & \implies x^2(f(x + 1) - f(x) - 1) = 3(x + 1)f(x)\\
&\implies f(x + 1) = \frac {x^2 + 3x + 3}{x^2}f(x) + 1\ \forall\ x\ne 0
\end{align*}
\begin{align*}
P(x + 1,1)& \implies  (x+1)^2(f(x+2)-f(x+1)-f(1))=3(x+2)f(x+1)f(1) \\
&\implies f(x + 2) = \frac {x^2 + 5x + 7}{(x + 1)^2}f(x + 1) + 1\\
&\implies f(x + 2)= \frac {(x^2 + 3x + 3)(x^2 + 5x + 7)}{x^2(x + 1)^2}f(x) + \frac {x^2 + 5x + 7}{(x + 1)^2} + 1
\end{align*} $\forall x\notin\{ - 1,0\}$\par
\begin{align*}
P(x,2)& \implies 4x^2(f(x+2)-f(x)-f(2))=3(x+2)f(x)f(2)\\
&\implies f(x + 2) = \frac {x^2 + 6x + 12}{x^2}f(x) + 8\ \forall\ x\ne 0
\end{align*}
And so $$ \frac {(x^2 + 3x + 3)(x^2 + 5x + 7)}{x^2(x + 1)^2}f(x) + \frac {x^2 + 5x + 7}{(x + 1)^2} + 1 = \frac {x^2 + 6x + 12}{x^2}f(x) + 8$$And simplifying this [it is very boring stuff so i am not showing that, you can verify that], we get $$ (6x + 9)f(x) = x^3(6x + 9)$$ and so $$ f(x) = x^3 \forall x\notin\left\{ - \frac{3}{2}, - 1, 0\right\}$$We already know that $ f(0) = 0$ and $ f(1) = 1$ and $$ P\left(  - 1, - \frac{1}{2}\right) \implies f\left(  - \frac{3}{2}\right)  = - \frac{27}{8}$$And so $ f(x) = x^3\ \forall\ x$ which indeed is a solution.\par 
Hence$ f(x) = x^3$ and $f(x)=0\ \forall\ x$ are the solutions of the functional equation.}
\end{document}