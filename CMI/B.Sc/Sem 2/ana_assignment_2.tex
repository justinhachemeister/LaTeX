\documentclass[a4paper, 11pt]{article}
\usepackage{comment} % enables the use of multi-line comments (\ifx \fi) 
\usepackage{fullpage} % changes the margin
\usepackage[a4paper, total={7in, 10in}]{geometry}
\usepackage[fleqn]{amsmath,mathtools}
\usepackage{amssymb,amsthm}  % assumes amsmath package installed
\usepackage{float}
\usepackage{xcolor}
\usepackage{mdframed}
\usepackage[shortlabels]{enumitem}
\usepackage{indentfirst}
\usepackage{hyperref}
\hypersetup{
	colorlinks=true,
	linkcolor=blue,
	filecolor=magenta,      
	urlcolor=blue!70!red,
	pdftitle={Analysis 2 - Assignment 2}, %%%%%%%%%%%%%%%%   WRITE ASSIGNMENT PDF NAME  %%%%%%%%%%%%%%%%%%%%
}
\usepackage[most,many,breakable]{tcolorbox}

%---------------------------------------
% BlackBoard Math Fonts :-
%---------------------------------------

%Captital Letters
\newcommand{\bbA}{\mathbb{A}}	\newcommand{\bbB}{\mathbb{B}}
\newcommand{\bbC}{\mathbb{C}}	\newcommand{\bbD}{\mathbb{D}}
\newcommand{\bbE}{\mathbb{E}}	\newcommand{\bbF}{\mathbb{F}}
\newcommand{\bbG}{\mathbb{G}}	\newcommand{\bbH}{\mathbb{H}}
\newcommand{\bbI}{\mathbb{I}}	\newcommand{\bbJ}{\mathbb{J}}
\newcommand{\bbK}{\mathbb{K}}	\newcommand{\bbL}{\mathbb{L}}
\newcommand{\bbM}{\mathbb{M}}	\newcommand{\bbN}{\mathbb{N}}
\newcommand{\bbO}{\mathbb{O}}	\newcommand{\bbP}{\mathbb{P}}
\newcommand{\bbQ}{\mathbb{Q}}	\newcommand{\bbR}{\mathbb{R}}
\newcommand{\bbS}{\mathbb{S}}	\newcommand{\bbT}{\mathbb{T}}
\newcommand{\bbU}{\mathbb{U}}	\newcommand{\bbV}{\mathbb{V}}
\newcommand{\bbW}{\mathbb{W}}	\newcommand{\bbX}{\mathbb{X}}
\newcommand{\bbY}{\mathbb{Y}}	\newcommand{\bbZ}{\mathbb{Z}}

%---------------------------------------
% MathCal Fonts :-
%---------------------------------------

%Captital Letters
\newcommand{\mcA}{\mathcal{A}}	\newcommand{\mcB}{\mathcal{B}}
\newcommand{\mcC}{\mathcal{C}}	\newcommand{\mcD}{\mathcal{D}}
\newcommand{\mcE}{\mathcal{E}}	\newcommand{\mcF}{\mathcal{F}}
\newcommand{\mcG}{\mathcal{G}}	\newcommand{\mcH}{\mathcal{H}}
\newcommand{\mcI}{\mathcal{I}}	\newcommand{\mcJ}{\mathcal{J}}
\newcommand{\mcK}{\mathcal{K}}	\newcommand{\mcL}{\mathcal{L}}
\newcommand{\mcM}{\mathcal{M}}	\newcommand{\mcN}{\mathcal{N}}
\newcommand{\mcO}{\mathcal{O}}	\newcommand{\mcP}{\mathcal{P}}
\newcommand{\mcQ}{\mathcal{Q}}	\newcommand{\mcR}{\mathcal{R}}
\newcommand{\mcS}{\mathcal{S}}	\newcommand{\mcT}{\mathcal{T}}
\newcommand{\mcU}{\mathcal{U}}	\newcommand{\mcV}{\mathcal{V}}
\newcommand{\mcW}{\mathcal{W}}	\newcommand{\mcX}{\mathcal{X}}
\newcommand{\mcY}{\mathcal{Y}}	\newcommand{\mcZ}{\mathcal{Z}}



%---------------------------------------
% Bold Math Fonts :-
%---------------------------------------

%Captital Letters
\newcommand{\bmA}{\boldsymbol{A}}	\newcommand{\bmB}{\boldsymbol{B}}
\newcommand{\bmC}{\boldsymbol{C}}	\newcommand{\bmD}{\boldsymbol{D}}
\newcommand{\bmE}{\boldsymbol{E}}	\newcommand{\bmF}{\boldsymbol{F}}
\newcommand{\bmG}{\boldsymbol{G}}	\newcommand{\bmH}{\boldsymbol{H}}
\newcommand{\bmI}{\boldsymbol{I}}	\newcommand{\bmJ}{\boldsymbol{J}}
\newcommand{\bmK}{\boldsymbol{K}}	\newcommand{\bmL}{\boldsymbol{L}}
\newcommand{\bmM}{\boldsymbol{M}}	\newcommand{\bmN}{\boldsymbol{N}}
\newcommand{\bmO}{\boldsymbol{O}}	\newcommand{\bmP}{\boldsymbol{P}}
\newcommand{\bmQ}{\boldsymbol{Q}}	\newcommand{\bmR}{\boldsymbol{R}}
\newcommand{\bmS}{\boldsymbol{S}}	\newcommand{\bmT}{\boldsymbol{T}}
\newcommand{\bmU}{\boldsymbol{U}}	\newcommand{\bmV}{\boldsymbol{V}}
\newcommand{\bmW}{\boldsymbol{W}}	\newcommand{\bmX}{\boldsymbol{X}}
\newcommand{\bmY}{\boldsymbol{Y}}	\newcommand{\bmZ}{\boldsymbol{Z}}
%Small Letters
\newcommand{\bma}{\boldsymbol{a}}	\newcommand{\bmb}{\boldsymbol{b}}
\newcommand{\bmc}{\boldsymbol{c}}	\newcommand{\bmd}{\boldsymbol{d}}
\newcommand{\bme}{\boldsymbol{e}}	\newcommand{\bmf}{\boldsymbol{f}}
\newcommand{\bmg}{\boldsymbol{g}}	\newcommand{\bmh}{\boldsymbol{h}}
\newcommand{\bmi}{\boldsymbol{i}}	\newcommand{\bmj}{\boldsymbol{j}}
\newcommand{\bmk}{\boldsymbol{k}}	\newcommand{\bml}{\boldsymbol{l}}
\newcommand{\bmm}{\boldsymbol{m}}	\newcommand{\bmn}{\boldsymbol{n}}
\newcommand{\bmo}{\boldsymbol{o}}	\newcommand{\bmp}{\boldsymbol{p}}
\newcommand{\bmq}{\boldsymbol{q}}	\newcommand{\bmr}{\boldsymbol{r}}
\newcommand{\bms}{\boldsymbol{s}}	\newcommand{\bmt}{\boldsymbol{t}}
\newcommand{\bmu}{\boldsymbol{u}}	\newcommand{\bmv}{\boldsymbol{v}}
\newcommand{\bmw}{\boldsymbol{w}}	\newcommand{\bmx}{\boldsymbol{x}}
\newcommand{\bmy}{\boldsymbol{y}}	\newcommand{\bmz}{\boldsymbol{z}}

\definecolor{mytheorembg}{HTML}{F2F2F9}
\definecolor{mytheoremfr}{HTML}{00007B}


\tcbuselibrary{theorems,skins,hooks}
\newtcbtheorem{problem}{Problem}
{%
	enhanced,
	breakable,
	colback = mytheorembg,
	frame hidden,
	boxrule = 0sp,
	borderline west = {2pt}{0pt}{mytheoremfr},
	sharp corners,
	detach title,
	before upper = \tcbtitle\par\smallskip,
	coltitle = mytheoremfr,
	fonttitle = \bfseries\sffamily,
	description font = \mdseries,
	separator sign none,
	segmentation style={solid, mytheoremfr},
}
{p}

\renewcommand{\thesubsection}{\thesection.\alph{subsection}}
\newcommand{\Z}{\mathbb{Z}}
\newcommand{\N}{\mathbb{N}}
\newcommand{\C}{\mathbb{C}}
\newcommand{\R}{\mathbb{R}}
\newcommand{\Qed}{\begin{flushright}\qed\end{flushright}}
\newcommand{\sol}[1]{\begin{solution}#1\end{solution}\Qed}
\newcommand{\parinn}{\setlength{\parindent}{1cm}}
\newcommand{\parinf}{\setlength{\parindent}{0cm}}

% The problem environment introduced.
%\newenvironment{problem}[2][Problem]
%{ \begin{mdframed}[backgroundcolor=gray!20] \textbf{#1 #2} \\}
	%	{  \end{mdframed}}

% Define solution environment
\newenvironment{solution}
{\textbf{\textit{Solution: }}\setlength{\parindent}{1cm}}
{}

%\renewcommand{\qed}{\quad\qedsymbol}

\setlength{\parindent}{0pt}

%%%%%%%%%%%%%%%%%%%%%%%%%%%%%%%%%%%%%%%%%%%%%%%%%%%%%%%%%%%%%%%%%%%%%%%%%%%%%%%%%%%%%%%%%%%%%%%%%%%%%%%%%%%%%%%%%%%%%%%%%%%%%%%%%%%%%%%%

\begin{document}
	%Header-Make sure you update this information!!!!
	
	%%%%%%%%%%%%%%%%%%%%%%%%%%%%%%%%%%%%%%%%%%%%%%%%%%%%%%%%%%%%%%%%%%%%%%%%%%%%%%%%%%%%%%%%%%%%%%%%%%%%%%%%%%%%%%%%%%%%%%%%%%%%%%%%%%%%%%%%
	
	\textsf{\noindent \large\textbf{Soham Chatterjee} \hfill \textbf{Assignment - 2}\\
		Email: \href{sohamc@cmi.ac.in}{sohamc@cmi.ac.in} \hfill Roll: BMC202175\\
		\normalsize Course: Analysis 2 \hfill Date: April 4, 2022 \\
		\noindent\rule{7in}{2.8pt}}
	
	%%%%%%%%%%%%%%%%%%%%%%%%%%%%%%%%%%%%%%%%%%%%%%%%%%%%%%%%%%%%%%%%%%%%%%%%%%%%%%%%%%%%%%%%%%%%%%%%%%%%%%%%%%%%%%%%%%%%%%%%%%%%%%%%%%%%%%%%
	% Problem 1
	%%%%%%%%%%%%%%%%%%%%%%%%%%%%%%%%%%%%%%%%%%%%%%%%%%%%%%%%%%%%%%%%%%%%%%%%%%%%%%%%%%%%%%%%%%%%%%%%%%%%%%%%%%%%%%%%%%%%%%%%%%%%%%%%%%%%%%%%
	
	\begin{problem}{Rudin Chapt. 9 Problem 6}{p1%Problem Name
		}
		%Problem		
		If $f(0,0)=0$ and
		$$
		f(x, y)=\frac{x y}{x^{2}+y^{2}} \quad \text { if }(x, y) \neq(0,0)
		$$
		prove that $\left(D_{1} f\right)(x, y)$ and $\left(D_{2} f\right)(x, y)$ exist at every point of $R^{2}$, although $f$ is not continuous at $(0,0)$.
	\end{problem}
	
	\sol{When $(x,y)\neq (0,0)$ then $$(D_1f)(x,y)=\frac{y(y^2-x^2)}{(x^2+y^2)^2}\text{ and }(D_2f)(X,y)=\frac{x(x^2-y^2)}{(x^2+y^2)^2}$$Now at $(0,0)$ \begin{align*}
		(D_1f)(0,0) & =\lim_{h\to 0}\frac{f(h,0)-f(0,0)}{|h|}              & (D_2f)(0,0) & =\lim_{k\to0} \frac{f(0,k)-f(0,0)}{|k|}                \\
		            & =\lim_{h\to0}\frac{\frac{h\times 0}{h^2+0} - 0}{|h|} &             & = \lim_{k\to 0}\frac{\frac{0\times k}{0+k^2} - 0}{|k|} \\
		            & =\lim_{h\to 0}\frac{0}{|h|}                          &             & =\lim_{k\to 0}\frac{0}{|k|}                            \\
		            & =0                                                   &             & =0
	\end{align*}Hence $(D_1f)(x,y)$ and $(D_2f)(x,y)$ exists at every point of $\bbR^2$. 

Now if we approach $(0,0)$ along the line then it  approaches to 0. But if we approach $(0,0)$ along the line $y=x$ then $$\lim_{h\to 0}f(h,h)=\lim_{h\to 0}\frac{h^2}{2h^2}=\frac12$$ Hence $f$ is not continuous at (0,0)
		
	}
	
	
	%%%%%%%%%%%%%%%%%%%%%%%%%%%%%%%%%%%%%%%%%%%%%%%%%%%%%%%%%%%%%%%%%%%%%%%%%
	% Problem 2
	%%%%%%%%%%%%%%%%%%%%%%%%%%%%%%%%%%%%%%%%%%%%%%%%%%%%%%%%%%%%%%%%%%%%%%%%%
	
	\begin{problem}{Rudin Chapt. 9 Problem 7}{p2%Problem Name
		}
		%Problem		
		Suppose that $f$ is a real-valued function defined in an open set $E \subset R^{n}$, and that the partial derivatives $D_{1} f, \ldots, D_{n} f$ are bounded in $E$. Prove that $f$ is continuous in $E$.\parinn
		
		Hint: Proceed as in the proof of Theorem 9.21.
	\end{problem}
	
	\sol{Let $a$ is any arbitrary point in $E$. Since $E$ is an open set there exists an open ball $B_r(a)$ centered at $a$ of radius $r>0$ inside $E$. We have to show that $\forall\ \epsilon>0$ $\exists \ \delta>0$ $|f(a+h)-f(a)|<\epsilon$ wherever $\|h\|<\delta$ where $a+h\in B_r(a)$. 
		
		Now any open ball $B_r(a)$ centered at $a$ of radius $r>0$ in $\bbR^n$ is a convex set. Because if we take any two points $x,y\in B_r(a)$ then for any $\theta\in (0,1)$ $$\|(\theta x+(1-\theta) y)-a\|=\|\theta (x-a)+(1-\theta)(y-a)\|\leq \theta\|x-a\|+(1-\theta)\|y-a\|< \theta r+(1-\theta )r=r$$and henceforth $\theta x+(1-\theta)y\in B_r(a)$. Therefore any open ball in $\bbR^n$ is a convex set.
		
		Now. Let $h=\sum\limits_{i=1}^n h_ie_i$ where $e_i$ is the $i-$th vector of the standard basis of $\bbR^m$. \begin{align*}
		f(a+h)-f(a) & =f\left(a+\sum_{i=1}^nh_ie_i\right)-f(a)                                                                                                                                                                                                      \\
		            & =\left[f\left(a+\sum_{i=1}^nh_ie_i\right)-f\left(a+\sum_{i=1}^{n-1}h_ie_i\right)\right]+\left[f\left(a+\sum_{i=1}^{n-1}h_ie_i\right)-f\left(a+\sum_{i=1}^{n-2}h_ie_i\right)\right]\\
		            & \qquad\qquad\qquad\qquad\qquad\qquad\qquad\qquad\qquad\qquad\qquad\qquad+\cdots+\left[f\left(a+h_1e_1\right)-f\left(a\right)\right] \\
		            & =\sum_{k=1}^n\left[f\left(a+\sum_{i=1}^kh_ie_i\right)-f\left(a+\sum_{i=1}^{k-1}h_ie_i\right)\right]
	\end{align*} Now each $f\left(a+\sum\limits_{i=1}^kh_ie_i\right)-f\left(a+\sum\limits_{i=1}^{k-1}h_ie_i\right)$ is a one-variable function from $\bbR$ to $\bbR$. By Mean Value Theorem $\exists \ \theta_k\in \left(0,1\right)$ such that $$f\left(a+\sum\limits_{i=1}^kh_ie_i\right)-f\left(a+\sum\limits_{i=1}^{k-1}h_ie_i\right)=\left.\frac{\partial f}{\partial x_k}\right|_{v_k}h_k$$where $v_k= a+\sum\limits_{i=1}^{k-1} h_ie_i+\theta_k h_k e_k$. Hence $$f(a+h)-f(a)=\sum_{i=1}^n \left.\frac{\partial f}{\partial x_i}\right|_{v_i}h_i=\sum_{i=1}^nD_if(v_i)h_i$$Since all $D_if$ are bounded in, Let $D_if$ is bounded by $M_i$ where $M_i>0$. Then take $M=\max\{M_i\mid 1\leq i\ \leq n\}$ Hence $\left|D_i(v_i)h_i\right|<M|h_i|$. Hence
	$$|f(a+h)-f(a)|=\left|\sum_{i=1}^nD_if(v_i)h_i\right|\leq \sum_{i=1}^n\left|D_if(v_i)\right|h_i\leq \sum_{i=1}^kM|h_i|=M\|h\|_1$$Now as $h\to 0$ we can say $\|h\|_1\to 0$. Hence $M\|h\|_1\to 0$ Therefore $|f(a+h)-f(a)|\to 0$. Hence $\forall\ \epsilon>0$ $\exists\ \delta>0$ such that $\|h\|_1<\delta$ whenever $|f(a+h)-f(a)|<\epsilon$. Therefore $f$ is continuous in $E$}
	
	
	%%%%%%%%%%%%%%%%%%%%%%%%%%%%%%%%%%%%%%%%%%%%%%%%%%%%%%%%%%%%%%%%%%%%%%%%%
	% Problem 3
	%%%%%%%%%%%%%%%%%%%%%%%%%%%%%%%%%%%%%%%%%%%%%%%%%%%%%%%%%%%%%%%%%%%%%%%%%
	
	\begin{problem}{Rudin Chapt. 9 Problem 8}{p3%Problem Name
		}
		%Problem		
		Suppose that $f$ is a differentiable real function in an open set $E \subset R^{n}$, and that $f$ has a local maximum at a point $x \in E$. Prove that $f^{\prime}(x)=0$.
	\end{problem}
	
	\sol{Let $u\in \bbR^n$ and $u$ is nonzero. Then consider the function $g(t)=x+tu$ which is a function from $\bbR$ to $\bbR^n$. Now $h=f\circ g$ is a $\bbR\to\bbR$ function and $h(t)=(f\circ g)(t)=f(x+ut)$. Hence $h$ has a maximum at $t=0$. Therefore $h'(0)=0$. Now by Chain Rule $h'(t)=f'(g(t))g'(t)=f'(g(t))u$. Hence $h'(0)=f'(g(0))u=f'(x)u$. Hence $f'(x)u=0$ since $u$ is nonzero $f'(x)=0$}
	
	
	%%%%%%%%%%%%%%%%%%%%%%%%%%%%%%%%%%%%%%%%%%%%%%%%%%%%%%%%%%%%%%%%%%%%%%%%%
	% Problem 4
	%%%%%%%%%%%%%%%%%%%%%%%%%%%%%%%%%%%%%%%%%%%%%%%%%%%%%%%%%%%%%%%%%%%%%%%%%
	
	\begin{problem}{Rudin Chapt. 9 Problem 10}{p4%Problem Name
		}
		%Problem
				If $f$ is a real function defined in a convex open set $E \subset R^{n}$, such that $\left(D_{1} f\right)(x)=0$ for every $x \in E$, prove that $f(x)$ depends only on $x_{2}, \ldots, x_{n}$.\parinn
				
				Show that the convexity of $E$ can be replaced by a weaker condition, but that some condition is required. For example, if $n=2$ and $E$ is shaped like a horseshoe, the statement may be false.
	\end{problem}
	
	\sol{To prove $f(x)$ depends only on $x_2,x_3,\cdots,x_n$ it is enough to show that $f(x,x_2,\cdots,x_n)=f(y,x_2,\cdots,x_n)$ where $(x,x_2,\cdots,x_n),(y,x_2,\cdots,x_n)\in E$. Hence for any $u\in (x,y)$ $(u,_2,\cdots,x_n)\in E$ as $E$ is convex. Now by Mean Value Theorem $f(x,x_2,\cdots,x_n)-f(y,x_2,\cdots,x_n)=(x-y)D_1f(z,x_2,\cdots,x_n)$ for some $z\in (x,y)$. Given that $D_if(x)=0$ Hence $f(x,x_2,\cdots,x_n)-f(y,x_2,\cdots,x_n)=0$. Therefore $f(x)$ depends only on $x_{2}, \ldots, x_{n}$.
		
		We need the property where for fixed $x_2,\cdots,x_n$ if $(x,x_2,\cdots,x_n)$ and $(y,x_2,\cdots,x_n)$ are in $E$ then $E$ must contain the line segment joining $(x,x_2,\cdots,x_n)$ and $(y,x_2,\cdots,x_n)$. Hence we can say that if $E$ intersects any line parallel to $X-$axis then it should be an interval on that line.
	
}
	
	
	%%%%%%%%%%%%%%%%%%%%%%%%%%%%%%%%%%%%%%%%%%%%%%%%%%%%%%%%%%%%%%%%%%%%%%%%%
	% Problem 5
	%%%%%%%%%%%%%%%%%%%%%%%%%%%%%%%%%%%%%%%%%%%%%%%%%%%%%%%%%%%%%%%%%%%%%%%%%
	
	\begin{problem}{Rudin Chapt. 9 Problem 13}{p5%Problem Name
		}
		%Problem		
		Suppose $f$ is a differentiable mapping of $R^{1}$ into $R^{3}$ such that $|f(t)|=1$ for every $t$. Prove that $f^{\prime}(t) \cdot f(t)=0$.\parinn
		
		Interpret this result geometrically.
	\end{problem}
	
	\sol{Let $f=\begin{bmatrix}
			f_1\\ f_2\\ f_3
		\end{bmatrix}$Given that $|f(t)|=1\implies f_1^2(t)+f_2^2(t)+f_3^2(t)=1$ for all $t\in \bbR$. Now $$0=\frac{d}{dt}|f(t)|^2=\frac{d}{dt}(f_1^(t)+f_2^2(t)+f_3^2(t))=2(f_1'(t)f_1(t)+f_2'(t)f_2(t)+f_3'(t)f_3(t))=2f'(t)\cdot f(t)$$Hence $f'(t)\cdot f(t)=0$ for all $t\in \bbR$.
	
	$f'(t)\cdot f(t)$ means the vector $f'(t)$ is perpendicular to $f(t)$. If $f(t)$ is the radius vector of a point on an unit sphere then $f'(t)$ is tangent vector on that point to the sphere
}
	
\end{document}
