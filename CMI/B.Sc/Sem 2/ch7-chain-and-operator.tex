\chapter{Chain Rule of Differentiation and Operator Norm}
\section{Operator Name}
$V,W$ are vector spaces. $\mcL(V,W)=$ Set of linear maps $V\to W$ is a vector space via $(A+B)(v)=A(v)+B(v)$ and $A(\lm v)=\lm A(v)$

If $V=\bbR^m$ and $\bbR^n$, $\dim (\bbR^m,\bbR^n)=mn$. We can identity $\mcL(\bbR^m,\bbR^n)$ with $n\times m$ matrices. $$\|A\|_{\mcL(\bbR^m,\bbR^n)}=\|A\|=\sup\limits_{\|u\|=1} \|A(u)\|$$This gives a norm because $\|A\|\geq 0$ and $\|A\|=0\implies A=0$ and  $\|\lm A\|=|\lm|\|A\|$. As  $(A+B)(u)=A(u)+B(u)$ we have $\|(A+B)(u)\|\leq \\A(u)\|+\|B(u)\|$ in $W$ and hemce $\|A+B\|\leq \|A\|+\|B\|$.
\qs{}{Why this is well defined ?}
\solve{The set $S=\{u\mid \|u\|=1\} $ is closed and bounded in $V$, therefore compact. $A$ being linear  is continuous. $\therefore A(S)$ is a compact subset of $W\implies A(S)$ is bounded.}

\textbf{Basic Properties:-}
\begin{enumerate}
	\item $\|Av\|\leq \|A\| \|v\|$ i.e. $\|Av\|_W\leq \|A\|_{\mcL}\|v\|_V$
	      \begin{myproof}
		      If $v=0$ then we are done. If $v\neq 0$, $u=\frac{v}{\|v\|}$ so $\|u\|=1$. Hence $$\|A\|\geq \|Au\|=\left\|A\left(\frac{v}{\|v\|}\right)\right\|=\frac{\|Av\|}{\|v\|}$$
	      \end{myproof}
	\item $\|A(v)\|\leq M\|v\|$ for all $v\implies \|A\|\leq M$ in fact $\inf\{M\mid \|A(v)\|\leq M\|v\|\ \forall\ v\}$
	      \begin{myproof}
		      Suppose $\|A(v)\|\leq M\|v\|$ $\forall\ v$. In particular $\forall\ v$ with $\|v\|=1$. So $\|A(v)\|\leq M$.

		      Rest exercise: If $L<\inf$ of the set  show $\exists \ u$ of norm=1 with $\|A(v)\| >L$.
	      \end{myproof}
	\item $U\xrightarrow{A}V\xrightarrow{B}W$ linear maps between finite dimensional vector spaces  then $\|BA\|\leq \|B\|\|A\|$
	      \begin{myproof}
		      Take $u$ with $\|u\|=1$. Then $$\|BA(u)\|\leq \|B\|\|A(u)\|\leq \|B\|\|A\|\|u\|=\|B\|\|A\|$$Now take $\sup$ over $u$.
	      \end{myproof}
	      \nt{$\underset{\mcL(U,V)\oplus\mcL(V,W)}{A,B}\underset{\to}{\mapsto}\underset{\mcL(U,W)}{BA}$ is continuous because  each slot of matrix of $BA$ is obtained by adding/multiplying entries of $A$ and $B$.}
\end{enumerate}
\qs{}{Show $A_n\to A$ in $\mcL(U,V)$, $B_n\to B$ in $\mcL(V,W)$ then $B_nA_n\to BA$ in $\mcL(U,W)$}

\section{Chain Rule}

\begin{Theorem}{}{}
	Let\begin{center}
		\begin{tikzcd}[ampersand replacement = \&]
			\&[-1cm]       \&\mathbb{R}^n     \&              \\[-0.7cm]
			\&[-1cm]          \& \usubset          \&              \\[-0.7cm]
			\mathbb{R}^m \supset \&[-1cm] U \arrow[r, "f"]     \& V \arrow[r, "g"] \& \mathbb{R}^k \\[-0.7cm]
			\&[-1cm] \uin                  \& \uin              \&              \\[-0.7cm]
			\&[-1cm] a \arrow[r, maps to] \& b                \&
		\end{tikzcd}
	\end{center} $f$ is differentiable at $a$. and $g$  is differentiable at $b=f(a)$. Then $g\circ f$ is differentiable at $a$  and $$(g\circ f)'(a)=\underbrace{g'(f(a))f'(a)}_{\substack{\text{Multiplication}\\ \text{of matrices}}}$$
\end{Theorem}
\begin{myproof}
	\subsubsection{1-Variable Case}
	$\frac{dz}{dx}=\frac{dz}{dy}\, \frac{dy}{dx}$\begin{align*}
		\frac{dz}{dx} =\lim_{\Delta x\to 0} \frac{\Delta z}{\Delta x} & =\lim_{\Delta x\to 0}\frac{\Delta z}{\Delta y}\, \frac{\Delta y}{\Delta x}                                                                  \\
		                                                              & =\lim_{\Delta x\to 0}\frac{\Delta z}{\Delta y}\lim_{\Delta x\to 0} \frac{\Delta y}{\Delta x}                                                \\
		                                                              & =\lim_{\Delta y\to 0}\frac{\Delta z}{\Delta y}\lim_{\Delta x\to 0} \frac{\Delta y}{\Delta x}\qquad [\text{As }\Delta x\to0,\ \Delta y\to 0]
	\end{align*}
	\subsubsection{Multi Variable Case}
	Note that $\bbR^m\xrightarrow{f'(a)}\bbR^n\xrightarrow{g'(f(a))}\bbR^k$

	If $T=f'(a)$ then $$\lim_{h\to 0}\frac{\|f(a+h)-f(a)-T(h)\|}{\|h\|}=0$$and $S=g'(f(a))=g'(b)$ then $$ \lim_{h\to 0}7\frac{\|fg(b+k)-g(b)-S(k)\|}{\|k\|}=0$$Let \begin{align*}
		\alpha(h) & = f(a+h)-f(a)-T(h) & \eps(h) & = \frac{\|\alpha(h)\|}{\|h\|}\to 0\text{ as }h\to 0  \\[3mm]
		\beta(k)  & = g(b+k)-g(b)-S(k) & \eta(k) & = \begin{cases*}
			                                             \dfrac{\|\beta(k)\|}{\|k\|}\to0  \text{ as }k\to 0 \\
			                                             0      \text{ when }                         k=0
		                                             \end{cases*}
	\end{align*}
	$\eta$ is continuous at $k=0$. Now note that $\eta:V-b\to \bbR$ because we ae always taking $b+k$ for $\eta$. We want to show that \[\lim_{h\to 0}\frac{\|g(f(a+h))-g(f(a))-ST(h)\|}{\|h\|}=0\iff \lim_{k\to 0}\frac{\|g(b+k)-g(b)-ST(h)\|}{\|h\|}=0\]where $f(a+h)=b+k\iff k=f(a+h)-f(a)$. We have taken a specific value of $k$ depending on $h$. So now $k$ is a function of $h$. Hence $T(h)=f(a+h)-f(a)-\alpha(h)=k-\alpha(h)$
	\begin{align*}
		  & g(b+k) -g(b) -ST(h)           \\
		= & g(b+k) -g(b) -S(k-\alpha(h))  \\
		= & g(b+k)-g(b)-S(k)+S(\alpha(h))
	\end{align*}Therefore\[ \frac{\|g(b+k)-g(b)-ST(h)\|}{\|h\|}\leq \frac{\|g(b+k)-g(b)-S(k)\|}{\|h\|}+\frac{\|S(\alpha(h))\|}{\|h\|}\] want to bound each of these separately
	\[\frac{\|S(\alpha(h))\|}{\|h\|}\leq \|S\|\frac{\|\alpha(h)\|}{\|h\|}\to 0\]Now how to bound the first term. In the first term $\frac{\|\beta(k)\|}{\|h\|}=\eta(k)\frac{\|k\|}{\|h\|}$. Now \begin{align*}
		         & k=T(h)+\alpha(h)                                                                                                                                                        \\
		\implies & \|k\|\leq \|T(h)\|+\|\alpha(h)\|                                                                                                                                        \\
		\implies & \frac{\|k\|}{\|h\|}\leq \frac{\|T(h)\|}{\|h\|}+\frac{\|\alpha(h)\|}{\|h\|}\leq \frac{\|T\|\|h\|}{\|h\|}+\frac{\|\alpha(h)\|}{\|h\|} = \|T\|+\frac{\|\alpha(h)\|}{\|h\|}
	\end{align*}Hence $$\frac{\|\beta(k)\|}{\|k\|}=\eta(k)\frac{\|k\|}{\|h\|}\leq \eta(k)\lt[\|T\|+\frac{\|\alpha(h)\|}{\|h\|}\rt]$$As $h\to 0$ $\|T\|+\frac{\|\alpha(h)\|}{\|h\|}\to \|T\|+0$ which is finite. And as $h\to 0$, $k\to 0\implies \eta(k)\to 0$ because $\eta$ is continuous at 0.
\end{myproof}

\section{Special Case of Chain Rule: When \texorpdfstring{$m=k=1$}{m=k=1}}
Open interval in $\bbR\xrightarrow{\gamma}\bbR^n\xrightarrow{g}\bbR$. $\gamma=$ parameterized curve in $\bbR^n$
$$(g\circ \gamma)'(t)=g'(\gamma(t))\cdot \gamma'(t)$$ ${\bbR\xrightarrow{\gamma}\bbR^n}$ maps $t\to\begin{bmatrix}\gamma_1(t)\\ \vdots\\ \gamma_n(t)	\end{bmatrix}$ hence $\gamma'(t)=\begin{bmatrix}
		\gamma_1'(t) \\ \vdots\\ \gamma_n'(t)
	\end{bmatrix}$

Now $g'(y)=\begin{bmatrix}
		\lt.\deld{g}{x_1}\rt|_y\cdots \lt.\deld{g}{x_n}\rt|_y
	\end{bmatrix}$. Hence \begin{align*}
	(g\circ \gamma)'(t) & = \begin{bmatrix}
		                        \lt.\deld{g}{x_1}\rt|_y\cdots \lt.\deld{g}{x_n}\rt|_y
	                        \end{bmatrix}\begin{bmatrix}
		                                     \gamma_1'(t) \\ \vdots\\ \gamma_n'(t)
	                                     \end{bmatrix}                                  \\
	                    & = \begin{bmatrix}
		                        \lt.\deld{g}{x_1}\rt|_y \\ \vdots \\ \lt.\deld{g}{x_n}\rt|_y
	                        \end{bmatrix}\cdot \begin{bmatrix}
		                                           \gamma_1'(t) \\ \vdots\\ \gamma_n'(t)
	                                           \end{bmatrix}\qquad[\text{Usual dot product of vectors in }\bbR^n] \\
\end{align*}
\pagebreak

Call $\begin{bmatrix}
		\lt.\del{g}{x_1}\rt|_y \\ \vdots \\ \lt.\del{g}{x_n}\rt|_y
	\end{bmatrix}$ $\nabla g(\gamma(t))=$ Gradient of $g$ at the point $\gamma(t)$. Hence $(g\circ\gamma)'(t)=\nabla g(\gamma(t))\cdot\gamma'(t)$
\qs{}{Fix $u,v\in \bbR^n$ and take parametrized curve $\gamma(t)=u+tv$. What does the above equation give at $t=0$ (for a given function $g$)}


