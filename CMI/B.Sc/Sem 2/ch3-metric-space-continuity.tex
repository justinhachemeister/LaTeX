\chapter{Continuity in Metric Space}
\section{Limit Point and Closure:-}

\dfn{Limit Point}{$S\subset X$ is a metric space. We say that $x\in X$ is a limit point of $S$ if $\exists$ a sequence $\{s_n\}$ with all $s_n\in S\setminus\{x\}$ such that $s_n\to x$ (each $s_n$ is different from $x$)}

\begin{Theorem}{}{limitpoint}	
	 $x$ is a limit point of $S$ $\iff$ every neighborhood of $x$ in $X$ contains a point of $S$ other than $S$.
\end{Theorem}
\begin{myproof}
	\subsubsection*{If Part:}
	Let $x$ be a limit point of $S$. Therefore take a sequence $\{s_n\}$ in $S\setminus \{x\}  $ with $s_n\to x$ . 
	
	To prove what we want it is enough to show that $B_r(x)\cap S$ contains a point other than $x$. As $s_n \to x$, $\exists $ $N$ s.t.  $\forall $ $n>N$ $d(x,s_n)<r$ i.e. $s_n\in B_r(x)$. In particular $s_n]in B_r(x)\cap (S\setminus \{x\})$
	\subsubsection*{Only If Part:}
	We need to produce  a sequence $\{s_n\} \in S\setminus \{x\}$ with $\lim s_n=x$. Take $s_n\in B_{\frac1n}(x)\cap (S\setminus \{x\})$ See that $\lim\limits_{n\to \infty}s_n=x$. This is essentially because $\frac1n\to 0$.
	
	Complete the rest of the proof.
	
\end{myproof}
\dfn{Closure}{Given a topological space $X$ and $S\subset X$, the closure of the set $S$ is $\overline{S}$ the smallest closed set containing $S$.}
\begin{Theorem}{}{closure}
	\begin{tabular}{rl}
		$\overline{S} $ & $=\text{Smallest closed set of } X\text{ containing }S $                                     = \setword{A}{A}                \\
		                & $ =S\cup (\text{limit points of }S)$                                                         =                \setword{B}{B} \\
		                & $=\{x\in X\mid x=\lim\limits_{n\to \infty} \text{ for some sequence }\{s_n\} \text{ in }S\}$ =                \setword{C}{C} \\
		                & $=\{x\in X\mid \text{Every neighborhood of }x\text{ intersects }S\}$                         =                \setword{D}{D}
	\end{tabular}
\end{Theorem}
\begin{myproof}
	\subsubsection*{$\bs{A\subset D}$}
		\begin{tabular}{l}\hspace{1.5cm}$A^c=\bigcup$ (All open set $V$ s.t. $V\cap S=\phi$) \\ \hspace{1.5cm}$D^c=\{x\in X\mid \exists \text{ open neighborhood of }x,B\text{ s.t. }B\cap S=\phi \}$\end{tabular}
		
		\setlength{\parindent}{0cm}Clearly for all $x\in D^c$, $x\in A^c$. Hence $D^c\subset A^c\implies A \subset D$\setlength{\parindent}{1cm}
		\subsubsection*{$\bs{D\subset B}$}
		Take $x\in D$. Suppose $x\notin S$. Now any neighborhood of $x$ intersects $S$ in a point hence it has to be a different point from $x$ since $x\notin S$. Therefore $x$ is a limit point of $S$. $D\subset B$
		\subsubsection*{$\bs{B\subset C}$}
		If $x\in S$ then take a sequence 
	
\end{myproof}

\qs{}{What does it mean to be smallest closed set containing the set $S$ here ?}
\sol $\bigcap$ All closed sets containing $S$ is automatically closed and hence the smallest closed set containing $S$.
\begin{myproof}
	For proof of \hyperref[th:closure]{Theorem \ref{th:closure}} notice \ref{A},\ref{B},\ref{C},\ref{D} all contains $S$ (obvious).
	\nt{We don't need to show \ref{B},\ref{C},\ref{D} are closed. We can also take the sets element wise and show each set is a subset of the other. This may simplify our way of proof. (exercise)}
	
	Now see $A$ and $D$ completely deal with topology. \ref{A} is about closed sets and \ref{D} is about open sets. So \ref{A} and \ref{D} close to each other. Now by the \ref{th:limitpoint} we have equivalence of \ref{C} and \ref{D}. So we can prove like this
	$$\ref{A}\iff\ref{D}\iff\ref{B}\ \&\ \ref{C}$$Left as exercise
\end{myproof}\nt{For these kind of proofs instead of looking for the most efficient way try to find a path that allows you to go from anywhere to anywhere}
\section{Continuity:-}
\dfn{Continuity}{$f:X\to Y$ function between metric spaces is continuous at $a\in X$ if $\forall$ $\eps>0$ $\exists$ $\delta>0$ s.t. \begin{align*}
	d(x,a)<\delta      & \implies d(f(a),f(x))<\eps      \\
	\Updownarrow\qquad & \qquad\qquad\qquad \Updownarrow \\
	x\in B_{\delta}(x) & \implies f(x)\in B_{\eps}(f(a))
\end{align*}}
\setlength{\parindent}{0cm}That means $f^{-1}$(Any ball around $f(a)$) $\supset$ Ball around $a$.\setlength{\parindent}{1cm}

So $f:X\to Y$ is continuous at all points $\iff$ $f^{-1}(\text{Any ball intersecting the range})\supset $ A ball
\nt{We can not say $f^{-1}(\text{Any ball})$ because because we need a ball that contains a point in the range
}
\begin{Theorem}{}{cont:invofopen}
	$f$ is continuous $\iff$ $f^{-1}(\text{Any open set in }Y)$ is open in $X$
\end{Theorem}
\begin{myproof}
	\subsubsection*{If Part:-}
	It is enough to show $f^{-1}(\text{Any ball})$ is open on $X$ because $f^{-1}$ preserves unions $f^{-1}\left(\bigcup\limits_{\alpha}V_{\alpha}\right)=\bigcup\limits_{\alpha}\left(f^{-1}(V_{\alpha})\right)$
	
	Let $B$ is any open set (as its conceptually simpler to take open set here instead of a ball) in $Y$. Let $a\in f^{-1}(B)$. Hence we can say $f(a)\in B$. Since $B$ is an open set we can say there is a ball $B_{\eps}(f(a))\subset B$. Since $f$ is continuous $\exists$ $\delta $ such that $f(x)\in B_{\eps}(f(a))$ whenever $x\in B_{\delta}(a)$. Now $f^{-1}(B)\supset f^{-1}\Big(B_{\eps}(f(a))\big)\supset B_{\delta}(a)$ Hence $f^{-1} (B)$ is open. 
	
	\subsubsection*{Only If Part:-}
	Lets prove continuity ar $a\in X$. We are given that $f^{-1}\Big(B_{\eps}(f(a))\Big)$ is open and obviously contains $a$. Therefore $f^{-1}\Big(B_{\eps}\big(f(a)\big)\Big)$ contains a ball around $a$. Take $\delta=$ Radius of the ball.
\end{myproof}

\qs{}{
		For a metric space $X$, show that $\overline{S} =\{x\in X\mid \lim\limits_{n\to\infty}s_n=x\}$ for some sequence $\{s_n\}$ in $S$.}

\qs{}{For a function $f:X\to Y$ between metric spaces, show that the followings are equivalent.\begin{enumerate}
			\item $f$ is continuous 
			\item $f^{-1}(\text{Open Set)}$ is open
			\item $f^{-1}(\text{Closed Set})$ is closed
			\item \label{itm:wrong} $f(\overline{S})=\overline{f(S)}$
			\item $x_n\to x \implies f(x_n)\to f(x)$
		\end{enumerate}
		One or more of the above are wrong so check if they are true and if not then find the true statement.}
\sol \ref{itm:wrong} is wrong. How to correct and rest is left as exercise
\qs{}{For $f:X\to Y$ any set map\begin{enumerate}[label=(\roman*)]
		\item $f^{-1}$ preserves unions, intersections, complements
		\item Is there any condition on $f$ under which $f$ possesses the property above ?
	\end{enumerate}}
\ex{Continuous Function}{\begin{enumerate}
		\item Any constant function.
		\item $X\xrightarrow{f} Y\xrightarrow{g}Z$ $f,g\text{ continuous} \implies g\cdot f$ is continuous
		\item Is $ S\subset X$ then $S\xrightarrow{\text{Inclusion}}X$ is continuous
		\item Projection $\underset{(x_1,x_2,\cdots,x_n)\mapsto x_i}{\bbR^n\to \bbR}$
		
		More generally for example $\underset{(x,y,z)\mapsto(x,x,y,y)}{\bbR^3\to \bbR^4}$
		\item Map from metric space to euclidean space.$$\begin{rcases*}
			X\to \bbR^n \\
			x\mapsto (f_1(x),f_2(x),\cdots,f_n(x))
		\end{rcases*}\substack{{f\text{ is continuous}}\\ {\iff} \\ {\text{each }f_i\text{is continuous}}}$$
	\item $\bbR\times \bbR\to\bbR: \ (x,y)\mapsto x\pm y,xy$ are continuous.
	
	We need to prove $x_n\to x$ and $y_n\to y$ in $\bbR\implies \begin{cases*}
		x_n\pm y_n\to x\pm y\\ x_ny_n \to xy
	\end{cases*}$

$\bbR\setminus\{0\}\to \bbR:\ x\mapsto \frac1x$ is continuous
\item sum and product of two continuous real valued function on $X$ are continuous
$$f,g:X\xrightarrow{f,g}\bbR\text{ continuous}\implies \underset{x}{X}\underset{\longmapsto}{\xrightarrow{f,g}}\underset{(f(x),g(x))}{\bbR\times\bbR}\xrightarrow{+}
\bbR$$ $$f:X\to\bbR\implies \frac1f:\underbrace{X\setminus f^{-1}(0)}_{\text{open set in }X}\to \bbR\text{ is continuous}$$ $\{0\}$ is closed in $\bbR$, so $f^{-1}(0)$ is closed in $X$ by  continuity of $f$\parinn

Therefore any polynomial in continuous real valued functions on $X$ is continuous.
\item \textbf{Special Case:} \begin{itemize}
	\item \parinn $\bbR^n\xrightarrow{T}\bbR^m$ linear map is continuous where $(x_1,x_2,\cdots,x_n)\longmapsto(a_{11}x_1+\cdots+a_{1n}x_n,a_{21}x_1+\cdots+a_{2n}x_n,\cdots,a_{m1}x_1+\cdots+a_{mn}x_n)$

Matrix of $T=\begin{bmatrix}
	a_{11} & a_{12} & \cdots & a_{1n} \\
	a_{21} & a_{22} & \cdots & a_{2n} \\
	\vdots & \vdots & \ddots & \vdots \\
	a_{m1} & a_{m2} & \cdots & a_{mn}
\end{bmatrix}$
\item  $\underset{A}{M_{n\times n}(\bbR)}\underset{\longmapsto}{\to} \underset{det(A)}{\bbR}$ is continuous

$\frac1{\text{det}}:GL_n(\bbR)\to\bbR$ \parinn

Here $M_{n\times n}$ is a vector space of dimension $n^2$ in which $GL_n(\bbR)=\{A\mid \det(A)\neq 0\}$ is an open set.

\item $\underset{A\longmapsto A^{-1}}{ GL_n(\bbR) \to GL_n(\bbR) }$ is continuous.

\end{itemize}
\item Any norm $(f)$ on $\bbR^n$ is uniformly continuous w.r.t usual topology on $\bbR^n$ i.e. $f:\bbR^n\to\bbR$ is continuous w.r.t usual norms ($\norm=p0$norm for $p=1,2,\infty$) on $\bbR^n(\norm)$ and $\bbR(|\cdot|)$
	\end{enumerate}
}
\begin{Theorem}{}{}
	Any norm $(f)$ on $\bbR^n$ is uniformly continuous w.r.t usual topology on $\bbR^n$ i.e. $\forall\ \eps>0$ $\forall\ x,y\in\bbR^n\ \exists\ \delta>0$ s.t. $\|x-y\|<\delta\implies |f(x)-f(y)<\eps$
\end{Theorem}
\begin{myproof}
	$$\begin{rcases*}
		f(x)\leq f(y)+f(x-y) \\ \qquad \qquad\qquad \qquad \parallel\\ f(y)\leq f(x)+f(y-x)
	\end{rcases*}|f(x)-f(y)|\leq f(x-y)$$Let $x=\sum x_ie_i$ and $y=\sum y_ie_i$ where $\{e_i\}$ is the standard basis of $\bbR^n$. 

$$f(x-y)=f\left(\sum (x_i-y_i)e_i\right)\leq \sum f\left((x_i-y_i)e_i\right)=|x_i-y_i|f(e_i)$$ Notice $\sum|x_i-y_i|=\|x-y\|_1$. Let $M=\max\{f(e_i)\}$ Then $$|f(x)-f(y)|\leq f(x-y)\leq M\|x-y\|_1$$

Thus $\|x-y\|<\frac{\eps}{M}\implies |f(x)-f(y)|<\eps$
\end{myproof}
