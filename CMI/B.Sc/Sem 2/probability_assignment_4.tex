\documentclass[a4paper, 11pt]{article}
\usepackage{comment} % enables the use of multi-line comments (\ifx \fi) 
\usepackage{fullpage} % changes the margin
\usepackage[a4paper, total={7in, 10in}]{geometry}
\usepackage{amsmath,mathtools}
\usepackage{amssymb,amsthm}  % assumes amsmath package installed
\usepackage{float}
\usepackage{xcolor}
\usepackage{mdframed}
\usepackage[shortlabels]{enumitem}
\usepackage{indentfirst}
\usepackage{hyperref}
\hypersetup{
	colorlinks=true,
	linkcolor=blue,
	filecolor=magenta,      
	urlcolor=blue!70!red,
	pdftitle={Probability Assignment 4}, %%%%%%%%%%%%%%%%   WRITE ASSIGNMENT PDF NAME  %%%%%%%%%%%%%%%%%%%%
}
\usepackage[most,many,breakable]{tcolorbox}



\definecolor{mytheorembg}{HTML}{F2F2F9}
\definecolor{mytheoremfr}{HTML}{00007B}


\tcbuselibrary{theorems,skins,hooks}
\newtcbtheorem{problem}{Problem}
{%
	enhanced,
	breakable,
	colback = mytheorembg,
	frame hidden,
	boxrule = 0sp,
	borderline west = {2pt}{0pt}{mytheoremfr},
	sharp corners,
	detach title,
	before upper = \tcbtitle\par\smallskip,
	coltitle = mytheoremfr,
	fonttitle = \bfseries\sffamily,
	description font = \mdseries,
	separator sign none,
	segmentation style={solid, mytheoremfr},
}
{p}

% To give references for any problem use like this
% suppose the problem number is p3 then 2 options either 
% \hyperref[p:p3]{<text you want to use to hyperlink> \ref{p:p3}}
%                  or directly 
%                   \ref{p:p3}



%---------------------------------------
% BlackBoard Math Fonts :-
%---------------------------------------

%Captital Letters
\newcommand{\bbA}{\mathbb{A}}	\newcommand{\bbB}{\mathbb{B}}
\newcommand{\bbC}{\mathbb{C}}	\newcommand{\bbD}{\mathbb{D}}
\newcommand{\bbE}{\mathbb{E}}	\newcommand{\bbF}{\mathbb{F}}
\newcommand{\bbG}{\mathbb{G}}	\newcommand{\bbH}{\mathbb{H}}
\newcommand{\bbI}{\mathbb{I}}	\newcommand{\bbJ}{\mathbb{J}}
\newcommand{\bbK}{\mathbb{K}}	\newcommand{\bbL}{\mathbb{L}}
\newcommand{\bbM}{\mathbb{M}}	\newcommand{\bbN}{\mathbb{N}}
\newcommand{\bbO}{\mathbb{O}}	\newcommand{\bbP}{\mathbb{P}}
\newcommand{\bbQ}{\mathbb{Q}}	\newcommand{\bbR}{\mathbb{R}}
\newcommand{\bbS}{\mathbb{S}}	\newcommand{\bbT}{\mathbb{T}}
\newcommand{\bbU}{\mathbb{U}}	\newcommand{\bbV}{\mathbb{V}}
\newcommand{\bbW}{\mathbb{W}}	\newcommand{\bbX}{\mathbb{X}}
\newcommand{\bbY}{\mathbb{Y}}	\newcommand{\bbZ}{\mathbb{Z}}

%---------------------------------------
% MathCal Fonts :-
%---------------------------------------

%Captital Letters
\newcommand{\mcA}{\mathcal{A}}	\newcommand{\mcB}{\mathcal{B}}
\newcommand{\mcC}{\mathcal{C}}	\newcommand{\mcD}{\mathcal{D}}
\newcommand{\mcE}{\mathcal{E}}	\newcommand{\mcF}{\mathcal{F}}
\newcommand{\mcG}{\mathcal{G}}	\newcommand{\mcH}{\mathcal{H}}
\newcommand{\mcI}{\mathcal{I}}	\newcommand{\mcJ}{\mathcal{J}}
\newcommand{\mcK}{\mathcal{K}}	\newcommand{\mcL}{\mathcal{L}}
\newcommand{\mcM}{\mathcal{M}}	\newcommand{\mcN}{\mathcal{N}}
\newcommand{\mcO}{\mathcal{O}}	\newcommand{\mcP}{\mathcal{P}}
\newcommand{\mcQ}{\mathcal{Q}}	\newcommand{\mcR}{\mathcal{R}}
\newcommand{\mcS}{\mathcal{S}}	\newcommand{\mcT}{\mathcal{T}}
\newcommand{\mcU}{\mathcal{U}}	\newcommand{\mcV}{\mathcal{V}}
\newcommand{\mcW}{\mathcal{W}}	\newcommand{\mcX}{\mathcal{X}}
\newcommand{\mcY}{\mathcal{Y}}	\newcommand{\mcZ}{\mathcal{Z}}



%---------------------------------------
% Bold Math Fonts :-
%---------------------------------------

%Captital Letters
\newcommand{\bmA}{\boldsymbol{A}}	\newcommand{\bmB}{\boldsymbol{B}}
\newcommand{\bmC}{\boldsymbol{C}}	\newcommand{\bmD}{\boldsymbol{D}}
\newcommand{\bmE}{\boldsymbol{E}}	\newcommand{\bmF}{\boldsymbol{F}}
\newcommand{\bmG}{\boldsymbol{G}}	\newcommand{\bmH}{\boldsymbol{H}}
\newcommand{\bmI}{\boldsymbol{I}}	\newcommand{\bmJ}{\boldsymbol{J}}
\newcommand{\bmK}{\boldsymbol{K}}	\newcommand{\bmL}{\boldsymbol{L}}
\newcommand{\bmM}{\boldsymbol{M}}	\newcommand{\bmN}{\boldsymbol{N}}
\newcommand{\bmO}{\boldsymbol{O}}	\newcommand{\bmP}{\boldsymbol{P}}
\newcommand{\bmQ}{\boldsymbol{Q}}	\newcommand{\bmR}{\boldsymbol{R}}
\newcommand{\bmS}{\boldsymbol{S}}	\newcommand{\bmT}{\boldsymbol{T}}
\newcommand{\bmU}{\boldsymbol{U}}	\newcommand{\bmV}{\boldsymbol{V}}
\newcommand{\bmW}{\boldsymbol{W}}	\newcommand{\bmX}{\boldsymbol{X}}
\newcommand{\bmY}{\boldsymbol{Y}}	\newcommand{\bmZ}{\boldsymbol{Z}}
%Small Letters
\newcommand{\bma}{\boldsymbol{a}}	\newcommand{\bmb}{\boldsymbol{b}}
\newcommand{\bmc}{\boldsymbol{c}}	\newcommand{\bmd}{\boldsymbol{d}}
\newcommand{\bme}{\boldsymbol{e}}	\newcommand{\bmf}{\boldsymbol{f}}
\newcommand{\bmg}{\boldsymbol{g}}	\newcommand{\bmh}{\boldsymbol{h}}
\newcommand{\bmi}{\boldsymbol{i}}	\newcommand{\bmj}{\boldsymbol{j}}
\newcommand{\bmk}{\boldsymbol{k}}	\newcommand{\bml}{\boldsymbol{l}}
\newcommand{\bmm}{\boldsymbol{m}}	\newcommand{\bmn}{\boldsymbol{n}}
\newcommand{\bmo}{\boldsymbol{o}}	\newcommand{\bmp}{\boldsymbol{p}}
\newcommand{\bmq}{\boldsymbol{q}}	\newcommand{\bmr}{\boldsymbol{r}}
\newcommand{\bms}{\boldsymbol{s}}	\newcommand{\bmt}{\boldsymbol{t}}
\newcommand{\bmu}{\boldsymbol{u}}	\newcommand{\bmv}{\boldsymbol{v}}
\newcommand{\bmw}{\boldsymbol{w}}	\newcommand{\bmx}{\boldsymbol{x}}
\newcommand{\bmy}{\boldsymbol{y}}	\newcommand{\bmz}{\boldsymbol{z}}

\newcommand{\eps}{\epsilon}
\newcommand{\veps}{\varepsilon}
\newcommand{\Qed}{\begin{flushright}\qed\end{flushright}}
\newcommand{\parinn}{\setlength{\parindent}{1cm}}
\newcommand{\parinf}{\setlength{\parindent}{0cm}}
\newcommand{\norm}{\|\cdot\|}
\newcommand{\inorm}{\norm_{\infty}}
\newcommand{\opensets}{\{V_{\alpha}\}_{\alpha\in I}}
\newcommand{\oset}{V_{\alpha}}
\newcommand{\opset}[1]{V_{\alpha_{#1}}}
\newcommand{\lub}{\text{lub}}
\newcommand{\del}[2]{\frac{\partial #1}{\partial #2}}
\newcommand{\Del}[3]{\frac{\partial^{#1} #2}{\partial^{#1} #3}}
\newcommand{\deld}[2]{\dfrac{\partial #1}{\partial #2}}
\newcommand{\Deld}[3]{\dfrac{\partial^{#1} #2}{\partial^{#1} #3}}
\newcommand{\lm}{\lambda}
\newcommand{\uin}{\mathbin{\rotatebox[origin=c]{90}{$\in$}}}
\newcommand{\ueq}{\mathbin{\rotatebox[origin=c]{90}{$=$}}}
\newcommand{\usubset}{\mathbin{\rotatebox[origin=c]{90}{$\subset$}}}
\newcommand{\lt}{\left}
\newcommand{\rt}{\right}
\newcommand{\bs}[1]{\boldsymbol{#1}}
\newcommand{\exs}{\exists}
\newcommand{\st}{\strut}
\newcommand{\dps}[1]{\displaystyle{#1}}

\newcommand{\sol}{\setlength{\parindent}{0cm}\textbf{\textit{Solution:}}\setlength{\parindent}{1cm} }
\newcommand{\solve}[1]{\setlength{\parindent}{0cm}\textbf{\textit{Solution: }}\setlength{\parindent}{1cm}#1 \Qed}
\DeclareRobustCommand{\rchi}{{\mathpalette\irchi\relax}}
\newcommand{\irchi}[2]{\raisebox{\depth}{$#1\chi$}}

\newcommand{\starx}{\textasteriskcentered}
\newcommand{\dnt}{\coloneqq}
\newcommand{\coef}{\bbF[\overline{x}]}
\newcommand{\per}{\text{per}}

\setlength{\parindent}{0pt}

%%%%%%%%%%%%%%%%%%%%%%%%%%%%%%%%%%%%%%%%%%%%%%%%%%%%%%%%%%%%%%%%%%%%%%%%%%%%%%%%%%%%%%%%%%%%%%%%%%%%%%%%%%%%%%%%%%%%%%%%%%%%%%%%%%%%%%%%

\begin{document}

%%%%%%%%%%%%%%%%%%%%%%%%%%%%%%%%%%%%%%%%%%%%%%%%%%%%%%%%%%%%%%%%%%%%%%%%%%%%%%%%%%%%%%%%%%%%%%%%%%%%%%%%%%%%%%%%%%%%%%%%%%%%%%%%%%%%%%%%

\textsf{\noindent \large\textbf{Soham Chatterjee} \hfill \textbf{Assignment - 4}\\
	Email: \href{sohamc@cmi.ac.in}{sohamc@cmi.ac.in} \hfill Roll: BMC202175\\
	\normalsize Course: Probability Theory \hfill Date: May 2, 2022 \\
	\noindent\rule{7in}{2.8pt}}

%%%%%%%%%%%%%%%%%%%%%%%%%%%%%%%%%%%%%%%%%%%%%%%%%%%%%%%%%%%%%%%%%%%%%%%%%%%%%%%%%%%%%%%%%%%%%%%%%%%%%%%%%%%%%%%%%%%%%%%%%%%%%%%%%%%%%%%%
% Problem 1
%%%%%%%%%%%%%%%%%%%%%%%%%%%%%%%%%%%%%%%%%%%%%%%%%%%%%%%%%%%%%%%%%%%%%%%%%%%%%%%%%%%%%%%%%%%%%%%%%%%%%%%%%%%%%%%%%%%%%%%%%%%%%%%%%%%%%%%%

\begin{problem}{%problem statemnet
}{p1% problem reference text
}
%Problem		
Let $X_{1}, \ldots, X_{n}$ be independent identically distributed (i.i.d.) random variables such that $E\left(X_{1}\right)$ and $E\left(1 / X_{1}\right)$ exist. Show that $E\left(S_{m} / S_{n}\right)=m / n$ where $S_{m}=X_{1}+\ldots+X_{m}$ where $m \leq n$.
\end{problem}

\solve{
	%Solution
	Since the random variables are independently identically distributed we have $$E\lt[\frac{S_m}{S_n}\rt]=\sum_{i=1}^nE\lt[\frac{X_i}{S_n}\rt]=mE\lt[\frac{X_1}{S_n}\rt]$$Now \begin{align*}
		     & E\lt[\frac{S_n}{S_n}\rt]=E[1]=1                                                                         \\
		\iff & \sum\limits_{i=1}^n E\lt[\frac{X_i}{S_n}\rt]=nE\lt[\frac{X_1}{S_n}\rt]=E\lt[\frac{X_1}{S_n}\rt]=\frac1n
	\end{align*}
	Hence we get $$E\lt[\frac{S_m}{S_n}\rt]=mE\lt[\frac{X_1}{S_n}\rt]=\frac{m}{n}$$}



%%%%%%%%%%%%%%%%%%%%%%%%%%%%%%%%%%%%%%%%%%%%%%%%%%%%%%%%%%%%%%%%%%%%%%%%%
% Problem 2
%%%%%%%%%%%%%%%%%%%%%%%%%%%%%%%%%%%%%%%%%%%%%%%%%%%%%%%%%%%%%%%%%%%%%%%%%

\begin{problem}{%problem statemnet
}{p2% problem reference text
}
%Problem		
\begin{enumerate}[label=(\alph*)]
	\item Let $X=N(0,1)$ and let $Y=e^{X}$. Find the density function of $Y$.
	\item Let $X$ be uniform on $(0,1)$. Let $g:(0,1) \rightarrow \mathbb{N}$ defined as $g(x)=k$ if $2^{-k} \leq x<2^{1-k}$. Find the mass function of $Y=g(X)$.
\end{enumerate}
\end{problem}

\solve{
	%Solution
	\begin{enumerate}[label=(\alph*)]
		\item Given that $X=N(0,1)$ and $Y=e^X$. Let $f\st_X(x)$ and $f\st_Y(y)$ are the density functions for random variable $X$ and $Y$ respectively. $F\st_Y(y)$ is the distribution function for random variable $Y$.

		      Hence $$f\st_X(x)=\frac1{\sqrt{2\pi}}e^{-\frac{x^2}2}$$Now $$F\st_Y(y)=P(Y\leq y)=P(e^X\leq y)=P(X\leq \ln y)=\int_{-\infty}^{\ln y}f\st_X(x)dx$$Differentiating both sides we get $$f\st_Y(y)=f\st_X(\ln y)\,\frac1y-\lim\limits_{x\to-\infty}f\st_X(x)$$
		      Since $\lim\limits_{x\to-\infty}f\st_X(x)=0$ we have $f\st_Y(y)=\frac1ye^{-\frac12(\ln y)^2}$\Qed
		\item $X$ is uniformly distributed on $(0,1)$. Let $f\st_X(x)$ is mass function of $X$. The mass function of $Y$ is $f\st_Y(y)$. Then $$f\st_X(x)=\begin{cases}
				      1 & 0< x <1         \\
				      0 & x\leq 0,x\geq 1
			      \end{cases}$$Now $Y$ is a discrete random variable. Hence $$f\st_Y(k)=P(Y=k)=P(2^{-k}\leq X<2^{-(k-1)})=\int_{2^{-k}}^{2^{-(k-1)}}f\st_X(x)dx=2^{-(k-1)}-2^{-k}=2^{-k}$$Therefore $f\st_Y(k)=2^{-k}$ where $k\in\bbN$
	\end{enumerate}

}


%%%%%%%%%%%%%%%%%%%%%%%%%%%%%%%%%%%%%%%%%%%%%%%%%%%%%%%%%%%%%%%%%%%%%%%%%
% Problem 3
%%%%%%%%%%%%%%%%%%%%%%%%%%%%%%%%%%%%%%%%%%%%%%%%%%%%%%%%%%%%%%%%%%%%%%%%%

\begin{problem}{%problem statemnet
}{p3% problem reference text
}
%Problem		
Let $X_{1}, \ldots, X_{n}$ be i.i.ds with finite mean and variance. Let $\bar{X}=\left(X_{1}+\ldots+X_{n}\right) / n$. Show that $\operatorname{cov}\left(\bar{X}, X_{n}-\bar{X}\right)=0$.
\end{problem}

\solve{
	%Solution
	Since $X_1,\dots,X_n$ are i.i.ds with finite mean and variance they all have same mean and variance. Let $E(X_i)=\mu$ and $Var(X_i)=\sigma^2$. Now \begin{align*}
		Cov\lt(\overline{X},X_n-\overline{X}\rt) & =E\lt[\overline{X}\lt(X_n-\overline{X}\rt)\rt]-E\lt[\overline{X}\rt]E\lt[X_n-\overline{X}\rt]                            \\
		                                         & =E\lt[X_n\overline{X}\rt]-E\lt[\overline{X}^2\rt]-E\lt[\overline{X}\rt]E\lt[X_n-\overline{X}\rt]                         \\
		                                         & =E\lt[X_n\overline{X}\rt]-Var\lt(\overline{X}\rt)-E\lt[\overline{X}\rt]^2-E\lt[\overline{X}\rt]E\lt[X_n-\overline{X}\rt]
	\end{align*}Now \begin{align*}
		E\lt[X_n\overline{X}\rt]  & =E\lt[X_n\frac{X_1+\cdots+X_n}{n}\rt]                                  \\
		                          & =\frac1n\lt(E[X_nX_1]+E[X_nX_2]+\cdots+E[X_nX_{n-1}]+E[X_n^2]\rt)      \\
		                          & =\frac1n\lt(E[X_n]E[X_1]+\cdots+E[X_n]E[X_{n-1}]+Var(X_n)+E[X_n]^2\rt) \\
		                          & =\frac1n((n-1)\mu^2+\sigma^2+\mu^2)=\mu^2+\frac{\sigma^2}n             \\
		                          &                                                                        \\
		E[\overline{X}]           & = E\lt[\frac{X_1+\cdots+X_n}{n}\rt]                                    \\
		                          & =\frac1nE[X_1+\cdots+X_n]                                              \\
		                          & = \frac1n(E[X_1]+\cdots+E[X_n])=\frac1n(n\mu)=\mu                      \\
		                          &                                                                        \\
		E\lt[X_n-\overline{X}\rt] & =E\lt[X_n\rt]-E\lt[\overline{X}\rt]=\mu - \mu=0                        \\
		                          &                                                                        \\
		Var(\overline{X})         & =Var\lt(\frac{X_1+\cdots+X_n}{n}\rt)                                   \\
		                          & = \frac1{n^2}Var\lt(X_1+\cdots+X_n\rt)                                 \\
		                          & = \frac1{n^2}\lt(Var(X_1)+\cdots+Var(X_n)\rt)                          \\
		                          & =\frac1{n^2}(n\sigma^2)=\frac{\sigma^2}{n}
	\end{align*}
	Hence putting these values we get\begin{align*}
		Cov\lt(\overline{X},X_n-\overline{X}\rt) & =E\lt[X_n\overline{X}\rt]-Var\lt(\overline{X}\rt)-E\lt[\overline{X}\rt]^2-E\lt[\overline{X}\rt]E\lt[X_n-\overline{X}\rt] \\
		                                         & =\mu^2+\frac{\sigma^2}{n}-\frac{\sigma^2}{n}-\mu^2-\mu\times 0=0
	\end{align*}
}


%%%%%%%%%%%%%%%%%%%%%%%%%%%%%%%%%%%%%%%%%%%%%%%%%%%%%%%%%%%%%%%%%%%%%%%%%
% Problem 4
%%%%%%%%%%%%%%%%%%%%%%%%%%%%%%%%%%%%%%%%%%%%%%%%%%%%%%%%%%%%%%%%%%%%%%%%%

\begin{problem}{%problem statemnet
}{p4% problem reference text
}
%Problem		
Let $X_{r}, r \geq 1$, be independent uniformly distributed on $[0,1]$. Let $0<x<1$ and let $N=\min \left\{k \mid X_{1}+\ldots+X_{k}>x\right\}$. Show that $P(N>n)=x^{n} / n !$. Find the mean and variance of $N$.
\end{problem}

\solve{
%Solution
Given that $N=\min \left\{k \mid X_{1}+\ldots+X_{k}>x\right\}$. Therefore $N$ is a discrete random variable. Hence $N>n\iff X_1+\cdots+X_n\leq x$. Let $f_n(x)=P(N>n)=P(X_{1}+\ldots+X_{n}\leq x)$. Let $X_n=a$ where $0\leq a_n\leq x$. Hence $$f_n(x)=\int_0^xf_{n-1}(x-a_n)da_n=\int_0^xf_{n-1}(a_n)da_n$$Similarly we can say let $X_{n-1}=a_{n-1}$ where $0\leq a_{n-1}\leq a_n$. Hence $$f_{n-1}(a_n)=\int_0^{a_n}f_{n-2}(a_n-a_{n-1})da_{n-1}=\int_0^{a_n}f_{n-2}(a_{n-1})da_{n-1}$$Therefore $$f_n(x)=\int_0^xf_{n-1}(a_n)da_n=\int_0^x\int_0^{a_n}f_{n-2}(a_{n-1})da_{n-1}da_n=\underbrace{\int_0^x\int_0^{a_n}\cdots\int_0^{a_2}}_{n\text{ times}}f_0(a_1)da_1\cdots da_{n-1}da_n$$Now $f_0(a_1)=P(N>0)=1$ since $N\geq 1$ always. Hence $$f_n(x)=\underbrace{\int_0^x\int_0^{a_n}\cdots\int_0^{a_2}}_{n\text{ times}}f_0(a_1)da_1\cdots da_{n-1}da_n=\underbrace{\int_0^x\int_0^{a_n}\cdots\int_0^{a_2}}_{n\text{ times}}da_1\cdots da_{n-1}da_n=\frac{x^n}{n!}$$

Now $P(N=n)=P(N>n-1)-P(N>n)=\frac{x^{n-1}}{(n-1)!}-\frac{x^n}{n!}$. Hence \begin{align*}
	E[N] & =\sum_{k=1}^{\infty}kP(N=k)                                                                                                       \\
	     & =\sum_{k=1}^{\infty}k\lt(\frac{x^{k-1}}{(k-1)!}-\frac{x^k}{k!}\rt)                                                                \\
	     & =\sum_{k=1}^{\infty}k\frac{x^{k-1}}{(k-1)!}-\sum_{k=1}^{\infty}k\frac{x^k}{k!}                                                    \\
	     & =\sum_{k=1}^{\infty}(k-1)\frac{x^{k-1}}{(k-1)!}+\sum_{k=1}^{\infty}\frac{x^{k-1}}{(k-1)!}-x\sum_{k=1}^{\infty}k\frac{x^{k-1}}{k!} \\
	     & =x\sum_{k=2}^{\infty}\frac{x^{k-2}}{(k-2)!}+\sum_{k=1}^{\infty}\frac{x^{k-1}}{(k-1)!}-x\sum_{k=1}^{\infty}\frac{x^{k-1}}{(k-1)!}  \\
	     & =xe^x+e^x-xe^x=e^x
\end{align*}
\begin{align*}
	E[N^2] & =\sum_{k=1}^{\infty}k^2P(N=k)                                                                                      \\
	       & =\sum_{k=1}^{\infty}k^2\lt(\frac{x^{k-1}}{(k-1)!}-\frac{x^k}{k!}\rt)                                               \\
	       & =\sum_{k=1}^{\infty}k^2\frac{x^{k-1}}{(k-1)!}-\sum_{k=1}^{\infty}k^2\frac{x^k}{k!}=G(x)-H(x)\qquad[\text{Suppose}]
\end{align*}

Now we will evaluate $H(x)$ first. \begin{align*}
	H(x) & = \sum_{k=1}^{\infty}k^2\frac{x^k}{k!}= \sum_{k=1}^{\infty}k\frac{x^k}{(k-1)!}           \\
	     & =\sum_{k=1}^{\infty}(k-1)\frac{x^{k}}{(k-1)!}+\sum_{k=1}^{\infty}\frac{x^{k}}{(k-1)!}    \\
	     & =x^2\sum_{k=2}^{\infty}\frac{x^{k-2}}{(k-2)!}+x\sum_{k=1}^{\infty}\frac{x^{k-1}}{(k-1)!} \\
	     & =x^2e^x+xe^x=x(x+1)e^x
\end{align*}\begin{align*}
	     & H(x)=\sum\limits_{k=1}^{\infty}k^2\dfrac{x^k}{k!}=\sum\limits_{k=1}^{\infty}(k-1)^2\dfrac{x^{k-1}}{(k-1)!}                                                      \\
	\iff & H(x)=\sum\limits_{k=1}^{\infty}k^2\dfrac{x^{k-1}}{(k-1)!}-2\sum\limits_{k=1}^{\infty}k\dfrac{x^{k-1}}{(k-1)!}+\sum\limits_{k=1}^{\infty}\dfrac{x^{k-1}}{(k-1)!} \\
	\iff & H(x)=G(x)-\frac2x\sum\limits_{k=1}^{\infty}k\dfrac{x^{k}}{(k-1)!}+\sum\limits_{k=1}^{\infty}\dfrac{x^{k-1}}{(k-1)!}                                             \\
	\iff & H(x)=G(x)-\frac2xH(x)+e^x                                                                                                                                       \\
	\iff & G(x)-H(x)=\frac2xH(x)-e^x                                                                                                                                       \\
	\iff & G(x)-H(x)=\frac{2}{x}x(x+1)e^x-e^x=(2x+1)e^x                                                                                                                    \\
	\iff & E[N^2]=(2x+1)e^x
\end{align*}Therefore$$Var(N)=E[N^2]-E[N]^2=(2x+1)e^x-e^{2x}$$
}


%%%%%%%%%%%%%%%%%%%%%%%%%%%%%%%%%%%%%%%%%%%%%%%%%%%%%%%%%%%%%%%%%%%%%%%%%
% Problem 5
%%%%%%%%%%%%%%%%%%%%%%%%%%%%%%%%%%%%%%%%%%%%%%%%%%%%%%%%%%%%%%%%%%%%%%%%%

\begin{problem}{%problem statemnet
}{p5% problem reference text
}
%Problem		
Find the density function of $Z=X+Y$ where $(X, Y)$ have joint density function $f(x, y)=(x+y) e^{-(x+y)} / 2, x, y \geq 0 .$
\end{problem}

\solve{
%Solution
Given that $Z=X+Y$. Let $f\st_Z(z)$ and $F\st_Z(z)$ are the mass function and distribution function of $Z$ respectively. Hence $$F\st_Z(z)=\int_{-\infty}^zf\st_Z(t)dt$$ and in another way we can say $$F\st_Z(z)=P(Z\leq z)=P(X\leq x,Y\leq z-x)=\int_{x=-\infty}^{\infty}\int_{y=-\infty}^{z-x}f(x,y)dydx$$Now let $U=X$ and $V=X+Y$. Then $X=U$ and $Y=V-U$. Hence the \textit{Jacobian Matrix}. $\mcJ$ becomes $$\mcJ=\begin{bmatrix}
		\deld{X}{U} & \deld{Y}{U}\\[3mm] \deld{X}{V} & \deld{Y}{V}\end{bmatrix}=\begin{bmatrix}
		\deld{U}{U} & \deld{(V-U)}{U}\\[3mm] \deld{U}{V} & \deld{(V-U)}{V}\end{bmatrix}=\begin{bmatrix}
		1 & -1 \\ 0 & 1
	\end{bmatrix}\implies |\mcJ|=1$$Hence
\begin{align*}
	F\st_Z(z) & =\int_{x=-\infty}^{\infty}\int_{y=-\infty}^{z-x}f(x,y)dydx       \\
	          & =\int_{u=-\infty}^{\infty}\int_{v=-\infty}^{z}f(u,v-u)|\mcJ|dvdu \\
	          & =\int_{-\infty}^{z}\lt[\int_{-\infty}^{\infty}f(u,v-u)du\rt]dv
\end{align*}Now differentiating both sides with respect to $z$ we get $f\st_Z(v)=\displaystyle{\int_{-\infty}^{\infty}}f(u,v-u)du$.

Now given that $$f(x,y)=\begin{cases}
		\frac12(x+y)e^{-(x+y)} & x,y\geq 0        \\
		0                      & \text{Otherwise}
	\end{cases}$$Now in case of $u,v-u\geq 0 \iff v\geq u\geq 0$. Hence when $v\geq u\geq 0$ $$f(u,v-u)=\frac12(u+v-u)e^{-(u+v-u)}=\frac12ve^{-v}$$Therefore $$f\st_Z(v)=\int_{-\infty}^{\infty}f(u,v-u)du=\int_{-\infty}^{\infty}\frac12ve^{-v}du=\frac12ve^{-v}\int_{0}^{v}du=\frac12v^2e^{-v}$$
}Therefore $$f\st_Z(v)=\begin{cases}
		\frac12v^2e^{-v} & v\geq 0 \\
		0                & v<0
	\end{cases}$$


%%%%%%%%%%%%%%%%%%%%%%%%%%%%%%%%%%%%%%%%%%%%%%%%%%%%%%%%%%%%%%%%%%%%%%%%%
% Problem 6
%%%%%%%%%%%%%%%%%%%%%%%%%%%%%%%%%%%%%%%%%%%%%%%%%%%%%%%%%%%%%%%%%%%%%%%%%

\begin{problem}{%problem statemnet
}{p5% problem reference text
}
%Problem
Suppose that $X, Y$ have bivariate normal distribution with zero expectations, variances $\sigma^{2}, \tau^{2}$ respectively, and, covariance $\rho>0$. Show that $E(Y \mid X)=\lambda X$ where $\lambda=\rho \tau / \sigma$.
\end{problem}

\solve{
%Solution
Given that $X,Y$ have bivariate normal distributions with zero expectations, variances $\sigma^2,\tau^2$ respectively and their covariance is $\rho>0$. Hence the join distribution function of $X,Y$ is $$f\st_{X,Y}(x,y)=\frac{1}{2\pi \sigma\tau
		\sqrt{1-\rho^2}}e^{-\frac1{2(1-\rho^2)}\lt(\frac{x^2}{\sigma^2}-\frac{2\rho xy}{\sigma\tau}+\frac{y^2}{\tau^2}\rt)}$$Hence we have $$f\st_{Y\mid X}(Y=y\mid X=x)=\frac{f\st_{X,Y}(x,y)}{f\st_X(x)}$$Now\begin{align*}
	f\st_X(x) & =\dps{\int_{-\infty}^{\infty}f\st_{X,Y}(x,y)dy}                                                                                                                                                                                      \\
	          & =\int_{-\infty}^{\infty}\frac{1}{2\pi \sigma\tau\sqrt{1-\rho^2}}e^{-\frac1{2(1-\rho^2)}\lt(\frac{x^2}{\sigma^2}-\frac{2\rho xy}{\sigma\tau}+\frac{y^2}{\tau^2}\rt)}dy                                                                \\
	          & =\frac{1}{2\pi \sigma\tau\sqrt{1-\rho^2}}e^{-\frac1{2(1-\rho^2)}\frac{x^2}{\sigma^2}(1-\rho^2)}\int_{-\infty}^{\infty}e^{-\frac1{2(1-\rho^2)} \lt( \frac{\rho^2 x^2}{\sigma^2}-\frac{2\rho xy}{\sigma\tau}+\frac{y^2}{\tau^2}\rt)}dy \\
	          & =\frac{1}{2\pi \sigma\tau\sqrt{1-\rho^2}}e^{-\frac12 \frac{x^2}{\sigma^2}}\int_{-\infty}^{\infty}e^{-\frac1{2(1-\rho^2)}\lt(\frac{\rho x}{\sigma}-\frac{y}{\tau}\rt)^2}dy                                                            \\
	          & =\frac{1}{\sqrt{2\pi}\sigma}e^{-\frac12 \frac{x^2}{\sigma^2}} \lt[ \frac{1}{\sqrt{2\pi(1-\rho^2)}\tau}\int_{-\infty}^{\infty}e^ {-\frac1{2(1-\rho^2)}\frac{(y\sigma -\rho \tau x^2)}{\sigma^2\tau^2}}dy   \rt]                       \\
	          & =\frac{1}{\sqrt{2\pi\sigma^2}}e^{-\frac12 \frac{x^2}{\sigma^2}}\lt[ \int_{-\infty}^{\infty}\frac{1}{\sqrt{2\pi\tau^2(1-\rho^2)}}e^{-\frac12 \frac{\lt(y-\frac{\rho \tau}{\sigma}x \rt)^2}{\tau^2(1-\rho^2)}  }dy \rt]                \\
	          & =\frac{1}{\sqrt{2\pi\sigma^2}}e^{-\frac12 \frac{x^2}{\sigma^2}}\lt[ \int_{-\infty}^{\infty}N\lt(\frac{\rho \tau}{\sigma}x,\tau^2(1-\rho^2)  \rt) \rt]                                                                                \\
	          & =\frac{1}{\sqrt{2\pi\sigma^2}}e^{-\frac12 \frac{x^2}{\sigma^2}}
\end{align*}

Hence \begin{align*}
	f\st_{Y\mid X}(Y=y\mid X=x) & =\frac{\frac{1}{2\pi \sigma\tau\sqrt{1-\rho^2}}e^{-\frac1{2(1-\rho^2)}\lt(\frac{x^2}{\sigma^2}-\frac{2\rho xy}{\sigma\tau}+\frac{y^2}{\tau^2}\rt)}}{\frac{1}{\sqrt{2\pi\sigma^2}}e^{-\frac12 \frac{x^2}{\sigma^2}}} \\
	                            & =\frac{1}{\sqrt{2\pi\tau^2(1-\rho^2)}}e^{-\frac1{2(1-\rho^2)}\lt(\frac{\rho^2x^2}{\sigma^2}-\frac{2\rho xy}{\sigma\tau}+\frac{y^2}{\tau^2}\rt)}                                                                     \\
	                            & = \frac{1}{\sqrt{2\pi\tau^2(1-\rho^2)}}e^{-\frac1{2(1-\rho^2)}\lt(\frac{\rho x}{\sigma}-\frac{y}{\tau}\rt)^2}
\end{align*}Now we will find the expectation of $Y\mid X=x$ \begin{align*}
	E[Y\mid X=x] & = \int_{-\infty}^{\infty}yf\st_{Y\mid X}(Y=y\mid X=x)dy =\int_{-\infty}^{\infty}y\frac{1}{\sqrt{2\pi\tau^2(1-\rho^2)}}e^{-\frac1{2(1-\rho^2)}\lt(\frac{\rho x}{\sigma}-\frac{y}{\tau}\rt)^2}dy
\end{align*}Now we substitute $$u=\frac{1}{\sqrt{1-\rho^2}}\lt(\frac{\rho x}{\sigma}-\frac{y}{\tau}\rt)\iff y =\tau\lt(\frac{\rho x}{\sigma}- u\sqrt{1-\rho^2} \rt)\implies dy = \tau\sqrt{1-\rho^2}du$$Substituting u in the integration we get \begin{align*}
	E[Y\mid X=x] & = \int_{-\infty}^{\infty}\tau\lt(\frac{\rho x}{\sigma}- u\sqrt{1-\rho^2} \rt)\frac{1}{\sqrt{2\pi\tau^2(1-\rho^2)}}e^{-\frac12u^2}\tau\sqrt{1-\rho^2}du                                                                            \\
	             & = \int_{-\infty}^{\infty}\tau\lt(\frac{\rho x}{\sigma}- u\sqrt{1-\rho^2} \rt)\frac{1}{\sqrt{2\pi}}{e^{-\frac12u^2}}du                                                                                                             \\
	             & =\frac{\tau\rho x}{\sigma}\int_{-\infty}^{\infty}\underbrace{\frac{1}{\sqrt{2\pi}}e^{-\frac12u^2}}_{N(0,1)}du- \tau\sqrt{1-\rho^2}\frac{1}{\sqrt{2\pi}}\int_{-\infty}^{\infty}\underbrace{ue^{-\frac12u^2}}_{\substack{\text{Odd} \\ \text{Function}}}du= \frac{\tau\rho }{\sigma}x
\end{align*} Hence $E[Y\mid X=x]=\lm x$ where $\lm=\frac{\tau\rho }{\sigma}$
}

\end{document}
