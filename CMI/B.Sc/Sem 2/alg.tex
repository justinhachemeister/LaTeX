\documentclass{article}
\usepackage{mathtools,amsmath,amsfonts,fullpage,enumitem,authblk}

%---------------------------------------
% BlackBoard Math Fonts :-
%---------------------------------------

%Captital Letters
\newcommand{\bbA}{\mathbb{A}}	\newcommand{\bbB}{\mathbb{B}}
\newcommand{\bbC}{\mathbb{C}}	\newcommand{\bbD}{\mathbb{D}}
\newcommand{\bbE}{\mathbb{E}}	\newcommand{\bbF}{\mathbb{F}}
\newcommand{\bbG}{\mathbb{G}}	\newcommand{\bbH}{\mathbb{H}}
\newcommand{\bbI}{\mathbb{I}}	\newcommand{\bbJ}{\mathbb{J}}
\newcommand{\bbK}{\mathbb{K}}	\newcommand{\bbL}{\mathbb{L}}
\newcommand{\bbM}{\mathbb{M}}	\newcommand{\bbN}{\mathbb{N}}
\newcommand{\bbO}{\mathbb{O}}	\newcommand{\bbP}{\mathbb{P}}
\newcommand{\bbQ}{\mathbb{Q}}	\newcommand{\bbR}{\mathbb{R}}
\newcommand{\bbS}{\mathbb{S}}	\newcommand{\bbT}{\mathbb{T}}
\newcommand{\bbU}{\mathbb{U}}	\newcommand{\bbV}{\mathbb{V}}
\newcommand{\bbW}{\mathbb{W}}	\newcommand{\bbX}{\mathbb{X}}
\newcommand{\bbY}{\mathbb{Y}}	\newcommand{\bbZ}{\mathbb{Z}}

%---------------------------------------
% MathCal Fonts :-
%---------------------------------------

%Captital Letters
\newcommand{\mcA}{\mathcal{A}}	\newcommand{\mcB}{\mathcal{B}}
\newcommand{\mcC}{\mathcal{C}}	\newcommand{\mcD}{\mathcal{D}}
\newcommand{\mcE}{\mathcal{E}}	\newcommand{\mcF}{\mathcal{F}}
\newcommand{\mcG}{\mathcal{G}}	\newcommand{\mcH}{\mathcal{H}}
\newcommand{\mcI}{\mathcal{I}}	\newcommand{\mcJ}{\mathcal{J}}
\newcommand{\mcK}{\mathcal{K}}	\newcommand{\mcL}{\mathcal{L}}
\newcommand{\mcM}{\mathcal{M}}	\newcommand{\mcN}{\mathcal{N}}
\newcommand{\mcO}{\mathcal{O}}	\newcommand{\mcP}{\mathcal{P}}
\newcommand{\mcQ}{\mathcal{Q}}	\newcommand{\mcR}{\mathcal{R}}
\newcommand{\mcS}{\mathcal{S}}	\newcommand{\mcT}{\mathcal{T}}
\newcommand{\mcU}{\mathcal{U}}	\newcommand{\mcV}{\mathcal{V}}
\newcommand{\mcW}{\mathcal{W}}	\newcommand{\mcX}{\mathcal{X}}
\newcommand{\mcY}{\mathcal{Y}}	\newcommand{\mcZ}{\mathcal{Z}}



%---------------------------------------
% Bold Math Fonts :-
%---------------------------------------

%Captital Letters
\newcommand{\bmA}{\boldsymbol{A}}	\newcommand{\bmB}{\boldsymbol{B}}
\newcommand{\bmC}{\boldsymbol{C}}	\newcommand{\bmD}{\boldsymbol{D}}
\newcommand{\bmE}{\boldsymbol{E}}	\newcommand{\bmF}{\boldsymbol{F}}
\newcommand{\bmG}{\boldsymbol{G}}	\newcommand{\bmH}{\boldsymbol{H}}
\newcommand{\bmI}{\boldsymbol{I}}	\newcommand{\bmJ}{\boldsymbol{J}}
\newcommand{\bmK}{\boldsymbol{K}}	\newcommand{\bmL}{\boldsymbol{L}}
\newcommand{\bmM}{\boldsymbol{M}}	\newcommand{\bmN}{\boldsymbol{N}}
\newcommand{\bmO}{\boldsymbol{O}}	\newcommand{\bmP}{\boldsymbol{P}}
\newcommand{\bmQ}{\boldsymbol{Q}}	\newcommand{\bmR}{\boldsymbol{R}}
\newcommand{\bmS}{\boldsymbol{S}}	\newcommand{\bmT}{\boldsymbol{T}}
\newcommand{\bmU}{\boldsymbol{U}}	\newcommand{\bmV}{\boldsymbol{V}}
\newcommand{\bmW}{\boldsymbol{W}}	\newcommand{\bmX}{\boldsymbol{X}}
\newcommand{\bmY}{\boldsymbol{Y}}	\newcommand{\bmZ}{\boldsymbol{Z}}
%Small Letters
\newcommand{\bma}{\boldsymbol{a}}	\newcommand{\bmb}{\boldsymbol{b}}
\newcommand{\bmc}{\boldsymbol{c}}	\newcommand{\bmd}{\boldsymbol{d}}
\newcommand{\bme}{\boldsymbol{e}}	\newcommand{\bmf}{\boldsymbol{f}}
\newcommand{\bmg}{\boldsymbol{g}}	\newcommand{\bmh}{\boldsymbol{h}}
\newcommand{\bmi}{\boldsymbol{i}}	\newcommand{\bmj}{\boldsymbol{j}}
\newcommand{\bmk}{\boldsymbol{k}}	\newcommand{\bml}{\boldsymbol{l}}
\newcommand{\bmm}{\boldsymbol{m}}	\newcommand{\bmn}{\boldsymbol{n}}
\newcommand{\bmo}{\boldsymbol{o}}	\newcommand{\bmp}{\boldsymbol{p}}
\newcommand{\bmq}{\boldsymbol{q}}	\newcommand{\bmr}{\boldsymbol{r}}
\newcommand{\bms}{\boldsymbol{s}}	\newcommand{\bmt}{\boldsymbol{t}}
\newcommand{\bmu}{\boldsymbol{u}}	\newcommand{\bmv}{\boldsymbol{v}}
\newcommand{\bmw}{\boldsymbol{w}}	\newcommand{\bmx}{\boldsymbol{x}}
\newcommand{\bmy}{\boldsymbol{y}}	\newcommand{\bmz}{\boldsymbol{z}}
\title{\LARGE{\textsc{Algebra 2 Week 6 -- Artin 5.1, 5.2, 5.6 Chapter 8}}}
\author{\LARGE{\textsf{\textbf{Name:} Soham Chatterjee\hfill \textbf{Roll:} BMC202175}\\\hrule}}
\date{}

\begin{document}
	
\maketitle
\thispagestyle{empty}
	\begin{enumerate}[label=5.\arabic*.]
		\item \begin{enumerate}
			\item Since its an Euclidean Space the form is positive definite and symmetric. Hence $\forall\ \lambda\in\bbR$,  $\langle v-\lambda w,v-\lambda w\rangle\geq0$. Hence \begin{align*}
				\langle v-\lambda w,v-\lambda w\rangle & = \langle v,v\rangle - \lambda\langle v,w\rangle +\lambda^2\langle w\rangle
			\end{align*}Now $\langle v,v\rangle - \lambda\langle v,w\rangle +2\lambda^2\langle w,w\rangle\geq 0$ is a quadratic  equation of $\lambda$ where it gives positive value when $v\neq \lambda w$ and 0 when $v=\lambda w$. Therefore the discriminant must be negative or zero. Hence \begin{align*}
				 & 4\langle v,w\rangle ^2-4\langle v,v\rangle\langle w,w\rangle\leq 0\\
			\iff & \langle v,w\rangle^2\leq \langle v,v\rangle\langle w ,w\rangle\\
			\iff & |\langle v,w\rangle| \leq | v||w|\qquad [\text{Proved}]
		\end{align*}
		\item We have for any $v\in V$ $|v|^2=\langle v,v\rangle$. Now \begin{align*}
			|v+w|^2 + |v-w|^2 & = \langle v+w,v+w\rangle +\langle v-w,v-w\rangle \\
			& = [\langle v,v\rangle +2\langle v,w\rangle  +\langle w,w\rangle] +[\langle v,v\rangle -2\langle v,w\rangle +\langle w,w\rangle]\\
			& = 2\langle v,v\rangle +2\langle w,w\rangle \\
			& =2|v|^2+2|w|^2\qquad [\text{Proved}]
		\end{align*}
	\item Given that $|v|=|w|\implies |v|^2=|w|^2\implies \langle v,v\rangle=\langle w,w\rangle$. To prove $(v+w)\perp (v-w)$ if we show that $\langle v+w,v-w\rangle=0$ we are done. Now \begin{align*}
		\langle v+w,v-w\rangle & = \langle v,v\rangle +\langle v,w\rangle - \langle w,v\rangle -\langle w,w\rangle \\
		& = \langle v,v\rangle -\langle w,w\rangle\\
		& = 0 \qquad [\text{Proved}]
	\end{align*}
		\end{enumerate}
	\item $W^{\perp}=\{v\mid v\in V, \ \langle v,w\rangle =0\ \forall\ w\in W\}$. Hence $W^{\perp\perp}=\{w'\mid w'\in V, \ \langle w',v\rangle =0\ \forall\ v\in W^{\perp}\}\}$. Let $w\in W$ then $\langle w,v\rangle =0\ \forall v\in W^{\perp}$. Hence $w\in W^{\perp\perp}$. Therefore $W\subseteq W^{\perp\perp}$. Now we know that $V=W\oplus W=W^{\perp}\oplus W^{\perp\perp}$ Hence $\dim W^{\perp}=\dim W^{\perp\perp}$. We can say then $W=W^{\perp\perp}$ [Proved]
	\addtocounter{enumi}{3}
	\item Let $\lambda \in \bbC$ be an eigen value of the unitary matrix $A$. Hence $AX=\lambda X$ for some column vector $X$. Now $$(PX)^*(PX)=X^*P^*PX=X^*X\qquad (PX)^*(PX)=(\lambda X)^*(\lambda X)=\overline{\lambda}X^*\lambda X=\overline{\lambda}\lambda X^*X $$Hence $X^*X=\overline{\lambda}\lambda X^*X\implies \overline{\lambda}\lambda =1\implies |\lambda|^2=1$. Hence the complex numbers with unit modulous 1 will appear as eigenvalues of a unitary matrix.
	\end{enumerate}
\end{document} 