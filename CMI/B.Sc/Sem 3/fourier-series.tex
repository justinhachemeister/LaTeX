\section{Fourier Series}
\defn[\textbf{Fourier Coefficient} \cite{rudin}]{
If $f$ is an integrable function given on an interval $[a, b]$ of length $L$ (i.e. $b-a=L$) then $n-$th fourier coefficient of $f$ is $$c_n=\hat{f}(n)=\frac1L\int_a^bf(x)e^{-\frac{2\pi inx}{l}}dx$$}

\defn[\textbf{Fourier Series} \cite{rudin}]{
If $f$ is an integrable function given on an interval $[a, b]$ of length $L$ (i.e. $b-a=L$) then if $n-$th fourier coefficient of $f$ is $\hat{f}_n$ then fourier series of $f$ is $$\sum_{-\infty}^{\infty}\hat{f}(n)e^{\frac{2\pi inx}{L}}$$and we denote it by $$f(x) \sim \sum_{-\infty}^{\infty}\hat{f}(n)e^{\frac{2\pi inx}{L}}$$ }
\defn[\textbf{Orthogonal System of Functions} \cite{rudin}]{Let $\{\phi_n\} $ $n\in \bbN$ be a sequence of complex functions on $[a,b]$ such that $$\int_a^b \phi_n(x)\overline{\phi_m(x)}dx = 0\qquad m\neq n$$Then $\{\phi_n\}$ is said to be an orthogonal system of functions on $[a,b]$. If, in addition, $$\int_a^b |\phi_n(x)|^2dx=1$$ for all $n\in \bbN$, $\{\phi_n\}$ is  said to be orthonormal. 
}
\begin{simplebox}
1, $\cos\frac{\pi x}{L}$, $\sin \frac{\pi x}{L}$, $\cos \frac{2\pi x}{L},\dots$ are orthogonal system of functions with common period = $\frac{2\pi}{\pi}L=2L$
\begin{align*}
    a)& \int _{-L}^{+L}\cos \frac{m\pi x}{L}\cos \frac{n\pi x}{L}dx = \int _{-a}^{a+2L}\cos \frac{m\pi x}{L}\cos \frac{n\pi x}{L}dx =\begin{cases} 0 &m\neq n \\ L & m=n\neq 0\end{cases}\\
    a)& \int _{-L}^{+L}\sin \frac{m\pi x}{L}\sin \frac{n\pi x}{L}dx = \int _{-a}^{a+2L}\sin \frac{m\pi x}{L}\sin \frac{n\pi x}{L}dx =\begin{cases} 0 &m\neq n \\ L & m=n\neq 0\end{cases}\\
    a)& \int _{-L}^{+L}\sin \frac{m\pi x}{L}\cos \frac{n\pi x}{L}dx = \int _{-a}^{a+2L}\sin \frac{m\pi x}{L}\cos \frac{n\pi x}{L}dx =0
\end{align*}
The functions $(2\pi)^{-\frac12 e^{inx}},$ $\frac{1}{\sqrt{2\pi}},$ $\frac{\cos x}{\sqrt{2\pi}},$  $\frac{\sin x}{\sqrt{2\pi}},$ $\frac{\cos 2x}{\sqrt{2\pi}},$ $\frac{\sin 2x}{\sqrt{2\pi}},\dots$ form orthonormal system of function of $[-\pi, \pi]$
\end{simplebox}
\begin{theorem}[\cite{rudin}]
If $\{\pi_n\}$ is orthonormal on $[a,b]$  then $$\hat{f}(n)=\int_a^bf(x)\overline{\phi_n(x)} dx\qquad \text{and} \qquad f(x)\sim \sum _1^{\infty}\hat{f}(n)\phi_n(x)$$
\end{theorem}
\begin{theorem}[\cite{rudin}]
Let $\left\{\phi_n\right\}$ be orthonormal on $[a, b]$. Let
$$
s_n(x)=\sum_{m=1}^n c_m \phi_m(x)
$$
be the nth partial sum of the Fourier series of $f$, and suppose
$$
t_n(x)=\sum_{m=1}^n \gamma_m \phi_m(x) .
$$
Then
$$
\int_a^b\left|f-s_n\right|^2 d x \leq \int_a^b\left|f-t_n\right|^2 d x,
$$
and equality holds if and only if
$$
\gamma_m=c_m \quad(m=1, \ldots, n) .
$$
That is to say, among all functions $t_n, s_n$ gives the best possible mean square approximation to $f$.
\end{theorem}
\begin{theorem}[\cite{rudin}]
If $\left\{\phi_n\right\}$ is orthonormal on $[a, b]$, and if
$$
f(x) \sim \sum_{n=1}^{\infty} c_n \phi_n(x),
$$
then
$$
\sum_{n=1}^{\infty}\left|c_n\right|^2 \leq \int_a^b|f(x)|^2 d x
$$
In particular,
$$
\lim _{n \rightarrow \infty} c_n=0
$$
\end{theorem}
\begin{theorem}[\cite{rudin}]
If, for some $x$, there are constants $\delta>0$ and $M<\infty$ such that
$$
|f(x+t)-f(x)| \leq M|t|
$$
for all $t \in(-\delta, \delta)$, then
$$
\lim _{N \rightarrow \infty} s_N(f ; x)=f(x) .
$$
\end{theorem}
\begin{theorem}[\cite{rudin}]
If $f$ is continuous (with period $2 \pi$ ) and if $\varepsilon>0$, then there is a trigonometric polynomial $P$ such that
$$
|P(x)-f(x)|<\varepsilon
$$
for all real $x$.
\end{theorem}

\begin{theorem}[\textbf{Parseval's Theorem} \cite{rudin}] Suppose $f$ and $g$ are Riemann-integrable functions with period $2 \pi$, and
$$
f(x) \sim \sum_{-\infty}^{\infty} c_n e^{i n x}, \quad g(x) \sim \sum_{-\infty}^{\infty} \gamma_n e^{i n x} .
$$
Then

\begin{align*}
&\lim _{N \rightarrow \infty} \frac{1}{2 \pi} \int_{-\pi}^\pi\left|f(x)-s_N(f ; x)\right|^2 d x =0, \\[2mm]
&\frac{1}{2 \pi} \int_{-\pi}^\pi f(x) \overline{g(x)} d x =\sum_{-\infty}^{\infty} c_n \bar{\gamma}_n, \\[2mm]
&\frac{1}{2 \pi} \int_{-\pi}^\pi|f(x)|^2 d x =\sum_{-\infty}^{\infty}\left|c_n\right|^2 .
\end{align*}

\end{theorem}
\defn[\textbf{Convolutions} \cite{stein}]{Given $2\pi$-periodic integrable functions $f$ and $g$  on $\bbR$ then their convolution $f\star g$ on $[-\pi, \pi$ is $$(f\star g)(x)=\frac1{2\pi} \int_{-\pi}^{\pi} f(y)g(x-y)dy$$}
\defn[\textbf{Dirichilet Kernel} \cite{rudin}]{
$$D_N(x)=\sum_{-N}^{N}e^{inx}$$
}
\begin{theorem}[\cite{rudin}]
$$D_N(x)=\frac{\sin\lt(N+\frac12\rt)x}{\sin \frac{x}{2}}$$
\end{theorem}

\begin{theorem}[\cite{stein}]
$$S_N(f)(x)=(f\star D_N)(x)$$
\end{theorem}


\begin{theorem}[\cite{stein}]
Suppose that $f,$ $g$, $h$ are $2\pi-$periodic integrable functions then \begin{enumerate}[label=(\roman*)]
    \item $f\star (g + h)=(f\star g) + (f\star h)$
    \item $(cf\star g)=c(f\star g)=(f\star cg)$ for any $c\in \bbC$
    \item $(f\star g)=(g\star f)$
    \item $(f\star (g\star h))= ((f\star g)\star h)$
    \item $(f\star g)$ is continuous 
    \item $\hat{(f \star g)} (n)=\hat{f}(n)\hat{g}(n)$
\end{enumerate}
\end{theorem}
\begin{theorem}[\cite{stein}]
Suppose $f$ is integrable on the circle and bounded by $B$. Then there exists a sequence $\left\{f_k\right\}_{k=1}^{\infty}$ of continuous functions on the circle so that
$$
\sup _{x \in[-\pi, \pi]}\left|f_k(x)\right| \leq B \quad \text { for all } k=1,2, \ldots
$$
and
$$
\int_{-\pi}^\pi\left|f(x)-f_k(x)\right| d x \rightarrow 0 \quad \text { as } k \rightarrow \infty
$$
\end{theorem}
\defn[\textbf{Good Kernel} \cite{stein}]{
A family of kernels $\left\{K_n(x)\right\}_{n=1}^{\infty}$ on the circle is said to be a family of good kernels if it satisfies the following properties:
\begin{enumerate}[label=(\alph*)]
\item For all $n \geq 1$,
$$
\frac{1}{2 \pi} \int_{-\pi}^\pi K_n(x) d x=1 .
$$
\item  There exists $M>0$ such that for all $n \geq 1$,
$$
\int_{-\pi}^\pi\left|K_n(x)\right| d x \leq M .
$$
\item For every $\delta>0$,
$$
\int\limits_{\delta \leq|x| \leq \pi}\left|K_n(x)\right| d x \rightarrow 0, \quad \text { as } n \rightarrow \infty$$
\end{enumerate}
}
\begin{theorem}[\cite{stein}]\label{apptoiden}
Let $\left\{K_n\right\}_{n=1}^{\infty}$ be a family of good kernels, and $f$ an integrable function on the circle. Then
$$
\lim _{n \rightarrow \infty}\left(f * K_n\right)(x)=f(x)
$$
whenever $f$ is continuous at $x$. If $f$ is continuous everywhere, then the above limit is uniform.
\end{theorem}
\defn[\textbf{Approximation to the Identity} \cite{stein}]{
Because of \href{\ref{apptoiden}}{Theorem \ref{apptoiden}}, the family $\{K_n\}$  is  sometimes referred to as an approximation to the identity.
}
\begin{theorem}
Prove that Dirichlet Kernel is not a good kernel
\end{theorem}
\defn[\textbf{Cesaro Summability and Cesaro Mean} \cite{cesarosum}]{
Let $(a_n)$ be a sequence of real numbers. $\sum\limits_{n=0}^{\infty}a_n$ be a series. Let $$s_n=\sum_{k=0}^{n}a_k$$Then the sequence $(a_n)$ is Cesaro Summable with Cesaro Sum $s\in \mathbb{R}$ if $$\lim_{n\to\infty}\sigma_n=\lim_{n\to\infty}\frac{1}{n+1}\sum_{k=0}^n s_n=s$$The term $\sigma_n$ is called  Cesaro Mean or the $n-$th Cesaro Sum of the series $\sum\limits_{n=0}^{\infty}a_n$
}
\defn[\textbf{Fej\'{e}r Kernel} \cite{stein}]{$F_N$ is the $N-$th fej\'{e}r kernel where $$F_N(x)=\frac{1}{N}\sum_{n=0}^ND_n(x)$$}
\begin{theorem}[\cite{stein}]
Prove that 
$$F_N(x)=\frac1N\frac{\sin^2(Nx/2)}{\sin^2(x/2)}$$
and fej\'{e}r kernel is good kernel
\end{theorem}
\begin{theorem}[\cite{stein}]
If $f$ is integrable on the circle, then the Fourier series of $f$ is Cesaro Summable to $f$ at every point of continuity of $f$.

Moreover, if $f$ is continuous on the circle, then the Fourier series of $f$ is uniformly Cesaro Summable to $f$.
\end{theorem}
\begin{theorem}[\cite{stein}]
If $f$ is integrable on the circle and $\hat{f}(n) = 0$ for all $n$, then $f = 0$ at all points of continuity of $f$.
\end{theorem}
\begin{theorem}[\cite{stein}]
Continuous functions on the circle can be uniformly approximated by trigonometric polynomials.
\end{theorem}
\defn[\textbf{Abel Summability and Abel Mean} \cite{abelsum}]{
Let $(a_n)$ be a sequence of real numbers. $\sum\limits_{n=0}^{\infty}a_n$ be a series. Let $$f(x)=\sum\limits_{k=0}^{\infty}a_nx^n$$ be power series. Then the sequence is Abel Summable if the power series $f(x)$ converges with a radius of convergence $|x|<1$. 
}
\begin{theorem}[\cite{cesaroimplyabel}]
Let $(a_n)$ be a sequence of real numbers. If the series $\sum\limits_{n=0}^{\infty}a_n$ is Cesaro Summable then it is Abel Summable. 
\end{theorem}