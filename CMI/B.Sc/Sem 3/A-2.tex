\documentclass[11pt]{article}




\usepackage[all]{xy}
\usepackage{graphics}
\usepackage{enumitem}
\usepackage{epsfig}
\usepackage{amsmath,amsthm}
\usepackage{amscd}
\usepackage{tikz-cd}
\usepackage{verbatim}
\usepackage{pdfpages}
%\usepackage{showkeys}
\usepackage{amsfonts,latexsym,amssymb, fullpage, mathtools}
\usepackage{parskip}
%\usepackage{MnSymbol}
\usepackage{hyperref}
\hypersetup{
	colorlinks=true,
	linkcolor=blue,
	filecolor=magenta,      
	urlcolor=blue!70!red,
	pdftitle={Assignment}, %%%%%%%%%%%%%%%%   WRITE ASSIGNMENT PDF NAME  %%%%%%%%%%%%%%%%%%%%
}
\usepackage{mdwlist}
%\usepackage{tgbonum}
\usepackage[utf8]{inputenc}
\usepackage{amsmath}





%%%%%%%%%%%%%%%%%%%%%%%%%%%%%%%%%%%%%%%%%%%%%%%%%%%%%%%%



\newcommand{\cA}{{\mathcal{A}}}   \newcommand{\cB}{{\mathcal{B}}}
\newcommand{\cC}{{\mathcal{C}}}   \newcommand{\cD}{{\mathcal{D}}}
\newcommand{\cE}{{\mathcal{E}}}   \newcommand{\cF}{{\mathcal{F}}}
\newcommand{\cG}{{\mathcal{G}}}   \newcommand{\cH}{{\mathcal{H}}}
\newcommand{\cI}{{\mathcal{I}}}   \newcommand{\cJ}{{\mathcal{J}}}
\newcommand{\cK}{{\mathcal{K}}}   \newcommand{\cL}{{\mathcal{L}}}
\newcommand{\cM}{{\mathcal{M}}}   \newcommand{\cN}{{\mathcal{N}}}
\newcommand{\cO}{{\mathcal{O}}}   \newcommand{\cP}{{\mathcal{P}}}
\newcommand{\cQ}{{\mathcal{Q}}}   \newcommand{\cR}{{\mathcal{R}}}
\newcommand{\cS}{{\mathcal{S}}}   \newcommand{\cT}{{\mathcal{T}}}
\newcommand{\cU}{{\mathcal{U}}}   \newcommand{\cV}{{\mathcal{V}}}
\newcommand{\cW}{{\mathcal{W}}}   \newcommand{\cX}{{\mathcal{X}}}
\newcommand{\cY}{{\mathcal{Y}}}   \newcommand{\cZ}{{\mathcal{Z}}}

\newcommand{\hcP}{\hat{\mathcal{P}}}
\newcommand{\hcQ}{\hat{\mathcal{Q}}}
\newcommand{\hcR}{\hat{\mathcal{R}}}
\newcommand{\hcL}{\hat{\mathcal{L}}}
\newcommand{\hcM}{\hat{\mathcal{M}}} \newcommand{\hphi}{\hat{\phi}}
\newcommand{\bbk}{\mathbb{k}}   \newcommand{\bfv}{\mathbf{v}}
\newcommand{\bfnu}{\mathbf{nu}}  \newcommand{\hXX}{\hat{\mathbb{X}}}
\newcommand{\bJ}{\mathbf{J}}

\newcommand{\hD}{{\hat{D}}}   \newcommand{\hE}{{\hat{E}}}
\newcommand{\hF}{{\hat{F}}}   \newcommand{\hH}{{\hat{H}}}
\newcommand{\hY}{{\hat{Y}}}   \newcommand{\hP}{{\hat{P}}}
\newcommand{\hT}{{\hat{T}}}   \newcommand{\hQ}{{\hat{Q}}}
\newcommand{\hq}{{\hat{q}}}
\newcommand{\hr}{{\hat{r}}}
\newcommand{\hu}{{\hat{u}}}
\newcommand{\hv}{{\hat{v}}}
\newcommand{\hf}{{\hat{f}}}
\newcommand{\hg}{{\hat{g}}}
\newcommand{\hw}{{\hat{w}}}
\newcommand{\hS}{{\hat{S}}}
\newcommand{\hV}{{\hat{V}}}
\newcommand{\hG}{{\hat{G}}}
\newcommand{\hmu}{{\hat{\mu}}}
\newcommand {\y}{\V{y}}
\newcommand {\V}[1]{\mbox{\boldmath$#1$}}
\newcommand{\iExp}{{\mathrm{iExp\,}}}

\newcommand{\htheta}{{\hat{\theta}}}



\newcommand{\htu}{{\hat{\tilde{u}}}}


\newcommand{\hTR}{{\widehat{TR}}}
\newcommand{\tsigma}{{\tilde{\sigma}}}
\newcommand{\tphi}{{\tilde{\phi}}}
\newcommand{\tpsi}{{\tilde{\psi}}}
\newcommand{\tzeta}{{\tilde{\zeta}}}
\newcommand{\tdelta}{{\tilde{\delta}}}
\newcommand{\tgamma}{{\tilde{\gamma}}}
\newcommand{\tGamma}{{\tilde{\Gamma}}}
\newcommand{\tlog}{{\widetilde{\log}}}


\newcommand{\txi}{{\tilde{\xi}}}
\newcommand{\tomega}{{\tilde{\omega}}}
\newcommand{\tH}{{\tilde{H}}}
\newcommand{\tI}{{\tilde{I}}}

\newcommand{\tX}{{\tilde{X}}}
\newcommand{\tV}{{\tilde{V}}}
\newcommand{\tz}{{\tilde{z}}}
\newcommand{\ty}{{\tilde{y}}}
\newcommand{\tx}{{\tilde{x}}}
\newcommand{\te}{{\tilde{e}}}
\newcommand{\tf}{{\tilde{f}}}
\newcommand{\tg}{{\tilde{g}}}
\newcommand{\tu}{{\tilde{u}}}
\newcommand{\tm}{{\tilde{m}}}
\newcommand{\tn}{{\tilde{n}}}
\newcommand{\tilt}{{\tilde{t}}}
\newcommand{\tT}{{\tilde{T}}}
\newcommand{\tL}{{\tilde{L}}}
\newcommand{\tQ}{{\tilde{Q}}}
\newcommand{\tB}{{\tilde{B}}}
\newcommand{\tC}{{\tilde{C}}}
\newcommand{\tD}{{\tilde{D}}}
\newcommand{\tU}{{\tilde{U}}}
\newcommand{\utL}{{\underline{\tilde{L}}}}
\newcommand{\tF}{{\tilde{F}}}\newcommand{\tilh}{{\tilde{h}}}
\newcommand{\tk}{{\tilde{k}}}
\newcommand{\tv}{{\tilde{v}}}
\newcommand{\tw}{{\tilde{w}}}
\newcommand{\bx}{\mathbf x}
\newcommand{\bz}{\mathbf z}
\newcommand{\bu}{\mathbf u}
\newcommand{\bv}{\mathbf v}
\newcommand{\bt}{\mathbf t}
\newcommand{\bi}{\mathbf i}
\newcommand{\bj}{\mathbf j}
\newcommand{\bL}{\mathbf L}
\newcommand{\bN}{\mathbf N}
\newcommand{\bM}{\mathbf M}
\newcommand{\bB}{\mathbf B}
\newcommand{\bA}{\mathbf A}
\newcommand{\tbz}{{\tilde{\mathbf z}}}
\newcommand{\hbx}{\hat{\mathbf x}}
\newcommand{\tcO}{{\tilde{\mathcal{O}}}}
\newcommand{\tcC}{{\tilde{\mathcal{C}}}}
\newcommand{\ocC}{{\overline{\mathcal{C}}}}
\newcommand{\tcR}{{\tilde{\mathcal{R}}}}
\newcommand{\tcA}{{\tilde{\mathcal{A}}}}








\newcommand{\uL}{\underline L}
\newcommand{\uM}{\underline M}
\newcommand{\uE}{\underline E}

\newcommand{\chA}{\check A}
\newcommand{\chE}{\check E}
\newcommand{\chL}{\check L}
\newcommand{\chV}{\check V}
\newcommand{\chv}{\check v}
\newcommand{\chw}{\check w}
\newcommand{\chW}{\check W}
\newcommand{\chM}{\check M}
\newcommand{\chQ}{\check Q}
\newcommand{\chsigma}{\check\sigma}



\newcommand{\uchL}{\underline{\check L}}






\newcommand{\BA}{\mathbb{A}}  \newcommand{\BB}{\mathbb{B}}
\newcommand{\CC}{\mathbb{C}}  \newcommand{\EE}{\mathbb{E}}
\newcommand{\FF}{\mathbb{F}}  \newcommand{\HH}{\mathbb{H}}
\newcommand{\JJ}{\mathbb{J}}  \newcommand{\LL}{\mathbb{L}}
\newcommand{\NN}{\mathbb{N}}  \newcommand{\PP}{\mathbb{P}}
\newcommand{\QQ}{\mathbb{Q}}  \newcommand{\RR}{\mathbb{R}}
\newcommand{\TT}{\mathbb{T}}  \newcommand{\VV}{\mathbb{V}}
\newcommand{\XX}{\mathbb{X}}  \newcommand{\WW}{\mathbb{W}}
\newcommand{\ZZ}{\mathbb{Z}}

\newcommand{\FM}{\mathfrak{M}}
\newcommand{\fm}{\mathfrak{m}}


\newcommand{\isom}{\cong}
\newcommand{\Ext}{\operatorname{Ext}}
\newcommand{\Grass}{\operatorname{Grass}}
\newcommand{\coker}{\operatorname{coker}}
\newcommand{\Hilb}{\operatorname{Hilb}}
\newcommand{\Hom}{\operatorname{Hom}}
\newcommand{\Quot}{\operatorname{Quot}}
\newcommand{\Pic}{\operatorname{Pic}}
\newcommand{\NS}{\operatorname{NS}}
\newcommand{\Sym}{\operatorname{Sym}}
\newcommand{\id}{\operatorname{I}}
\newcommand{\im}{\operatorname{im}}
\newcommand{\surj}{\twoheadrightarrow}
\newcommand{\inj}{\hookrightarrow}
\newcommand{\gr}{\operatorname{gr}}
\newcommand{\rk}{\operatorname{rk}}
\newcommand{\reg}{\operatorname{reg}}
\newcommand{\wt}{\widetilde}
\newcommand{\del}{{\partial}}
\newcommand{\delb}{{\overline\partial}}

\newcommand{\oX}{{\overline X}}
\newcommand{\oD}{{\overline D}}
\newcommand{\ox}{{\overline x}}
\newcommand{\ow}{{\overline w}}
\newcommand{\oz}{{\overline z}}
\newcommand{\oh}{{\overline{h}}}
\newcommand{\oalpha}{{\overline \alpha}}
\newcommand{\ndiv}{\hspace{-4pt}\not|\hspace{2pt}}




\newcommand{\Res}{\operatorname{Res}}
\newcommand{\ch}{\operatorname{ch}}
\newcommand{\tr}{\operatorname{tr}}
\newcommand{\pardeg}{\operatorname{par-deg}}
\newcommand{\ad}{{ad\,}}
\newcommand{\diag}{\operatorname{diag}}
\newcommand{\codim}{\operatorname{codim}}

\hyphenation{pa-ra-bo-lic}
\newcommand{\bbQ}{\mathbb{Q}}
\newcommand{\bbR}{\mathbb{R}}
\newcommand{\bbP}{\mathbb{P}}
\newcommand{\bbC}{\mathbb{C}}
\newcommand{\bbT}{\mathbb{T}}
\newcommand{\bbU}{\mathbb{U}}
\newcommand{\bbZ}{\mathbb{Z}}
\newcommand{\bbN}{\mathbb{N}}
\newcommand{\bbF}{\mathbb{F}}





\newtheorem{proposition}{Proposition}[section]
\newtheorem{theorem}[proposition]{Theorem}
\newtheorem{lemma}[proposition]{Lemma}
\newtheorem{conjecture}[proposition]{Conjecture}
\newtheorem{corollary}[proposition]{Corollary}



%\theoremstyle{definition}
%\newtheorem{definition}[proposition]{Definition}
%\newtheorem{remark}[proposition]{Remark}
%\newtheorem{notation}[proposition]{Notation}
%\newtheorem{example}[proposition]{Example}
%\newtheorem{ex}{Exercise}[section]

\usepackage[most,many,breakable]{tcolorbox}



\definecolor{mytheorembg}{HTML}{F2F2F9}
\definecolor{mytheoremfr}{HTML}{00007B}


\tcbuselibrary{theorems,skins,hooks}
\newtcbtheorem{problem}{Problem}
{%
	enhanced,
	breakable,
	colback = mytheorembg,
	frame hidden,
	boxrule = 0sp,
	borderline west = {2pt}{0pt}{mytheoremfr},
	sharp corners,
	detach title,
	before upper = \tcbtitle\par\smallskip,
	coltitle = mytheoremfr,
	fonttitle = \bfseries\sffamily,
	description font = \mdseries,
	separator sign none,
	segmentation style={solid, mytheoremfr},
}
{p}

\newcommand{\Qed}{\begin{flushright}\qed\end{flushright}}
\newcommand{\solve}[1]{\setlength{\parindent}{0cm}\textbf{\textit{Solution: }}\setlength{\parindent}{1cm}#1 \Qed}
 

%%%%%%%%%%%%%%%%%%%%%%%%%%%%%%%%%%%%%%%%%%%%%%%%%%%%%%%%

\begin{document}

\title{\textbf{Fulton chapter 3: Local Properties of Plane Curves}\\ Intersection Numbers}
\date{}
\author{}
\maketitle
\textsf{\noindent \large\textbf{Aritra Kundu} \hfill \textbf{Problem Set - 2}\\
	\textbf{Email}: \href{aritra@cmi.ac.in}{aritra@cmi.ac.in} \hfill \textbf{Topic}: Algebraic Geometry\\
	\noindent\rule{\textwidth}{2.8pt}}
$k$ is an algebraically closed field. $F,G$ are two affine plane curves in $k[X,Y]$. $P\in A^2$


\textbf{Property 1: }$I(P,F\cap G)\geq 0$,the equality holds if and only if $P\notin V(F)\cap V(G)$

\textbf{Property 2: }Assume $F,G$ both passes through $P$. $I(P,F\cap G)<\infty \iff F,G$ has no common component passes through $P$. Otherwise $I(P,F\cap G)=\infty$

\textbf{Property 3: }For any affine transformation $T$ $I(P,F\cap G)=I(Q,F^T\cap G^T)$ where $T(P)=Q$

\textbf{Property 4: }$I(P,F\cap G)=I(P,G\cap F)$. Let $P$ be a simple point of both $F,G$. $F$ and $G$ are said to be intersected transversely if  they do not share the tangent at $P$.

\textbf{Property 5: }$I(P,F\cap G)\geq m_P(F)m_P(G)$ the equality holds if and only if $F$ and $G$ have no common tangent at $P$. This property requires to ensure the condition that  $F,G$ intersect transversely if and only if $I(P,F\cap G)=1$

\textbf{Property 6: }$I(P,F\cap GH)=I(P,F\cap G)+I(P,F\cap H)$

\textbf{Property 7: }$I(P,F\cap G)=I(P,F\cap G+AF) \forall$ polynomial $A$ in $k[x,y]$. If $F$ is irreducible then  $I(P,F\cap G)$ only depends on the image of $G$ in $\tau(F)$

\textbf{Definition: } If $F,G$ has no common component passing through $P$ then $F,G$ has said to be intersected properly. Lets assume $I(P,F\cap G)$ exists for any two affine curves. 

\begin{problem}{Claim}{}
	Intersection number of $F,G$ at $P$,$I(P,F\cap G)$ which has the 7 properties is unique.
\end{problem}
\solve{We can assume $P=(0,0)$[as by property 3 we can apply an affine transformation to make $P$ to be origin by keeping unchanged  $I(P,F\cap G)$]
	\begin{enumerate}
		\item If $F,G$ has a  common component passing through $P$ 
then $I(P,F\cap G)=\infty$[by property 2]
\item $I(P,F\cap G)=0\iff$ either $F(P)\neq 0$ or $G(P)\neq 0$
\item $I(P,F\cap G)=m_P(F)m_p(G)\iff F,G$ do not share any tangent at $P$.
\end{enumerate}
So we can assume  $F,G$ has intersected properly and $I(P,F\cap G)$ can not be computed directly from the 3 properties mentioned. Let $F(X,0),G(X,0)$ are  polynomials of degree of degree $r,s$ respectively. Let's assume $r=0$. $P(n)$ be the statement that if $I(P,F\cap G)$ has a value less than $n$ then it is unique.

$P(1)$ is true [by property 1][as then $I(P,F\cap G)=0\iff P\notin V(F)\cap V(G)]$. Let $P(n)$ be true. Let $F,G$ be affine curves such that $I(P,F\cap G)$ has a value equal to $n$. So, $F=YH_1$ and $G=Yg_1+g_2(X)$ where $$g_2(X)=X^{m_1}(a_0+a_1X...a_{s-m_1}X^{s-m_1}.a_0\neq 0\ [ m_1>0\text{ as }P\in V(G)]$$Now by property 6 $I(P,F\cap G)=I(P,Y\cap G)+I(P, H_1\cap G)$
\begin{multline*}
	I(P,Y\cap G)=I(P,Y\cap X^{m_1})+I(P,Y\cap (a_0+a_1X+..a_{s-m_1}X^{s-m_1}))=m_1(I(P,Y\cap X)\\
	=m_1\ [\text{as }a_0\neq 0\implies I(P,Y\cap (a_0+a_1X+..a_{s-m_1}X^{s-m_1}))=0]
\end{multline*}so,$$I(P,H_1\cap G)+m_1=I(P,F\cap G)$$

Now $I(P,H_1\cap G)<I(P,F\cap G)$ so $I(P,H_1\cap G)<n$ so $I(P,H_1\cap G)$ is unique and so $I(P,F\cap G)$ is unique. Therefore $P(n+1)$ is true. So, $P(n)$ is true $\forall n\in \mathbb{N}$[by principle of mathematical induction]. 

Let $r> 0$. WLOG assume that $r\leq s$[by  property 4]. $a,b$ be the leading coefficients of $F(X,0)$ and $G(X,0)$. Let $H_1=G-(b/a)X^{s-r}F$. So, $$I(P,F\cap G)=I(P,F\cap H_1)$$clearly $deg(H_1(X,0))<deg(G(X,0))$. If $deg(H_1(X,0))>deg(F(X,0))$ then repeating the process for finite number of times we get $H$ s.t.$deg(H(X,0))<deg(F(X,0))$  and  $I(P,F\cap G)=I(P,F\cap H)$. Then interchanging the role of $F,H$  and after repeating the process for finitely many times we can make the minimum of $degA(X,0),degB(X,0)$ is 0 and then go to the previous case.
}














\end{document}