\documentclass[11pt]{article}




\usepackage[all]{xy}
\usepackage{graphics}
\usepackage{enumitem}
\usepackage{epsfig}
\usepackage{amsmath,amsthm}
\usepackage{amscd}
\usepackage{tikz-cd}
\usepackage{verbatim}
\usepackage{pdfpages}
%\usepackage{showkeys}
\usepackage{amsfonts,latexsym,amssymb, fullpage, mathtools}
\usepackage{parskip}
%\usepackage{MnSymbol}
\usepackage{hyperref}
\hypersetup{
	colorlinks=true,
	linkcolor=blue,
	filecolor=magenta,      
	urlcolor=blue!70!red,
	pdftitle={Assignment}, %%%%%%%%%%%%%%%%   WRITE ASSIGNMENT PDF NAME  %%%%%%%%%%%%%%%%%%%%
}
\usepackage{mdwlist}
%\usepackage{tgbonum}
\usepackage[utf8]{inputenc}
\usepackage{amsmath}





%%%%%%%%%%%%%%%%%%%%%%%%%%%%%%%%%%%%%%%%%%%%%%%%%%%%%%%%



\newcommand{\cA}{{\mathcal{A}}}   \newcommand{\cB}{{\mathcal{B}}}
\newcommand{\cC}{{\mathcal{C}}}   \newcommand{\cD}{{\mathcal{D}}}
\newcommand{\cE}{{\mathcal{E}}}   \newcommand{\cF}{{\mathcal{F}}}
\newcommand{\cG}{{\mathcal{G}}}   \newcommand{\cH}{{\mathcal{H}}}
\newcommand{\cI}{{\mathcal{I}}}   \newcommand{\cJ}{{\mathcal{J}}}
\newcommand{\cK}{{\mathcal{K}}}   \newcommand{\cL}{{\mathcal{L}}}
\newcommand{\cM}{{\mathcal{M}}}   \newcommand{\cN}{{\mathcal{N}}}
\newcommand{\cO}{{\mathcal{O}}}   \newcommand{\cP}{{\mathcal{P}}}
\newcommand{\cQ}{{\mathcal{Q}}}   \newcommand{\cR}{{\mathcal{R}}}
\newcommand{\cS}{{\mathcal{S}}}   \newcommand{\cT}{{\mathcal{T}}}
\newcommand{\cU}{{\mathcal{U}}}   \newcommand{\cV}{{\mathcal{V}}}
\newcommand{\cW}{{\mathcal{W}}}   \newcommand{\cX}{{\mathcal{X}}}
\newcommand{\cY}{{\mathcal{Y}}}   \newcommand{\cZ}{{\mathcal{Z}}}

\newcommand{\hcP}{\hat{\mathcal{P}}}
\newcommand{\hcQ}{\hat{\mathcal{Q}}}
\newcommand{\hcR}{\hat{\mathcal{R}}}
\newcommand{\hcL}{\hat{\mathcal{L}}}
\newcommand{\hcM}{\hat{\mathcal{M}}} \newcommand{\hphi}{\hat{\phi}}
\newcommand{\bbk}{\mathbb{k}}   \newcommand{\bfv}{\mathbf{v}}
\newcommand{\bfnu}{\mathbf{nu}}  \newcommand{\hXX}{\hat{\mathbb{X}}}
\newcommand{\bJ}{\mathbf{J}}

\newcommand{\hD}{{\hat{D}}}   \newcommand{\hE}{{\hat{E}}}
\newcommand{\hF}{{\hat{F}}}   \newcommand{\hH}{{\hat{H}}}
\newcommand{\hY}{{\hat{Y}}}   \newcommand{\hP}{{\hat{P}}}
\newcommand{\hT}{{\hat{T}}}   \newcommand{\hQ}{{\hat{Q}}}
\newcommand{\hq}{{\hat{q}}}
\newcommand{\hr}{{\hat{r}}}
\newcommand{\hu}{{\hat{u}}}
\newcommand{\hv}{{\hat{v}}}
\newcommand{\hf}{{\hat{f}}}
\newcommand{\hg}{{\hat{g}}}
\newcommand{\hw}{{\hat{w}}}
\newcommand{\hS}{{\hat{S}}}
\newcommand{\hV}{{\hat{V}}}
\newcommand{\hG}{{\hat{G}}}
\newcommand{\hmu}{{\hat{\mu}}}
\newcommand {\y}{\V{y}}
\newcommand {\V}[1]{\mbox{\boldmath$#1$}}
\newcommand{\iExp}{{\mathrm{iExp\,}}}

\newcommand{\htheta}{{\hat{\theta}}}



\newcommand{\htu}{{\hat{\tilde{u}}}}


\newcommand{\hTR}{{\widehat{TR}}}
\newcommand{\tsigma}{{\tilde{\sigma}}}
\newcommand{\tphi}{{\tilde{\phi}}}
\newcommand{\tpsi}{{\tilde{\psi}}}
\newcommand{\tzeta}{{\tilde{\zeta}}}
\newcommand{\tdelta}{{\tilde{\delta}}}
\newcommand{\tgamma}{{\tilde{\gamma}}}
\newcommand{\tGamma}{{\tilde{\Gamma}}}
\newcommand{\tlog}{{\widetilde{\log}}}


\newcommand{\txi}{{\tilde{\xi}}}
\newcommand{\tomega}{{\tilde{\omega}}}
\newcommand{\tH}{{\tilde{H}}}
\newcommand{\tI}{{\tilde{I}}}

\newcommand{\tX}{{\tilde{X}}}
\newcommand{\tV}{{\tilde{V}}}
\newcommand{\tz}{{\tilde{z}}}
\newcommand{\ty}{{\tilde{y}}}
\newcommand{\tx}{{\tilde{x}}}
\newcommand{\te}{{\tilde{e}}}
\newcommand{\tf}{{\tilde{f}}}
\newcommand{\tg}{{\tilde{g}}}
\newcommand{\tu}{{\tilde{u}}}
\newcommand{\tm}{{\tilde{m}}}
\newcommand{\tn}{{\tilde{n}}}
\newcommand{\tilt}{{\tilde{t}}}
\newcommand{\tT}{{\tilde{T}}}
\newcommand{\tL}{{\tilde{L}}}
\newcommand{\tQ}{{\tilde{Q}}}
\newcommand{\tB}{{\tilde{B}}}
\newcommand{\tC}{{\tilde{C}}}
\newcommand{\tD}{{\tilde{D}}}
\newcommand{\tU}{{\tilde{U}}}
\newcommand{\utL}{{\underline{\tilde{L}}}}
\newcommand{\tF}{{\tilde{F}}}\newcommand{\tilh}{{\tilde{h}}}
\newcommand{\tk}{{\tilde{k}}}
\newcommand{\tv}{{\tilde{v}}}
\newcommand{\tw}{{\tilde{w}}}
\newcommand{\bx}{\mathbf x}
\newcommand{\bz}{\mathbf z}
\newcommand{\bu}{\mathbf u}
\newcommand{\bv}{\mathbf v}
\newcommand{\bt}{\mathbf t}
\newcommand{\bi}{\mathbf i}
\newcommand{\bj}{\mathbf j}
\newcommand{\bL}{\mathbf L}
\newcommand{\bN}{\mathbf N}
\newcommand{\bM}{\mathbf M}
\newcommand{\bB}{\mathbf B}
\newcommand{\bA}{\mathbf A}
\newcommand{\tbz}{{\tilde{\mathbf z}}}
\newcommand{\hbx}{\hat{\mathbf x}}
\newcommand{\tcO}{{\tilde{\mathcal{O}}}}
\newcommand{\tcC}{{\tilde{\mathcal{C}}}}
\newcommand{\ocC}{{\overline{\mathcal{C}}}}
\newcommand{\tcR}{{\tilde{\mathcal{R}}}}
\newcommand{\tcA}{{\tilde{\mathcal{A}}}}








\newcommand{\uL}{\underline L}
\newcommand{\uM}{\underline M}
\newcommand{\uE}{\underline E}

\newcommand{\chA}{\check A}
\newcommand{\chE}{\check E}
\newcommand{\chL}{\check L}
\newcommand{\chV}{\check V}
\newcommand{\chv}{\check v}
\newcommand{\chw}{\check w}
\newcommand{\chW}{\check W}
\newcommand{\chM}{\check M}
\newcommand{\chQ}{\check Q}
\newcommand{\chsigma}{\check\sigma}



\newcommand{\uchL}{\underline{\check L}}






\newcommand{\BA}{\mathbb{A}}  \newcommand{\BB}{\mathbb{B}}
\newcommand{\CC}{\mathbb{C}}  \newcommand{\EE}{\mathbb{E}}
\newcommand{\FF}{\mathbb{F}}  \newcommand{\HH}{\mathbb{H}}
\newcommand{\JJ}{\mathbb{J}}  \newcommand{\LL}{\mathbb{L}}
\newcommand{\NN}{\mathbb{N}}  \newcommand{\PP}{\mathbb{P}}
\newcommand{\QQ}{\mathbb{Q}}  \newcommand{\RR}{\mathbb{R}}
\newcommand{\TT}{\mathbb{T}}  \newcommand{\VV}{\mathbb{V}}
\newcommand{\XX}{\mathbb{X}}  \newcommand{\WW}{\mathbb{W}}
\newcommand{\ZZ}{\mathbb{Z}}

\newcommand{\FM}{\mathfrak{M}}
\newcommand{\fm}{\mathfrak{m}}


\newcommand{\isom}{\cong}
\newcommand{\Ext}{\operatorname{Ext}}
\newcommand{\Grass}{\operatorname{Grass}}
\newcommand{\coker}{\operatorname{coker}}
\newcommand{\Hilb}{\operatorname{Hilb}}
\newcommand{\Hom}{\operatorname{Hom}}
\newcommand{\Quot}{\operatorname{Quot}}
\newcommand{\Pic}{\operatorname{Pic}}
\newcommand{\NS}{\operatorname{NS}}
\newcommand{\Sym}{\operatorname{Sym}}
\newcommand{\id}{\operatorname{I}}
\newcommand{\im}{\operatorname{im}}
\newcommand{\surj}{\twoheadrightarrow}
\newcommand{\inj}{\hookrightarrow}
\newcommand{\gr}{\operatorname{gr}}
\newcommand{\rk}{\operatorname{rk}}
\newcommand{\reg}{\operatorname{reg}}
\newcommand{\wt}{\widetilde}
\newcommand{\del}{{\partial}}
\newcommand{\delb}{{\overline\partial}}

\newcommand{\oX}{{\overline X}}
\newcommand{\oD}{{\overline D}}
\newcommand{\ox}{{\overline x}}
\newcommand{\ow}{{\overline w}}
\newcommand{\oz}{{\overline z}}
\newcommand{\oh}{{\overline{h}}}
\newcommand{\oalpha}{{\overline \alpha}}
\newcommand{\ndiv}{\hspace{-4pt}\not|\hspace{2pt}}




\newcommand{\Res}{\operatorname{Res}}
\newcommand{\ch}{\operatorname{ch}}
\newcommand{\tr}{\operatorname{tr}}
\newcommand{\pardeg}{\operatorname{par-deg}}
\newcommand{\ad}{{ad\,}}
\newcommand{\diag}{\operatorname{diag}}
\newcommand{\codim}{\operatorname{codim}}

\hyphenation{pa-ra-bo-lic}
\newcommand{\bbQ}{\mathbb{Q}}
\newcommand{\bbR}{\mathbb{R}}
\newcommand{\bbP}{\mathbb{P}}
\newcommand{\bbC}{\mathbb{C}}
\newcommand{\bbT}{\mathbb{T}}
\newcommand{\bbU}{\mathbb{U}}
\newcommand{\bbZ}{\mathbb{Z}}
\newcommand{\bbN}{\mathbb{N}}
\newcommand{\bbF}{\mathbb{F}}





\newtheorem{proposition}{Proposition}[section]
\newtheorem{theorem}[proposition]{Theorem}
\newtheorem{lemma}[proposition]{Lemma}
\newtheorem{conjecture}[proposition]{Conjecture}
\newtheorem{corollary}[proposition]{Corollary}



%\theoremstyle{definition}
%\newtheorem{definition}[proposition]{Definition}
%\newtheorem{remark}[proposition]{Remark}
%\newtheorem{notation}[proposition]{Notation}
%\newtheorem{example}[proposition]{Example}
%\newtheorem{ex}{Exercise}[section]

\usepackage[most,many,breakable]{tcolorbox}



\definecolor{mytheorembg}{HTML}{F2F2F9}
\definecolor{mytheoremfr}{HTML}{00007B}


\tcbuselibrary{theorems,skins,hooks}
\newtcbtheorem{problem}{Problem}
{%
	enhanced,
	breakable,
	colback = mytheorembg,
	frame hidden,
	boxrule = 0sp,
	borderline west = {2pt}{0pt}{mytheoremfr},
	sharp corners,
	detach title,
	before upper = \tcbtitle\par\smallskip,
	coltitle = mytheoremfr,
	fonttitle = \bfseries\sffamily,
	description font = \mdseries,
	separator sign none,
	segmentation style={solid, mytheoremfr},
}
{p}

\newcommand{\Qed}{\begin{flushright}\qed\end{flushright}}
\newcommand{\solve}[1]{\setlength{\parindent}{0cm}\textbf{\textit{Solution: }}\setlength{\parindent}{1cm}#1 \Qed}


%%%%%%%%%%%%%%%%%%%%%%%%%%%%%%%%%%%%%%%%%%%%%%%%%%%%%%%%

\begin{document}
	
	\title{\textbf{Fulton Chapter 5}\\ Linear System of Curves and B\'{e}zout's Theorem}
	\date{}
	\author{}
	\maketitle
	\textsf{\noindent \large\textbf{Aritra Kundu} \hfill \textbf{Problem Set - 3}\\
		\textbf{Email}: \href{aritra@cmi.ac.in}{aritra@cmi.ac.in} \hfill \textbf{Topic}: Algebraic Geometry\\
		\noindent\rule{\textwidth}{2.8pt}}
	
\begin{problem}{Linear System of Curves: 5.19}{p1}
	$A=\{(a,b,1)\mid\forall\ a,b\in \{0,1,2\}\}$. Show that there is infinitely many cubics passing through the 9 points in $A$
\end{problem}
\solve{Let $F$ be a cubic passing through the 9 points in $A$. So, $F_*$ will pass through $$B=\{(a,b)|\forall a,b\in \{0,1,2\}\}\quad  [*\text{ taking w.r.t }Z]$$

Now $F_*=g(X)+Y(H(X,Y))$ as \begin{multline*}
	F_*(a,0)=0\ \forall\ a\in\{0,1,2\}\implies g(x)=\lambda (X-2)(X-1)X \\
	[\text{as }F\text{ is cubic }\implies g\text{ is a deg 3 polynomial}]
\end{multline*}. 

Similarly $$F_*=g_1(Y)+XH(X,Y)\implies g_1(Y)=\mu (Y-2)(Y-1)Y$$So, $$F_*=\lambda (X-2)(X-1)X+\mu (Y-2)(Y-1)Y+XY(aX+bY+c)$$But $$F_*(1,2)=0=F_*(2,1)=F_*(1,1)\implies a+2b+c=0=2a+b+c=a+b+c\implies a=b=c=0$$So, any polynomial passing through $B$ must be of the form $$\lambda (X-2)(X-1)X+\mu (Y-2)(Y-1)Y$$ where $\lambda,\mu\in k$. So,  any $F$ passing through the points in $A$ will be of the form $$\lambda (X-2Z)(X-Z)X+\mu (Y-2Z)(Y-Z)Y$$
so there are infinitely curves passing through 9 points in $A$}


\begin{problem}{B\'{e}zout's Theorem: 5.23}{p2}
	$F$ is a projective plane curve of degree n ,it contains no lines and char($k$)=0 then
\begin{enumerate}[label=(\alph*)]
	\item if $P\in H\cap F$ then either it is a multiple point or a flex.
	\item $I(P,H\cap F)=1\iff P$ is an ordinary flex.
\end{enumerate}
\end{problem}

\solve{\parinf
	
\textbf{\textit{Claim 1 }}: $T$ be a projective change of co-ordinates then hessian of $$F^{T}=det(A)^2 H^T $$

\textbf{\textit{Proof: }}Let $T=(T_1,T_2,T_3);T_i=a_iX+b_iY+c_iZ$. So the matrix of $T$ is $A=\begin{bmatrix}
a_1 & b_1 & c_1\\
a_2 & b_2 & c_2\\
a_3 & b_3 & c_3
\end{bmatrix}$

\begin{multline*}
	(F(T_1,T_2,T_3))_{x}=a_1F_{T_1}(T_1,T_2,T_3)+a_2F_{T_2}(T_1,T_2,T_3)+a_3F_{T_3}(T_1,T_2,T_3)\\
	=\begin{bmatrix}
 F_{T_1} & F_{T_2} & F_{T_3}\\
\end{bmatrix} \begin{bmatrix}
a_1 \\
a_2 \\
a_3
\end{bmatrix}\quad [\text{by chain rule}]
\end{multline*} 

$$(F(T_1,T_2,T_3))_{xx}=\sum_{i=1}^3\sum_{j=1}^3a_i(a_j F_{T_iT_j})$$
So, $F_{xx}(T_1,T_2,T_3)$ is the (1,1) position of $\begin{bmatrix}
a_1 & a_2 & a_3\\
b_1 & b_2 & b_3\\
c_1 & c_2 & c_3
\end{bmatrix}\begin{bmatrix}
F_{T_1T_1} & F_{T_1T_2} & F_{T_1T_3}\\
F_{T_2T_1} & F_{T_2T_2} & F_{T_2T_3}\\
F_{T_3T_1} & F_{T_3T_2} & F_{T_3T_3}
\end{bmatrix}\begin{bmatrix}
a_1 & b_1 & c_1\\
a_2 & b_2 & c_2\\
a_3 & b_3 & c_3
\end{bmatrix}$. So, $$\begin{bmatrix}
a_1 & a_2 & a_3\\
b_1 & b_2 & b_3\\
c_1 & c_2 & c_3
\end{bmatrix}\begin{bmatrix}
F_{T_1T_1} & F_{T_1T_2} & F_{T_1T_3}\\
F_{T_2T_1} & F_{T_2T_2} & F_{T_2T_3}\\
F_{T_3T_1} & F_{T_3T_2} & F_{T_3T_3}
\end{bmatrix}\begin{bmatrix}
a_1 & b_1 & c_1\\
a_2 & b_2 & c_2\\
a_3 & b_3 & c_3
\end{bmatrix}=\begin{bmatrix}
(F^T)_{xx} & (F^T)_{xy} & (F^T)_{xz}\\
(F^T)_{yx} & (F^T)_{yy} & (F^T)_{yz}\\
(F^T)_{zx} & (F^T)_{zy} & (F^T)_{zz}
\end{bmatrix}$$
So, $R=$ hessian of $F^{T}=det(A)^2 H^T $. So, $$P\in H\cap F \iff T(P)\in R\cap F^T\iff T(P) \in H^T\cap F^T$$and $$I(P,F\cap H)=1\iff I(T(P),F^T\cap H^T)=1\iff I(T(P),F^T\cap R)=1$$so we can assume $P=(0,0,1)$\Qed

\textbf{\textit{Claim 2: }}$(n-1)F_j=\sum_{i}X_i F_{ij}$

\textbf{\textit{Proof: }}Let \begin{multline*}
	F=\sum_{k=0}^r X_j^k F_k\quad[\text{where }\deg F_k=n-k]\implies F_j=\sum_{k=1}^r kX_j^{k-1}F_k \\
	\implies deg(F_j)=n-1\quad [\text{as char }K=0]
\end{multline*}
Applying Euler's theorem on $F_j$ we get the relation $$(n-1)F_j=\sum_iX_iF_{ji}=\sum_iX_iF_{ij}\quad [\text{as }F_{ij}=F_{ji}]$$\Qed

\textbf{\textit{Claim 3: }}$I(P,f\cap h)=I(P,f\cap g)$ where $g=f_y^2f_{xx}+f_x^2f_{yy}-2f_xf_yf_{xy}$


\textbf{\textit{Proof: }}Let $$F(X,Y,Z)=\sum_{k=0}^r X^kF_k(Y,Z)\implies F_X(X,Y,1)=\sum_{k=1}^{r} k X^{k-1}F_k(Y,1)=f_x(X,y,1) $$
so, $F_{XY}(X,Y,1)=f_{XY}(X,Y);F_{XX}(X,Y,1)=f_{XX}(X,Y,1)$
\begin{align*}
	H(X,Y,Z) & = \begin{vmatrix}
		F_{xx} & F_{xy} & F_{xz}\\
		F_{yx} & F_{yy} & F_{yz}\\
		F_{zx} & F_{zy} & F_{zz}
	\end{vmatrix}=\begin{vmatrix}
		F_{xx} & F_{xy} & F_{xz}\\
		F_{yx} & F_{yy} & F_{yz}\\
		xF_{xx}+yF_{yx}+zF_{zx} &xF_{xy}+yF_{yy}+zF_{zy} & xF_{xz}+yF_{zy}+zF_{zz}
	\end{vmatrix}\\[2mm]
	&=\begin{vmatrix}
		F_{xx} & F_{xy} & F_{xz}\\
		F_{yx} & F_{yy} & F_{yz}\\
		(n-1)F_x & (n-1)F_y & (n-1)F_z
	\end{vmatrix}=\begin{vmatrix}
		F_{xx} & F_{xy} & zF_{xz}+yF_{xy}+xF_{xx}\\
		F_{yx} & F_{yy} & zF_{yz}+yF_{yy}+xF_{yx}\\
		(n-1)F_x & (n-1)F_y & (n-1)(zF_z+yF_y+xF_x)
	\end{vmatrix}\\[2mm]
	&=\begin{vmatrix}
		F_{xx} & F_{xy} & (n-1)F_x\\
		F_{yx} & F_{yy} & (n-1)F_y\\
		(n-1)F_x & (n-1)F_y & (n-1)nF
	\end{vmatrix}
\end{align*}

So, \begin{align*}
	h(x,y)&=H(x,y,1)=\begin{vmatrix}
f_{xx} & f_{xy} & (n-1)f_x\\
f_{yx} & f_{yy} & (n-1)f_y\\
(n-1)f_x & (n-1)f_y & (n-1)nf
\end{vmatrix}\\[2mm]
&=(n-1)nf[f_{yy}f_{xx}-f_{yx}f_{xy}]-(n-1)^2f_y[f_{xx}f_y-f_xf_{xy}]+(n-1)^2f_x[f_yf_{yx}-f_{yy}f_{x}]\\
&=(n-1)nf[f_{yy}f_{xx}-f_{yx}f_{xy}]-(n-1)^2[f_y^2f_{xx}+f_x^2f_{yy}-2f_xf_yf_{xy}]\quad[\text{as }f_{xy}=f_{yx}]\\
\end{align*}
So, $I(P,f\cap g)=I(P,f\cap h)$\Qed

\textbf{\textit{Claim 4: }}If $P$ is a multiple point then $I(P,f\cap g)\geq 2$

\textbf{\textit{Proof: }}$P\in F\cap H$ $P$ is a multiple point of $F\implies I(P,F\cap H)=I(P,f\cap h)\geq 2$ [as $m_P(f)\geq 2$]. So, $I(P,f\cap g)\geq 2$\Qed

\textbf{\textit{Claim 5: }}$P=(0,0)$ be a simple point of  $f=y+ax^2+bxy+cy^2+dx^3+$ higher terms $y=0$ be the tangent at $P$. $P$ is a flex iff $a=0$; $P$ is an ordinary flex iff $a=0;d\neq 0$

\textbf{\textit{Proof: }}As $P$ is a simple point $O_P(f)$ is a d.v.r and the maximal ideal is generated by $x$. $P$ is a flex iff $ord(y)\geq 3$. So, $$y(1+bx+cy+higher terms)=x^2(a+dx+higher terms)$$so $ord(y)\geq 3\iff a=0$

\parinn
$P$ is an ordinary flex iff $ord(y)= 3$. So, $$y(1+bx+cy+higher terms)=x^2(a+dx+higher terms)$$so $ord(y)= 3\iff a=0;d\neq 0$. Now 
\begin{center}
	\begin{tabular}{ccl}
$f_x$&$=$&$2ax+by+3dx^2+$ other terms\\[1mm]
$f_y$&$=$&$1+bx+2cy+$ other terms\\ [1mm]
$f_{xx}$&$=$&$2a+6dx+$ other terms\\[1mm]
$f_{yy}$&$=$&$2c+$ other terms\\[1mm]
$f_{xy}$&$=$&$b+$ other terms
\end{tabular}
\end{center}
Now 

\begin{center}
	\begin{tabular}{ccl}
$f_xf_yf_{xy}$&$=$&$(2ax+by)b+$ higher terms\\[1mm]
$f_y^2f_{xx}$&$=$&$2a+6dx$+ higher terms\\[1mm]
$f_x^2f_{yy}$ && has no 1 degree terms\\[1mm]
$g$&$=$&$2a+(6d-4ab)x-2b^2 y+$ higher terms
\end{tabular}
\end{center}
Now $$P=(0,0)\in f\cap g \iff a=0 \iff P\text{ is a flex}$$ $I(P,g\cap f)=1\iff $ they do not share tangent at $P\iff d\neq 0\iff P$ is an ordinary flex.
\parinf

\textbf{\textit{Corollary: }}A non singular cubic has 9 flexes all are ordinary. 

\textbf{\textit{Proof: }}Let $F$ be a cubic non singular curve $H$ be its hessian. So $H$ is also cubic. $\sum\limits_P(I(P,F\cap H)=9$


Now let for some $P,I(p.F\cap H)\neq 0$. As $P$ is non singular so $P$ is simple so by the previous theorem $P$ is a flex. It must be ordinary as $F$ is cubic. So, $I(P,H\cap F)=1$. So, there are 9 flexes of $F$
}
\newpage
\begin{problem}{B\'{e}zout's Theorem: 5.24}{p3}
\begin{enumerate}[label=(\alph*)]
	\item \label{p3a} Let (0,1.0) be a flex and $Z=0$ be the tangent at that point. char $k$=0. Show that $F=ZY^2+bYZ^2+cYXZ+$ terms in $X,Z$ find a projective change of co-ordinates  s.t $F$ reduced to a form $Y^2Z=$ cubic in $X,Z$
	\item Show that any irreducible cubic is projectively equivalent to one of the following  $G_1=Y^2Z-X^3$ , $G_2=Y^2Z-X^2(X+1)$ , $G_3=Y^2Z=X(X-Z)(X-\lambda Z)$ where $\lambda\neq 0, 1$
\end{enumerate}
\end{problem}
\solve{\begin{enumerate}[label=(\alph*)]
\item $P=(0,0)$. So, $O_P(F_*)$ is a d.v.r. and generated by $X$. So, $$Z(1+bZ+cX+dXZ+eX^2+fZ^2)=X^3k\quad [\text{as }F\text{ is a cubic}]$$so, \begin{multline*}
	F_*(X,Z)=Z+bZ^2+cXZ+dXZ^2+eX^2Z+fZ^3-kX^3\\ 
	\implies F=ZY^2+bZ^2Y+cXYZ+dXZ^2+X^2Z+fZ^3-kX^3
\end{multline*}Now,  using $Y\to (Y-b/2Z-c/2X)$  and keeping others same. $ ZY^2+bZ^2Y+cXYZ$ becomes $$ZY^2-\frac{b^2Z^3}{4}-\frac{c^2X^2Z}{4}-\frac{bcXZ^2}{4}$$so, we get $F$ to the form $ZY^2=$cubic in $X,Z$\Qed
\parinn

\item Assume char $F$=0. Let $F$ be an irreducible curve. So, it has at most finitely many singular point. 


Let $Q_1=(a,b,c)\in H\cap F$ be a simple point of $F$ and $L$ be its tangent [as $H\cap F$ is infinite]. Let $Q_2=(d,e,f)$ be another point on $L$. So, $\exists$ a projective change of co-ordinates $T$ s.t. $T(Q_1)=(0,1,0),T(Q_2)=(1,0,0)$. So, (0,1,0) a simple point  of $F^T$ and its tangent is $Z=0$. By \hyperref[p:p2]{problem 5.23} it is a flex. So, by \hyperref[p3a]{problem (a)} $F^T$ is projectively equivalent to $Y^2Z-$cubic in $X,Z$. Let $$G=Y^2Z-(X-\lambda_1Z)(X-\lambda_2 Z)(X-\lambda_3 Z)$$
\textbf{\underline{Case 1:}} \ If all $\lambda_i's$ are equal then using $X\to (X+\lambda Z)$ and keeping others unchanged we get $X^3=Y^2Z$
\parinf

\textbf{\underline{Case 2.}} \ If $\lambda=\lambda_1=\lambda_2\neq \lambda_3$. Then using $X\to (X+\lambda Z)$ and keeping others unchanged we get$$Y^2Z=X^2(X+(\lambda-\lambda_3)Z)$$using $Z\to \frac{Z}{\lambda-\lambda_3}$ , $Y\to\sqrt{\lambda-\lambda_3}Y$ keeping $X$ unchanged we get$$Y^2Z=X^2(X+1)\quad [\text{as }\lambda-\lambda_3\neq 0]$$

\textbf{\underline{Case 3.}} \  All $\lambda_i$'s are distinct. Then using $X\to (X+\lambda_1 Z)$ and keeping others unchanged we get $$Y^2Z=X(X-(\lambda_2-\lambda_1)Z)(X-(\lambda_3-\lambda_1)Z)$$again  using $$Z\to \frac{Z}{\lambda_2-\lambda_1},\ Y\to\sqrt{\lambda_2-\lambda_1}Y$$ keeping $X$ unchanged we get $$Y^2Z=X(X-Z)(X-\frac{\lambda_3-\lambda_1}{\lambda_2-\lambda_1}Z)$$and also$$\frac{\lambda_3-\lambda_1}{\lambda_2-\lambda_1}\neq 0,1$$so, it is projectively equivalent to $Y^2Z=X(X-Z)(X-\lambda Z)$ where $\lambda\neq 0, 1$. 
\parinn

So, $F$ is projectively equivalent to $G_1=Y^2Z-X^3$ or $G_2=Y^2Z-X^2(X+1)$ or $G_3=Y^2Z=X(X-Z)(X-\lambda Z)$ where $\lambda\neq 0, 1$
\parinf

\boxed{\textbf{\textit{Remark:}}}


\textbf{\textit{Claim 1: }}$G_1$ has a cusp at $(0,0,1)$. ${G_1}_*(X,Z)=Y^2-X^3$ which has a cusp at $(0,0)$ and the tangent is $Y=0$

\textbf{\textit{Claim 2: }}$G_2$ has a node at $(0,0,1)$. ${G_2}_*(X,Z)=Y^2-X^2-X^3$ which has a node at $(0,0)$ and the tangents are $X+Y=0,X-Y=0$

\textbf{\textit{Claim 3: }}If $F$ is an irreducible conic then $\forall P\in F$ $m_P(F)\leq 2$. If $\exists P\in V(F)$ s.t. $m_P(F)>2$ $\exists$   a projective change of co-ordinates $T$ s.t.$T(P)=(0,1,0)$ $G=F^T$ $G_*=L_1L_2L_3\implies G$ is reducible which is not possible. 
\parinn

So, if $F$ is an irreducible conic then either is non singular or it has cusp or it has a node. Now both claim 1 and problem 5.10 implies that if $F$ has a cusp then it is projectively equivalent to $G_1$. Both claim 2 and problem 5.11 implies that if $F$ has a node then it is projectively equivalent to $G_2$. So, if $F$ is non singular then it is projectively equivalent to $G_3$.
\end{enumerate}
}




\end{document}