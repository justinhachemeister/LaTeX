\documentclass[a4paper, 11pt]{article}
\usepackage{fullpage} % changes the margin
\usepackage[a4paper, total={7in, 10in}]{geometry}
\usepackage{amsmath,mathtools}
\usepackage{amssymb,amsthm}  % assumes amsmath package installed
\usepackage{xcolor}
\usepackage[shortlabels]{enumitem}
\usepackage{hyperref}
\hypersetup{
	colorlinks=true,
	linkcolor=blue,
	filecolor=magenta,      
	urlcolor=blue!70!red,
	pdftitle={Assignment}, %%%%%%%%%%%%%%%%   WRITE ASSIGNMENT PDF NAME  %%%%%%%%%%%%%%%%%%%%
}
\usepackage[most,many,breakable]{tcolorbox}



\definecolor{mytheorembg}{HTML}{F2F2F9}
\definecolor{mytheoremfr}{HTML}{00007B}


\tcbuselibrary{theorems,skins,hooks}
\newtcbtheorem{problem}{Problem}
{%
	enhanced,
	breakable,
	colback = mytheorembg,
	frame hidden,
	boxrule = 0sp,
	borderline west = {2pt}{0pt}{mytheoremfr},
	sharp corners,
	detach title,
	before upper = \tcbtitle\par\smallskip,
	coltitle = mytheoremfr,
	fonttitle = \bfseries\sffamily,
	description font = \mdseries,
	separator sign none,
	segmentation style={solid, mytheoremfr},
}
{p}

% To give references for any problem use like this
% suppose the problem number is p3 then 2 options either 
% \hyperref[p:p3]{<text you want to use to hyperlink> \ref{p:p3}}
%                  or directly 
%                   \ref{p:p3}



%---------------------------------------
% BlackBoard Math Fonts :-
%---------------------------------------

%Captital Letters
\newcommand{\bbA}{\mathbb{A}}	\newcommand{\bbB}{\mathbb{B}}
\newcommand{\bbC}{\mathbb{C}}	\newcommand{\bbD}{\mathbb{D}}
\newcommand{\bbE}{\mathbb{E}}	\newcommand{\bbF}{\mathbb{F}}
\newcommand{\bbG}{\mathbb{G}}	\newcommand{\bbH}{\mathbb{H}}
\newcommand{\bbI}{\mathbb{I}}	\newcommand{\bbJ}{\mathbb{J}}
\newcommand{\bbK}{\mathbb{K}}	\newcommand{\bbL}{\mathbb{L}}
\newcommand{\bbM}{\mathbb{M}}	\newcommand{\bbN}{\mathbb{N}}
\newcommand{\bbO}{\mathbb{O}}	\newcommand{\bbP}{\mathbb{P}}
\newcommand{\bbQ}{\mathbb{Q}}	\newcommand{\bbR}{\mathbb{R}}
\newcommand{\bbS}{\mathbb{S}}	\newcommand{\bbT}{\mathbb{T}}
\newcommand{\bbU}{\mathbb{U}}	\newcommand{\bbV}{\mathbb{V}}
\newcommand{\bbW}{\mathbb{W}}	\newcommand{\bbX}{\mathbb{X}}
\newcommand{\bbY}{\mathbb{Y}}	\newcommand{\bbZ}{\mathbb{Z}}

%---------------------------------------
% MathCal Fonts :-
%---------------------------------------

%Captital Letters
\newcommand{\mcA}{\mathcal{A}}	\newcommand{\mcB}{\mathcal{B}}
\newcommand{\mcC}{\mathcal{C}}	\newcommand{\mcD}{\mathcal{D}}
\newcommand{\mcE}{\mathcal{E}}	\newcommand{\mcF}{\mathcal{F}}
\newcommand{\mcG}{\mathcal{G}}	\newcommand{\mcH}{\mathcal{H}}
\newcommand{\mcI}{\mathcal{I}}	\newcommand{\mcJ}{\mathcal{J}}
\newcommand{\mcK}{\mathcal{K}}	\newcommand{\mcL}{\mathcal{L}}
\newcommand{\mcM}{\mathcal{M}}	\newcommand{\mcN}{\mathcal{N}}
\newcommand{\mcO}{\mathcal{O}}	\newcommand{\mcP}{\mathcal{P}}
\newcommand{\mcQ}{\mathcal{Q}}	\newcommand{\mcR}{\mathcal{R}}
\newcommand{\mcS}{\mathcal{S}}	\newcommand{\mcT}{\mathcal{T}}
\newcommand{\mcU}{\mathcal{U}}	\newcommand{\mcV}{\mathcal{V}}
\newcommand{\mcW}{\mathcal{W}}	\newcommand{\mcX}{\mathcal{X}}
\newcommand{\mcY}{\mathcal{Y}}	\newcommand{\mcZ}{\mathcal{Z}}



%---------------------------------------
% Bold Math Fonts :-
%---------------------------------------

%Captital Letters
\newcommand{\bmA}{\boldsymbol{A}}	\newcommand{\bmB}{\boldsymbol{B}}
\newcommand{\bmC}{\boldsymbol{C}}	\newcommand{\bmD}{\boldsymbol{D}}
\newcommand{\bmE}{\boldsymbol{E}}	\newcommand{\bmF}{\boldsymbol{F}}
\newcommand{\bmG}{\boldsymbol{G}}	\newcommand{\bmH}{\boldsymbol{H}}
\newcommand{\bmI}{\boldsymbol{I}}	\newcommand{\bmJ}{\boldsymbol{J}}
\newcommand{\bmK}{\boldsymbol{K}}	\newcommand{\bmL}{\boldsymbol{L}}
\newcommand{\bmM}{\boldsymbol{M}}	\newcommand{\bmN}{\boldsymbol{N}}
\newcommand{\bmO}{\boldsymbol{O}}	\newcommand{\bmP}{\boldsymbol{P}}
\newcommand{\bmQ}{\boldsymbol{Q}}	\newcommand{\bmR}{\boldsymbol{R}}
\newcommand{\bmS}{\boldsymbol{S}}	\newcommand{\bmT}{\boldsymbol{T}}
\newcommand{\bmU}{\boldsymbol{U}}	\newcommand{\bmV}{\boldsymbol{V}}
\newcommand{\bmW}{\boldsymbol{W}}	\newcommand{\bmX}{\boldsymbol{X}}
\newcommand{\bmY}{\boldsymbol{Y}}	\newcommand{\bmZ}{\boldsymbol{Z}}
%Small Letters
\newcommand{\bma}{\boldsymbol{a}}	\newcommand{\bmb}{\boldsymbol{b}}
\newcommand{\bmc}{\boldsymbol{c}}	\newcommand{\bmd}{\boldsymbol{d}}
\newcommand{\bme}{\boldsymbol{e}}	\newcommand{\bmf}{\boldsymbol{f}}
\newcommand{\bmg}{\boldsymbol{g}}	\newcommand{\bmh}{\boldsymbol{h}}
\newcommand{\bmi}{\boldsymbol{i}}	\newcommand{\bmj}{\boldsymbol{j}}
\newcommand{\bmk}{\boldsymbol{k}}	\newcommand{\bml}{\boldsymbol{l}}
\newcommand{\bmm}{\boldsymbol{m}}	\newcommand{\bmn}{\boldsymbol{n}}
\newcommand{\bmo}{\boldsymbol{o}}	\newcommand{\bmp}{\boldsymbol{p}}
\newcommand{\bmq}{\boldsymbol{q}}	\newcommand{\bmr}{\boldsymbol{r}}
\newcommand{\bms}{\boldsymbol{s}}	\newcommand{\bmt}{\boldsymbol{t}}
\newcommand{\bmu}{\boldsymbol{u}}	\newcommand{\bmv}{\boldsymbol{v}}
\newcommand{\bmw}{\boldsymbol{w}}	\newcommand{\bmx}{\boldsymbol{x}}
\newcommand{\bmy}{\boldsymbol{y}}	\newcommand{\bmz}{\boldsymbol{z}}
\newcommand{\eps}{\epsilon}
\newcommand{\veps}{\varepsilon}
\newcommand{\Qed}{\begin{flushright}\qed\end{flushright}}
\newcommand{\parinn}{\setlength{\parindent}{1cm}}
\newcommand{\parinf}{\setlength{\parindent}{0cm}}
\newcommand{\norm}{\|\cdot\|}
\newcommand{\inorm}{\norm_{\infty}}
\newcommand{\opensets}{\{V_{\alpha}\}_{\alpha\in I}}
\newcommand{\oset}{V_{\alpha}}
\newcommand{\opset}[1]{V_{\alpha_{#1}}}
\newcommand{\lub}{\text{lub}}
\newcommand{\del}[2]{\frac{\partial #1}{\partial #2}}
\newcommand{\Del}[3]{\frac{\partial^{#1} #2}{\partial^{#1} #3}}
\newcommand{\deld}[2]{\dfrac{\partial #1}{\partial #2}}
\newcommand{\Deld}[3]{\dfrac{\partial^{#1} #2}{\partial^{#1} #3}}
\newcommand{\lm}{\lambda}
\newcommand{\uin}{\mathbin{\rotatebox[origin=c]{90}{$\in$}}}
\newcommand{\ueq}{\mathbin{\rotatebox[origin=c]{90}{$=$}}}
\newcommand{\usubset}{\mathbin{\rotatebox[origin=c]{90}{$\subset$}}}
\newcommand{\lt}{\left}
\newcommand{\rt}{\right}
\newcommand{\bs}[1]{\boldsymbol{#1}}
\newcommand{\exs}{\exists}
\newcommand{\st}{\strut}
\newcommand{\dps}[1]{\displaystyle{#1}}

\newcommand{\sol}{\setlength{\parindent}{0cm}\textbf{\textit{Solution:}}\setlength{\parindent}{1cm} }
\newcommand{\solve}[1]{\setlength{\parindent}{0cm}\textbf{\textit{Solution: }}\setlength{\parindent}{1cm}#1 \Qed}
\DeclareRobustCommand{\rchi}{{\mathpalette\irchi\relax}}
\newcommand{\irchi}[2]{\raisebox{\depth}{$#1\chi$}}

\newcommand{\starx}{\textasteriskcentered}
\newcommand{\dnt}{\coloneqq}
\newcommand{\coef}{\bbF[\overline{x}]}
\newcommand{\per}{\text{per}}

\setlength{\parindent}{0pt}

%%%%%%%%%%%%%%%%%%%%%%%%%%%%%%%%%%%%%%%%%%%%%%%%%%%%%%%%%%%%%%%%%%%%%%%%%%%%%%%%%%%%%%%%%%%%%%%%%%%%%%%%%%%%%%%%%%%%%%%%%%%%%%%%%%%%%%%%

\begin{document}

%%%%%%%%%%%%%%%%%%%%%%%%%%%%%%%%%%%%%%%%%%%%%%%%%%%%%%%%%%%%%%%%%%%%%%%%%%%%%%%%%%%%%%%%%%%%%%%%%%%%%%%%%%%%%%%%%%%%%%%%%%%%%%%%%%%%%%%%

\textsf{\noindent \large\textbf{Soham Chatterjee} \hfill \textbf{Assignment - 1}\\
	Email: \href{sohamc@cmi.ac.in}{sohamc@cmi.ac.in} \hfill Roll: BMC202175\\
	\normalsize Course: Calculus \hfill Date: August 29, 2022 \\
	\noindent\rule{7in}{2.8pt}}

%%%%%%%%%%%%%%%%%%%%%%%%%%%%%%%%%%%%%%%%%%%%%%%%%%%%%%%%%%%%%%%%%%%%%%%%%%%%%%%%%%%%%%%%%%%%%%%%%%%%%%%%%%%%%%%%%%%%%%%%%%%%%%%%%%%%%%%%
% Problem 1
%%%%%%%%%%%%%%%%%%%%%%%%%%%%%%%%%%%%%%%%%%%%%%%%%%%%%%%%%%%%%%%%%%%%%%%%%%%%%%%%%%%%%%%%%%%%%%%%%%%%%%%%%%%%%%%%%%%%%%%%%%%%%%%%%%%%%%%%

\begin{problem}{%problem statement
}{p1% problem reference text
}
Is the function $\log  x$ uniformly continuous on $[1,\infty)$? 
%Problem		
\end{problem}

\solve{
	%Solution
	For all $x\in [1,\infty)$ we have $$\log x\leq x-1$$Hence for  $\eps>0$ , $x,y\in [1,\infty)$ ]let $$|\log x - \log y|<\eps \iff \lt|\log\frac{x}{y}\rt|<\eps$$Now if $|x-y|<\delta$ where $\delta>0$ and suppose $x\geq y$ then $x=y+k$ for some $0\geq k<\delta$ then $$\log\frac{x}{y}=\log \lt(1+\frac{k}{y}\rt)\leq \frac{k}{y}\leq \frac{k}{1}<\delta$$Hence if we take $\eps=\delta$ then  for all $x,y\in [1,\infty)$,  if $|x-y|<\delta $ then $$\lt|\log x-\log y\rt|<\eps$$Therefore $\log x$ is uniformly continuous in $[1,\infty)$
}


%%%%%%%%%%%%%%%%%%%%%%%%%%%%%%%%%%%%%%%%%%%%%%%%%%%%%%%%%%%%%%%%%%%%%%%%%
% Problem 2
%%%%%%%%%%%%%%%%%%%%%%%%%%%%%%%%%%%%%%%%%%%%%%%%%%%%%%%%%%%%%%%%%%%%%%%%%

\begin{problem}{%problem statement
}{p2% problem reference text
}
%Problem		
Let us agree that  \emph{a complex-valued function $f=f_r+if_i$ defined on $I$ will be called Riemann integrable} if its real and imaginary parts $f_r,f_i$ are Riemann Integrable; in which case, we define
\[
\int_0^1 f(x) dx  \equiv  \int_0^1 f_r(x) dx + i \int_0^1 f_i(x) dx \ .
\]
Prove that if $f$ is Riemann integrable then $|f|$ is Riemann integrable, and
\[
\left|\int_0^1 f(x) dx\right| \le \int_0^1 |f(x)| dx
\]
\end{problem}

\solve{
	%Solution
	As $\sqrt{f_r^2(x)+f_i^2(x)}\geq 0$ for all $x\in [0,1]$. Now if $\dps{\int_0^1 f_r(x)dx=\int_0^1f_i(x)dx=0}$ then $$\int_0^1|f(x)|dx\geq \lt|\int_0^1f(x)dx\rt|$$If at least one of $\int_0^1 f_r(x)dx,\int_0^1f_i(x)dx$ is nonzero then let $$a=\dfrac{\int_0^1f_r(x)dx}{\sqrt{\lt(\int_0^1f_r(x)dx\rt)^2+\lt(\int_0^1f_i(x)dx\rt)^2}}\qquad b=\dfrac{\int_0^1f_i(x)dx}{\sqrt{\lt(\int_0^1f_r(x)dx\rt)^2+\lt(\int_0^1f_i(x)dx\rt)^2}}$$Therefore $a^2+b^2=1$.  Now by Cauchy Schwarz Inequality $$\sqrt{a^2+b^2}\sqrt{f_r^2(x)+f_i^2(x)} =\sqrt{f_r^2(x)+f_i^2(x)}\geq af_r(x)+bf_i(x)\qquad \forall \ x\in I$$Hence \begin{align*}
		     & a\int_0^1f_r(x)dx+b\int_0^1f_i(x)dx=\int_0^1(af_r(x)+bf_i(x))dx                                      \\
		\iff & a\int_0^1f_r(x)dx+b\int_0^1f_i(x)dx\leq \int_0^1\sqrt{f_r^2(x)+f_i^2(x)}dx                           \\
		\iff & \sqrt{\lt(\int_0^1f_r(x)dx\rt)^2+\lt(\int_0^1f_i(x)dx\rt)^2}\leq \int_0^1 \sqrt{f_r^2(x)+f_i^2(x)}dx \\
		\iff & \lt|\int_0^1f(x)dx\rt|\leq \int_0^1|f(x)|dx
	\end{align*}
	
}


%%%%%%%%%%%%%%%%%%%%%%%%%%%%%%%%%%%%%%%%%%%%%%%%%%%%%%%%%%%%%%%%%%%%%%%%%
% Problem 3
%%%%%%%%%%%%%%%%%%%%%%%%%%%%%%%%%%%%%%%%%%%%%%%%%%%%%%%%%%%%%%%%%%%%%%%%%

\begin{problem}{%problem statement
}{p3% problem reference text
}
%Problem	
The following problems are (with minor changes) taken from Rudin.  Let $p,q$ be positive real numbers satisfying
\[
\frac{1}{p} + \frac{1}{q} =1 
\]
(These  are said to be \emph{conjugate exponents} to each other. Note that $p=2,q=2$ are conjugate. Note also the ``limiting cases'' $p=1,q=\infty$, $p=\infty, q=1$.)
\begin{enumerate}[label=(\alph*)]
	\item If $u \ge 0, \ v \ge 0$, prove that
	\[
	uv \le \frac{u^p}{p}+\frac{v^q}{q}
	\]
	(Hint: Reduce to proving the case when $u=1$ and $0 \le v \le 1$. When $v=0$ or $v=1$, the inequality is clear; now use convexity. )
	
	\item  If $f,g$ are Riemann integrable non-negative functions on $I$, then
	\[
	\int_0^1 fg\ dx  \le \left\{\int_a^b f^p dx\right\}^{\frac{1}{p}} \left\{\int_a^b g^q dx\right\}^{\frac{1}{q}}
	\]
	(Hint: Reduce to the case when both factors on the right are equal to one. Then use (a).)
	
	\item If $f,g$ are complex-valued and Riemann integrable on $I$,
	\[
	\left|\int_0^1 fg\ dx\right|  \le \left\{\int_0^1 |f|^p dx\right\}^{\frac{1}{p}} \left\{\int_0^1 |g|^q dx\right\}^{\frac{1}{q}}
	\]
\end{enumerate} 	
\end{problem}

\solve{
	%Solution
	\begin{enumerate}[label=(\alph*)]
		\item If $u=v=0$ then we are done. Suppose at least one of is nonzero. Let $v^q\geq u^p$. Hence $v\neq 0$. Then \begin{align*}
			     & uv \leq \frac{u^p}{p}+\frac{v^q}{q}                     \\
			\iff & \frac{uv}{v^q} \leq \frac1p\, \frac{u^p}{v^q}+\frac1q   \\
			\iff & \frac{u}{v^{q-1}}\leq \frac1p\, \frac{u^p}{v^q}+\frac1q
		\end{align*}Let $x=\frac{u^p}{v^q}$. Now $$\frac1p+\frac1q=1\iff \frac1p=\frac{q-1}q\iff \frac{q}{p}=q-1$$Hence $x^{\frac1p}=\frac{u}{v^q}$. Hence substituting the values we need to prove $$x^{\frac1p}\leq \frac{x}{p}+\frac1q\qquad \forall\ x\in [0,1]$$Now take the function $f(x)=x^{\frac1p}-\frac{x}{p}$. Now at $x=1$ we have $$f(1)=1-\frac1p=\frac1q$$ and at $x=0$ we have $f(0)=0$. Now $f$ is a differentiable function. $$f'(x)=\frac1p x^{\frac1p-1}-\frac1p=\frac1p\lt(x^{-\frac1q}-1\rt)$$Now since $0\leq x\leq 1$ we have $0\leq x^{\frac1q}\leq 1$ and therefore $x^{-\frac1q}\geq 1$. Hence $\forall \ x\in [0,1]$ we have $f'(x)\geq 0$. Hence $f$ is increasing. Hence the maximum value $f$ attains on $[0,1]$ is $\frac1q$. Hence $$x^{\frac1p}\leq \frac{x}{p}+\frac1q$$Therefore $$uv\leq \frac{u^p}{p}+\frac{v^q}{q}$$\Qed
	\item $A=\lt(\dps{\int_0^1f^p(x)dx}\rt)^{\frac1p}$ and $B=\lt(\dps{\int_0^1g^q(x)dx}\rt)^{\frac1q}$. Now using part (a) we get
	\begin{align*}
		     & \frac{f(x)}{A}\frac{ g(x) }{B}\leq \frac1p\lt(\frac{ f(x) }{A}\rt)^p+\frac1q\lt(\frac{ g(x) }{B}\rt)^q                           \\
		\iff & \int_0^1\frac{ f(x)g(x) }{AB}dx\leq \int_0^1\lt[\frac1p\lt(\frac{ f(x) }{A}\rt)^p+\frac1q\lt(\frac{ g(x) }{B}\rt)^q\rt]dx        \\
		\iff & \frac{ \dps{\int_0^1f(x)g(x)} dx}{AB}\leq \frac1p\frac{\dps{\int_0^1 f^p(x) dx}}{A^p}+\frac1q\frac{\dps{\int_0^1 g^q(x)dx}}{B^q} \\
		\iff & \frac{ \dps{\int_0^1f(x)g(x)} }{AB}\leq \frac1p+\frac1q=1                                                                        \\
		\iff & \int_0^1f(x)g(x)dx \leq AB                                                                                                       \\
		\iff & \int_0^1f(x)g(x)dx \leq  \lt(\dps{\int_0^1f^p(x)dx}\rt)^{\frac1p}\lt(\dps{\int_0^1g^q(x)dx}\rt)^{\frac1q}
	\end{align*}\Qed
	\item $A=\lt(\dps{\int_0^1|f(x)|^pdx}\rt)^{\frac1p}$ and $B=\lt(\dps{\int_0^1|g(x)|^qdx}\rt)^{\frac1q}$. Now using part (a) we get
	\begin{align*}
		     & \frac{|f(x)|}{A}\frac{ |g(x)| }{B}\leq \frac1p\lt(\frac{ |f(x)| }{A}\rt)^p+\frac1q\lt(\frac{ |g(x)| }{B}\rt)^q                         \\
		\iff & \int_0^1\frac{ |f(x)g(x)| }{AB}dx\leq \int_0^1\lt[\frac1p\lt(\frac{ |f(x)| }{A}\rt)^p+\frac1q\lt(\frac{ |g(x)| }{B}\rt)^q\rt]dx        \\
		\iff & \frac{ \dps{\int_0^1|f(x)g(x)|} dx}{AB}\leq \frac1p\frac{\dps{\int_0^1 |f(x)|^p dx}}{A^p}+\frac1q\frac{\dps{\int_0^1 |g(x)|^qdx}}{B^q} \\
		\iff & \frac{ \dps{\int_0^1|f(x)g(x)|} }{AB}\leq \frac1p+\frac1q=1                                                                            \\
		\iff & \int_0^1|f(x)g(x)|dx \leq AB                                                                                                           \\
		\iff & \int_0^1|f(x)g(x)|dx \leq  \lt(\dps{\int_0^1|f(x)|^pdx}\rt)^{\frac1p}\lt(\dps{\int_0^1|g(x)|^qdx}\rt)^{\frac1q}
	\end{align*}Using (\ref{p:p1}) we have $$\lt|\int_0^1f(x)g(x)dx \rt|\leq \int_0^1|f(x)g(x)|dx $$Hence $$\lt|\int_0^1f(x)g(x)dx \rt|\leq \lt(\dps{\int_0^1|f(x)|^pdx}\rt)^{\frac1p}\lt(\dps{\int_0^1|g(x)|^qdx}\rt)^{\frac1q}$$
	\end{enumerate}
}


%%%%%%%%%%%%%%%%%%%%%%%%%%%%%%%%%%%%%%%%%%%%%%%%%%%%%%%%%%%%%%%%%%%%%%%%%
% Problem 4
%%%%%%%%%%%%%%%%%%%%%%%%%%%%%%%%%%%%%%%%%%%%%%%%%%%%%%%%%%%%%%%%%%%%%%%%%

\begin{problem}{%problem statement
}{p4% problem reference text
}
%Problem		
Consider the set $S=\{(x,y)|x,y \in \bbQ\} \subset I^2$. Is $S$ Jordan-measurable? If yes, compute its area. If not \begin{enumerate}[label = (\alph*)]
	\item  show directly that $\rchi_S$ is not integrable \emph{and} \item  show that $\partial S$ does not have content zero.
\end{enumerate}
\end{problem}

\solve{
	%Solution
	 Since $\bbQ\cap I$ is dense in $I$, $\bbQ^2\cap I^2=S $ is dense in $I^2$. FOr any closed rectangle $R$ in $I^2$ we can write $R=I_1\times I_2$ where $I_1,I_2$ are closed intervals in $I$. Then we will always have a point from $\bbQ\cap I$ and a point from $(\bbR\setminus\bbQ)\cap I$ in eaech of $I_1,I_2$. Now We take the closed rectangle $R=I^2$ which covers the whole $S$. Let $P$ be any partition of $I^2$. Since $S$ is dense in $I^2$ for any rectangle $R_i$ we have $S\cap R_i\neq \phi$ and $(I^2\setminus S)\cap R_i\neq \phi$. Hence $M_{R_i}(x)=1,m_{R_i}(x)=0$ for all $x\in R_i$. Hence $$U(P,R)-L(P,R)=\sum_i(M_{R_i}(x)-m_{R_i}(x))Vol(R_i)=\sum_i(1-0)Vol(R_i)=\sum_iVol(R_i)=Vol(R)=1$$for all partitions $P$ of $R$ we have this. Hence $\rchi_S$ is not integrable. Therefore $\partial S$ does not have measure zero.  Hence $S$ is not Jordan-Measurable.
	\begin{enumerate}[label=(\alph*)]
		\item We just show that $\rchi_S$ is not integrable.\Qed
		\item Since $\rchi_S $ is not integrable $\iff$ $\partial S$ is content zero and since $\rchi_S$ is not integrable we have $\partial S$ is not content zero.
	\end{enumerate}
}


%%%%%%%%%%%%%%%%%%%%%%%%%%%%%%%%%%%%%%%%%%%%%%%%%%%%%%%%%%%%%%%%%%%%%%%%%
% Problem 5
%%%%%%%%%%%%%%%%%%%%%%%%%%%%%%%%%%%%%%%%%%%%%%%%%%%%%%%%%%%%%%%%%%%%%%%%%

\begin{problem}{%problem statement
}{p5% problem reference text
}
%Problem		
What about the set 
\[
S=\{(x,y)\}\setminus \underset{n=1,\dots}\bigcup \left\{\left(\frac{1}{n},y\right)\right\}\subset I^2 \ ?
\]
\end{problem}

\solve{
	%Solution
	Since $$S=\lt\{(x,y)\in I^2\rt\}    \setminus \lt\{\lt(\frac1n,y\rt) \mid y\in I,\ n\in \bbN\rt\}=\lt[\{0\}\times I \rt]\cup \lt[\bigcup_{i=1}^{\infty} \lt(\frac{1}{i+1},\frac1i\rt)\times I\rt]$$Hence the   \begin{multline*}
		\partial S=S\cup  \lt\{ \{x\}\times I,I\times \{x\}\mid x=0,1 \rt\}    =\lt\{\lt(\frac1n,y\rt) \mid y\in I,\ n\in \bbN\rt\}\cup  \lt\{ \{x\}\times I,I\times \{x\}\mid x=0,1 \rt\} \\
		=  \lt\{ \{x\}\times I,I\times \{x\}\mid x=0,1 \rt\}\cup\lt[\bigcup_{i=1}^{\infty}\lt\{\frac1i\rt\}\times I\rt]
	\end{multline*}  Now for any $\eps>0$ the set $\lt\{\frac1i\rt\}\times I$ can be covered by the rectangle $\lt[\frac1i-\frac{\eps}{2\times2^{i+1}},\frac1i+\frac{\eps}{2\times2^{i+1}}\rt]$ Hence $$Vol\lt( \lt\{\frac1i\rt\}\times I\rt)\leq Vol\lt(\lt[\frac1i-\frac{\eps}{2\times2^{i+1}},\frac1i+\frac{\eps}{2\times2^{i+1}}\rt]\times I\rt)=\frac{\eps}{2\times2^i}\times 1=\frac{\eps}{2\times2^i}$$Now $$\bigcup_{i=1}^{\infty}\lt\{\frac1i\rt\}\times I\subseteq \bigcup_{i=1}^{\infty}\lt[\frac1i-\frac{\eps}{2\times2^{i+1}},\frac1i+\frac{\eps}{2\times2^{i+1}}\rt]\times I$$Therefore \begin{align*}
		Vol\lt(\bigcup_{i=1}^{\infty}\lt\{\frac1i\rt\}\times I\rt) & \leq Vol\lt(\bigcup_{i=1}^{\infty}\lt[\frac1i-\frac{\eps}{2\times2^{i+1}},\frac1i+\frac{\eps}{2\times2^{i+1}}\rt]\times I\rt)\\
		& =\sum_{i=1}^{\infty} Vol\lt(\lt[\frac1i-\frac{\eps}{2\times2^{i+1}},\frac1i+\frac{\eps}{2\times2^{i+1}}\rt]\times I\rt)\\
		& < \sum_{i=1}^{\infty}\frac{\eps}{2\times2^i}=\frac{\eps}{2}
	\end{align*}Now each $\{0\}\times I,\{1\}\times I,I\times \{0\},I\times \{1\}$ can be covered with respectively  
$ \lt[ -\frac{\eps}{8},\frac{\eps}{8} \rt]\times I  , \lt[1 -\frac{\eps}{8},1+\frac{\eps}{8} \rt]\times I,I\times \lt[ -\frac{\eps}{8},\frac{\eps}{8} \rt],I\times \lt[1 -\frac{\eps}{8},1+\frac{\eps}{8} \rt]$. Therefore there total volume is less than $\frac{\eps}{2}$. Hence $$Vol(\partial S)<\frac{\eps}{2}+\frac{\eps}{2}=\eps$$
Hence $\partial S$ has measure zero. Therefore $S$ is Jordan-Measurable.

We know that for any closed rectangle $R$ if $R^0$ is the interior of $R$ then $\int\rchi_{R^0}=Vol(R)$ i.e. $Vol(R^0)=Vol(R)$. Now in $S$ for each $ \lt(\frac{1}{i+1},\frac1i\rt)\times I$ $$ \lt(\frac{1}{i+1},\frac1i\rt)\times (0,1) \subset  \lt(\frac{1}{i+1},\frac1i\rt)\times I\subset  \lt[\frac{1}{i+1},\frac1i\rt]\times I$$Since $$Vol\lt( \lt(\frac{1}{i+1},\frac1i\rt)\times (0,1) \rt)=Vol\lt(\lt[\frac{1}{i+1},\frac1i\rt]\times I\rt)=\frac{1}{i(i+1)}$$ we have $$Vol\lt(\lt(\frac{1}{i+1},\frac1i\rt)\times I\rt)=\frac1{i(i+1)}$$Now since the set $\{0\}\times I$ is measure zero, $Vol(\{0\}\times I)=0$. Hence $$Vol(S)=\sum_{i=1}^{\infty}Vol\lt(\lt(\frac{1}{i+1},\frac1i\rt)\times I\rt)=\sum_{i=1}^{\infty}\frac1{i(i+1)}=\sum_{i=1}^{\infty}\frac1{i}-\frac1{i+1}=1$$
}


%%%%%%%%%%%%%%%%%%%%%%%%%%%%%%%%%%%%%%%%%%%%%%%%%%%%%%%%%%%%%%%%%%%%%%%%%
% Problem 6
%%%%%%%%%%%%%%%%%%%%%%%%%%%%%%%%%%%%%%%%%%%%%%%%%%%%%%%%%%%%%%%%%%%%%%%%%

\begin{problem}{%problem statement
	}{p5% problem reference text
	}
	%Problem		
	Let $f:I \to \bbR$ be a continuous function, and let $\Gamma_f=\{(x,f(x))|x \in I\} \subset \bbR^2$ be its graph. Show that $\Gamma_f$ has content zero. What if $f$ is only integrable?
\end{problem}

\solve{
	%Solution
	Since $f$ is continuous on a closed interval $f$ is uniformly continuous. Hence for every $\eps>0$, $\exs \ \delta>0$ such that for all $x,y\in I$, $|f(x)-f(y)|<\eps$ whenever $|x-y|<\delta$. Hence there exists $n\in \bbN$ such that $\frac1n\leq \delta<\frac1{n-1}$. Now we partition $I$ into $x_0,x_1,\ldots, x_{n-1},x_n$ where $x_i=\frac{i}{n}$ for $i=0,1,2,\ldots, n$. Now for any closed interval $x,y\in [x_i,x_{i+1}]$ we have $|f(x)-(y)|<\eps$ and therefore $M_{f_i}(x)-m_{f_i}(x)<\eps$. Now the set $\Gamma_f$ can be covered with the rectangles $R_i=\lt[m_{f_i}(x),M_{f_i}(x)\rt]\times [x_i,x_i+1]$ for $i=0,1,\ldots, n-1$. Now for each $R_i$ $$Vol(R_i)< \eps\times \frac1n=\frac{\eps}{n}$$Hence $$Vol(\Gamma_f)\leq \sum_{i=0}^{n-1}Vol(R_i)<\sum_{i=0}^{n-1}\frac{\eps}{n}=\eps$$Hence $\Gamma_f$ has content zero.\Qed
	
	Given that $f$ is integrable. Therefore for all $\eps>0$ there exists a partition $P$ of $I$ such that $$U(P,I)-L(P,I)<\eps$$Let the partition $P=\{x_0,x_1,\ldots,x_n\}$ where $x_0=0$ and $x_n=1$. Now $$U(P,I)-L(P,I)=\sum_{i=0}^{n-1}\lt(M_{f_i}(x)-m_{f_i}(x)\rt)(x_{i+1}-x_i)$$
If We choose the rectangles $\{R_i\}$ such that $R_i=\lt[ m_{f_i}(x), M_{f_i}(x)\rt]\times [x_i,x_{i+1}]$ then $\Gamma_f$ is covered by the union of all rectangles $R_i$. Hence $Vol(\Gamma_f)\leq \sum\limits_{i=0}^{n-1}Vol(R_i)<\eps$. Hence $\Gamma_f$ has content zero.

}
\pagebreak

%%%%%%%%%%%%%%%%%%%%%%%%%%%%%%%%%%%%%%%%%%%%%%%%%%%%%%%%%%%%%%%%%%%%%%%%%
% Problem 7
%%%%%%%%%%%%%%%%%%%%%%%%%%%%%%%%%%%%%%%%%%%%%%%%%%%%%%%%%%%%%%%%%%%%%%%%%

\begin{problem}{%problem statement
	}{p5% problem reference text
	}
	%Problem		
	Let $D:I \to \bbR$ be the function $D(t)=1-t$. 
	\begin{enumerate} 
		\item What is $\int_\bbR \tilde{D}$? (Notation of the notes.)
		\item Compute
		\[
		\int_{I \times I} D(x)D(xy) dxdy
		\]
		Justify the steps, even when you have to just refer to a definition.
	\end{enumerate} 
\end{problem}

\solve{
	%Solution
	\begin{enumerate}[label=(\alph*)]
		\item $\tilde{D}=D\rchi_I$. Since both $D$ and $\rchi_I$ are integrable $\tilde{D}$ is also integrable. Then $$\int_{\bbR}\tilde{D}=\int_{I}\tilde{D}=\int_{I}D\rchi_I=\int_0^1(1-t)dt=\lt[t-\frac{t^2}{2}\rt]_0^1=1-\frac12=\frac12$$\Qed
		\item $D(x)D(xy)$  is continous function on a closed interval $I^2$. Hence $D(x)D(xy)$ is integrable. Now By Fubini's Theorem where $$\mcL(x)=\underline{\int_I}D(x)D(xy)dy,\qquad \mcU(x)=\overline{\int_I}D(x)D(xy)dy$$ we have $$\int_{I\times I} D(x)D(xy)dxdy=\int_I\mcL(x)dx=\int_I\mcU(x)dx$$ Since $D(x)D(xy)$ is continuous we have $\mcL(x)=\mcU(x)=\mcK(x)$. Hence \begin{multline*}
			\mcK(x)=\int_ID(x)D(xy)dy=\int_0^1(1-x)(1-xy)dy=	(1-x)\int_0^1(1-xy)dy\\
			=(1-x)\lt[1-x\frac{y^2}{2}\rt]_0^1=(1-x)\lt(1-\frac{x}{2}\rt)
		\end{multline*}
	Now \begin{align*}
		 \int_I\mcK(x)dx=\int_0^1(1-x)\lt(1-\frac{x}{2}\rt)dx
		=  \int_0^1\lt(1-\frac{3x}{2}+\frac{x^2}{2}\rt)dx
		= \lt[ 1-3\frac{x^2}{4}+\frac{x^3}{6}\rt]_0^1
		= \frac5{12}
	\end{align*}
	\end{enumerate}
}


\end{document}
