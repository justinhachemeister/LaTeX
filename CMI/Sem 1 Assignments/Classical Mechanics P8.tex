\documentclass{article}
\usepackage{fullpage}
\usepackage{amsmath}
\usepackage{amsfonts}
\usepackage{authblk,caption}
\usepackage{titling}
\usepackage{tikz}
\usepackage{hyperref}
\usepackage{graphicx}
\graphicspath{{images/}}
\title{\huge{Classical Mechanics 1, Autumn 2021 CMI \\ Problem set 8\\\hspace{7cm}- Govind S. Krishnaswami}
}
\author{Soham Chatterjee\\Roll: BMC202175}
\date{}
\renewcommand\maketitlehooka{\null\mbox{}\vfill}
\renewcommand\maketitlehookd{\vfill\null}

\setlength{\parindent}{1cm}
\begin{document}
%---------------------------------------
% BlackBoard Math Fonts :-
%---------------------------------------

%Captital Letters
\newcommand{\bbA}{\mathbb{A}}	\newcommand{\bbB}{\mathbb{B}}
\newcommand{\bbC}{\mathbb{C}}	\newcommand{\bbD}{\mathbb{D}}
\newcommand{\bbE}{\mathbb{E}}	\newcommand{\bbF}{\mathbb{F}}
\newcommand{\bbG}{\mathbb{G}}	\newcommand{\bbH}{\mathbb{H}}
\newcommand{\bbI}{\mathbb{I}}	\newcommand{\bbJ}{\mathbb{J}}
\newcommand{\bbK}{\mathbb{K}}	\newcommand{\bbL}{\mathbb{L}}
\newcommand{\bbM}{\mathbb{M}}	\newcommand{\bbN}{\mathbb{N}}
\newcommand{\bbO}{\mathbb{O}}	\newcommand{\bbP}{\mathbb{P}}
\newcommand{\bbQ}{\mathbb{Q}}	\newcommand{\bbR}{\mathbb{R}}
\newcommand{\bbS}{\mathbb{S}}	\newcommand{\bbT}{\mathbb{T}}
\newcommand{\bbU}{\mathbb{U}}	\newcommand{\bbV}{\mathbb{V}}
\newcommand{\bbW}{\mathbb{W}}	\newcommand{\bbX}{\mathbb{X}}
\newcommand{\bbY}{\mathbb{Y}}	\newcommand{\bbZ}{\mathbb{Z}}

%---------------------------------------
% MathCal Fonts :-
%---------------------------------------

%Captital Letters
\newcommand{\mcA}{\mathcal{A}}	\newcommand{\mcB}{\mathcal{B}}
\newcommand{\mcC}{\mathcal{C}}	\newcommand{\mcD}{\mathcal{D}}
\newcommand{\mcE}{\mathcal{E}}	\newcommand{\mcF}{\mathcal{F}}
\newcommand{\mcG}{\mathcal{G}}	\newcommand{\mcH}{\mathcal{H}}
\newcommand{\mcI}{\mathcal{I}}	\newcommand{\mcJ}{\mathcal{J}}
\newcommand{\mcK}{\mathcal{K}}	\newcommand{\mcL}{\mathcal{L}}
\newcommand{\mcM}{\mathcal{M}}	\newcommand{\mcN}{\mathcal{N}}
\newcommand{\mcO}{\mathcal{O}}	\newcommand{\mcP}{\mathcal{P}}
\newcommand{\mcQ}{\mathcal{Q}}	\newcommand{\mcR}{\mathcal{R}}
\newcommand{\mcS}{\mathcal{S}}	\newcommand{\mcT}{\mathcal{T}}
\newcommand{\mcU}{\mathcal{U}}	\newcommand{\mcV}{\mathcal{V}}
\newcommand{\mcW}{\mathcal{W}}	\newcommand{\mcX}{\mathcal{X}}
\newcommand{\mcY}{\mathcal{Y}}	\newcommand{\mcZ}{\mathcal{Z}}



%---------------------------------------
% Bold Math Fonts :-
%---------------------------------------

%Captital Letters
\newcommand{\bmA}{\boldsymbol{A}}	\newcommand{\bmB}{\boldsymbol{B}}
\newcommand{\bmC}{\boldsymbol{C}}	\newcommand{\bmD}{\boldsymbol{D}}
\newcommand{\bmE}{\boldsymbol{E}}	\newcommand{\bmF}{\boldsymbol{F}}
\newcommand{\bmG}{\boldsymbol{G}}	\newcommand{\bmH}{\boldsymbol{H}}
\newcommand{\bmI}{\boldsymbol{I}}	\newcommand{\bmJ}{\boldsymbol{J}}
\newcommand{\bmK}{\boldsymbol{K}}	\newcommand{\bmL}{\boldsymbol{L}}
\newcommand{\bmM}{\boldsymbol{M}}	\newcommand{\bmN}{\boldsymbol{N}}
\newcommand{\bmO}{\boldsymbol{O}}	\newcommand{\bmP}{\boldsymbol{P}}
\newcommand{\bmQ}{\boldsymbol{Q}}	\newcommand{\bmR}{\boldsymbol{R}}
\newcommand{\bmS}{\boldsymbol{S}}	\newcommand{\bmT}{\boldsymbol{T}}
\newcommand{\bmU}{\boldsymbol{U}}	\newcommand{\bmV}{\boldsymbol{V}}
\newcommand{\bmW}{\boldsymbol{W}}	\newcommand{\bmX}{\boldsymbol{X}}
\newcommand{\bmY}{\boldsymbol{Y}}	\newcommand{\bmZ}{\boldsymbol{Z}}
%Small Letters
\newcommand{\bma}{\boldsymbol{a}}	\newcommand{\bmb}{\boldsymbol{b}}
\newcommand{\bmc}{\boldsymbol{c}}	\newcommand{\bmd}{\boldsymbol{d}}
\newcommand{\bme}{\boldsymbol{e}}	\newcommand{\bmf}{\boldsymbol{f}}
\newcommand{\bmg}{\boldsymbol{g}}	\newcommand{\bmh}{\boldsymbol{h}}
\newcommand{\bmi}{\boldsymbol{i}}	\newcommand{\bmj}{\boldsymbol{j}}
\newcommand{\bmk}{\boldsymbol{k}}	\newcommand{\bml}{\boldsymbol{l}}
\newcommand{\bmm}{\boldsymbol{m}}	\newcommand{\bmn}{\boldsymbol{n}}
\newcommand{\bmo}{\boldsymbol{o}}	\newcommand{\bmp}{\boldsymbol{p}}
\newcommand{\bmq}{\boldsymbol{q}}	\newcommand{\bmr}{\boldsymbol{r}}
\newcommand{\bms}{\boldsymbol{s}}	\newcommand{\bmt}{\boldsymbol{t}}
\newcommand{\bmu}{\boldsymbol{u}}	\newcommand{\bmv}{\boldsymbol{v}}
\newcommand{\bmw}{\boldsymbol{w}}	\newcommand{\bmx}{\boldsymbol{x}}
\newcommand{\bmy}{\boldsymbol{y}}	\newcommand{\bmz}{\boldsymbol{z}}

	\maketitle\pagebreak
	\begin{enumerate}
		\item 
			Given that $\ddot{\theta}=-\omega^2 \sin\theta$. Let $f(\theta)=-\omega^2\sin\theta$ Now \begin{align*}
				f(\theta)\ &\approx f(\theta_0)+\left.\frac{d}{d\theta}f(\theta)\right|_{\theta=\theta_0}(\theta-\theta_0)+\frac12\, \left.\frac{d^2}{d\theta^2}f(\theta)\right|_{\theta=\theta_0}(\theta-\theta_0)^2\\
			\end{align*}Given that $\theta_0=\pi$. Hence \begin{align*}
			f(\theta)\ &\approx f(\pi)+\left.\frac{d}{d\theta}f(\theta)\right|_{\theta=\pi}(\theta-\pi)+\frac12\, \left.\frac{d^2}{d\theta^2}f(\theta)\right|_{\theta=\pi}(\theta-\pi)^2\\
			& = -\omega^2\sin\pi+(-\omega^2\cos\pi)(\theta-\pi)+\frac12 (\omega^2\sin\pi)(\theta-\pi)^2\\
			&=-\omega^2\cos(\pi)(\theta-\pi)\\
			&=\omega^2(\theta-\pi)
		\end{align*}\begin{enumerate}\item Putting $\theta=\pi+\phi$ we get $$\dot{\theta}=\dot{\phi}\implies  \ddot{\theta}=\ddot{\phi}$$Hence $$f(\theta)=f(\pi+\phi)=\omega^2(\pi+\phi-\pi)=\omega^2\phi=\frac{g}{l}\phi$$Therefore we get $$\ddot{\phi}=\frac{g}{l}\phi$$
		\item Suppose the solution of the equation $\ddot{\phi}=\omega^2\phi $ will be proportional to $e^{\lambda t}$ for some real number $\lambda\in\mathbb{R}$. Hence $\phi=e^{\lambda t}$. Therefore$$\frac{d^2}{dt^2}(e^{\lambda t}-\omega^2e^{\lambda t})\implies \lambda^2e^{\lambda t}-\omega^2e^{\lambda t}=0\implies (\omega+\lambda)(\omega-\lambda)e^{\lambda t}=0$$As $e^{\lambda t}\neq 0$ the solutions of $\lambda $ are $\omega $ and $-\omega$.
		\setlength{\parindent}{1cm}
		
		Hence the linearly solutions for $\phi$ are $c_1e^{\omega t}$ and $c_2e^{-\omega t}$  where $c_1,c_2$ are two real numbers. Hence the general solution of $\phi$ is $$\phi(t)=c_1e^{\sqrt{\frac{g}{l}} t}+c_2e^{-\sqrt{\frac{g}{l}} t}$$Now $\phi(0)=c_1+c_2$ and $\dot{\phi}(0)=\omega(c_1-c_2)$ hence $c_1=\frac12(\phi(0)+\frac{\dot{\phi}(0)}{\omega})$ and $c_2=\frac12(\phi(0)-\frac{\dot{\phi}(0)}{\omega})$.
		
		\hspace{1cm}Therefore the general solution of the equation is $$\phi(t)=c_1e^{\sqrt{\frac{g}{l}} t}+c_2e^{-\sqrt{\frac{g}{l}} t}$$ where $c_1=\frac12(\phi(0)+\frac{\dot{\phi}(0)}{\omega})$ and $c_2=\frac12(\phi(0)-\frac{\dot{\phi}(0)}{\omega})$.
		\item $\phi$ is positive when it rotates counter-clock wise and negative when it rotates clockwise. Therefore\begin{itemize}
			\item When $c_1=c_2=0$ we have $\phi(0)=\dot{\phi}(0)=0$. Hence $\theta(t)=\pi$  for all $t$. Hence it is a static solution.
			\item When $c_1=0,c_2>0$ we have $\phi(0)>0$ and $\dot{\phi}(0)<0$. Hence initial angle is positive and it is decreasing i.e. the bob is rotating clockwise and approaches $\phi=0$ but never reaches it in finite time.
			\item When $c_1=0,c_2<0$ we have $\phi(0)<0$ and $\dot{\phi}(0)>0$. Hence initial angle is negative and it is increasing i.e. the bob is rotating counter-clockwise and approaches $\phi=0$ but never reaches it in finite time.
			%%%%%%%%%%%%%%%%%%%%%%%%%%%%%%%%%%%%%%%%%%%%%%%%%%%%%%%%%%%%%%%%%%%%
			\item When $c_1>0, c_2=0$ we have $\phi(0)>0$ and $\dot{\phi}(0)>0$. hence initial angle is positive and it is increasing i.e. the bob is rotating counter-clockwise.
			\item When $c_1>0, c_2> 0$ and $c_1=c_2$ we have $\phi(0)>0$ and $\dot{\phi}(0)=0$. Hence initial angle is positive and  the bob will start rotating counter-clockwise
			\item When $c_1>0, c_2> 0$ and $c_1> c_2$ we have $\phi(0)>0$ and $\dot{\phi}(0)> 0$. Hence initial angle is positive and it is increasing i.e. the bob is rotating counter-clockwise.
			\item When $c_1>0, c_2> 0$ and $c_1< c_2$ we have $\phi(0)>0$ and $\dot{\phi}(0)<0$. Hence initial angle is positive and it is decreasing i.e. the bob  is rotating clockwise  and eventually angle becomes negative.
			\item When $c_1>0, c_2<0$ and $|c_1|= |c_2|$ we have $\phi(0)=0$ and $\dot{\phi}(0)> 0$. Hence initial angle is zero and it keeps increasing i.e. the bob is rotating counter-clockwise.
			\item When $c_1>0, c_2<0$ and $|c_1|> |c_2|$ we have $\phi(0)>0$ and $\dot{\phi}(0)> 0$. Hence initial angle is positive and it keeps increasing i.e. the bob is rotating counter-clockwise.
			\item When $c_1>0, c_2<0$ and $|c_1|< |c_2|$ we have $\phi(0)<0$ and $\dot{\phi}(0)> 0$. Hence initial angle is negative and it keeps increasing i.e. the bob is rotating counter-clockwise  and eventually angle becomes positive.
			%%%%%%%%%%%%%%%%%%%%%%%%%%%%%%%%%%%%%%%%%%%%%%%%%%%%%%%%%%%%%%%%%%%%
			\item When $c_1<0,c_2=0$ we have $\phi(0)<0$ and $\dot{\phi}(0)<0$. hence initial angle is negative and it is decreasing i.e. the bob is rotating clockwise.
			\item When $c_1<0,c_2>0$ and $|c_1|=|c_2|$ we have $\phi(0)=0$ and $\dot{\phi}(0)<0$. Hence initial angle is zero and it is decreasing i.e. the bob is rotating clockwise.
			\item When $c_1<0,c_2>0$ and $|c_1|>|c_2|$ we have $\phi(0)<0$ and $\dot{\phi}(0)<0$. Hence initial angle is negative and it is decreasing i.e. the bob is rotating clockwise.
			\item When $c_1<0, c_2> 0$ and $|c_1|< |c_2|$ we have $\phi(0)> 0$ and $\dot{\phi}(0)<0$. Hence initial angle is positive and it is  decreasing i.e. the bob  is rotating clockwise and eventually angle becomes negative.
			\item When $c_1<0, c_2< 0$ and $c_1=c_2$ we have $\phi(0)<0$ and $\dot{\phi}(0)= 0$. Hence initial angle is negative and the bob will start rotating  clockwise
			\item When $c_1<0, c_2< 0$ and $c_1>c_2$ we have $\phi(0)<0$ and $\dot{\phi}(0)>0$. Hence initial angle is negative and it keeps increasing i.e. the bob is rotating counter-clockwise  and eventually angle becomes positive.
			\item When $c_1<0, c_2< 0$ and $c_1<c_2$ we have $\phi(0)<0$ and $\dot{\phi}(0)< 0$. Hence initial angle is negative and it keeps decreasing i.e. the bob is rotating clockwise.
		
		\end{itemize}
		\end{enumerate}
		\item \begin{enumerate}
			\item Gravitational Forces acting between
			\begin{itemize}
				\item point masses $m_e,m_m$ is $\dfrac{G\, m_e\,m_m}{|\bmr_e-\bmr_m|^2}$ 
				\item point masses $m_e,m_s$ is $\dfrac{G\, m_e\,m_s}{|\bmr_e-\bmr_s|^2}$ 
				\item point masses $m_m,m_s$ is $\dfrac{G\, m_m\,m_s}{|\bmr_m-\bmr_s|^2}$ 
			\end{itemize}
		Hence for the particle with mass $m_e$ Newton’s 2nd law equations of motion is \begin{align*}
			&m_e\ddot{\bmr}_e=-\dfrac{G\, m_e\,m_m}{|\bmr_e-\bmr_m|^3}(\bmr_e-\bmr_m)-\dfrac{G\, m_e\,m_s}{|\bmr_e-\bmr_s|^3}(\bmr_e-\bmr_s)\\
			\implies &\ddot{\bmr}_e=\dfrac{G\, m_m}{|\bmr_e-\bmr_m|^3}(\bmr_m-\bmr_e)+\dfrac{G\, m_s}{|\bmr_e-\bmr_s|^3}(\bmr_s-\bmr_e)
		\end{align*}
		For the particle with mass $m_m$ Newton’s 2nd law equations of motion is \begin{align*}
			&m_m\ddot{\bmr}_m=-\dfrac{G\, m_m\,m_e}{|\bmr_m-\bmr_e|^2}(\bmr_m-\bmr_e)-\dfrac{G\, m_m\,m_s}{|\bmr_m-\bmr_s|^2}(\bmr_m-\bmr_s)\\
			\implies & \ddot{\bmr}_m=\dfrac{G\, m_e}{|\bmr_m-\bmr_e|^2}(\bmr_e-\bmr_m)+\dfrac{G\, m_s}{|\bmr_m-\bmr_s|^2}(\bmr_s-\bmr_m)
		\end{align*}
		For the particle with mass $m_s$ Newton’s 2nd law equations of motion is \begin{align*}
			& m_s\ddot{\bmr}_s=-\dfrac{G\, m_s\,m_e}{|\bmr_s-\bmr_e|^2}(\bmr_s-\bmr_e)-\dfrac{G\, m_s\,m_m}{|\bmr_s-\bmr_m|^2}(\bmr_s-\bmr_m)\\
			\implies&\ddot{\bmr}_s=\dfrac{G\,m_e}{|\bmr_s-\bmr_e|^2}(\bmr_e-\bmr_s)
			+\dfrac{G\,m_m}{|\bmr_s-\bmr_m|^2}(\bmr_m-\bmr_s)
		\end{align*}
	\item Now \begin{align*}
	&	m_e\ddot{\bmr}_e+m_m\ddot{\bmr}_m+m_s\ddot{\bmr}_s=\Bigg[-\dfrac{G\, m_e\,m_m}{|\bmr_e-\bmr_m|^3}(\bmr_e-\bmr_m)-\dfrac{G\, m_e\,m_s}{|\bmr_e-\bmr_s|^3}(\bmr_e-\bmr_s)\Bigg]\\
	&\qquad\qquad\qquad\qquad\qquad\qquad+\Bigg[-\dfrac{G\, m_m\,m_e}{|\bmr_m-\bmr_e|^2}(\bmr_m-\bmr_e)-\dfrac{G\, m_m\,m_s}{|\bmr_m-\bmr_s|^2}(\bmr_m-\bmr_s)\Bigg]\\
	&\qquad\qquad\qquad\qquad\qquad\qquad+\Bigg[-\dfrac{G\, m_s\,m_e}{|\bmr_s-\bmr_e|^2}(\bmr_s-\bmr_e)-\dfrac{G\, m_s\,m_m}{|\bmr_s-\bmr_m|^2}(\bmr_s-\bmr_m)\Bigg]\\
	\implies &\frac{d}{dt}(m_e\dot{\bmr}_e+m_m\dot{\bmr}_m+m_s\dot{\bmr}_s)=0\\
	\implies & m_e\dot{\bmr}_e+m_m\dot{\bmr}_m+m_s\dot{\bmr}_s=\text{Constant}
	\end{align*}Therefore total linear momentum of the system is conserved.
		\end{enumerate}
	\item \begin{enumerate}
		\item Magnitude of Relativistic effect on the tennis ball served at  the speed $v=100$ km/h $ =\dfrac{100\times 10^3}{3600}$m/s $=\dfrac{1000}{36} $m/s is  $$\Bigg(\frac{\frac{1000}{36}}{3\times 10^8}\Bigg)^2=\Bigg(\frac{1}{108\times 10^5}\Bigg)^2\approx 8.57\times 10^{-15}$$
		\item  Magnitude of Relativistic effect of Earth's motion at a speed of 30 km/s=$3\times 10^4$ m/s around the sun is $$\Bigg(\frac{3\times 10^4}{3\times 10^8}\Bigg)^2=(10^{-4})^2=10^{-8}$$
	\end{enumerate}
	\end{enumerate}
\end{document}